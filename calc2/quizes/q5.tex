\documentclass[a4paper]{article}

\usepackage[utf8]{inputenc}
\usepackage[T1]{fontenc}
\usepackage{textcomp}
\usepackage[english]{babel}
\usepackage{amsmath, amssymb}
\title{Quiz #5}
\author{yawnbo}
\date{\today}

\maketitle

\pdfsuppresswarningpagegroup=1

\begin{document}
\section{Create a partial fraction decomposition for $\frac{1}{x^2-9}$}%
\label{sec:Question 1}

\[
\frac{1}{6\left( x-3 \right) } - \frac{1}{6\left( x+3 \right) }
\] 

\section{Create a partial fraction decomposition for $\frac{1}{\left( x-1 \right) ^2\left( x^2+9 \right) }$}%
\[
  -\frac{1}{50\left( x-1 \right) }+\frac{1}{10\left( x-1 \right) ^2}+ \frac{x-4}{50\left( x^2+9 \right) }
.\] 
\newpage
\section{Evaluate the following integral}%

\[
  \int_{}^{} \frac{x^{4}+x^3-2x^2-x}{x^2+x-2}dx
.\] 
\subsection{Rewrite the improper rational function}%
\label{sub:Rewrite the improper rational function}
\[
\int_{}^{} \left(   x^2-\frac{x}{x^2+x-2}\right)dx
.\] 
\subsection{Construct a partial on the remainder}%
\label{sub:Construct a partial}
\paragraph{Rewrite in an intermediate}
\[
\frac{x}{x^2+x-2}=\frac{A}{x+2}+\frac{B}{x-1}
.\] 
\paragraph{Get rid of denominators and evaluate}
\[
x=A(x-1)+B(x+2)
.\] 
\paragraph{Giving us $A=\frac{2}{3}$ and $B=\frac{1}{3}$ so our partial looks like,}
\[
\frac{2}{3\left( x+2 \right) }+\frac{1}{3\left( x-1 \right) }
.\] 
\subsection{Integrate}%
\label{sub:Integrate}
\[
  \int_{}^{} \left(   x^2-\frac{2}{3\left( x+2 \right) }-\frac{1}{3\left( x-1 \right) }\right)dx
.\] 
\paragraph{Distribute and integrate}
\begin{align*}
  \int_{}^{} x^2dx &\implies \frac{x^3}{3} \\
  -\int_{}^{} \frac{2}{3\left( x+2 \right) }dx&\implies -\frac{2}{3}\ln^{}( \left\| x+2 \right\| )\\
  -\int_{}^{} \frac{1}{3\left( x-1 \right) }dx&\implies -\frac{1}{3}\ln^{}( \left\| x-1 \right\| )
.\end{align*}
\paragraph{Giving us,}

\begin{gather*}
  \frac{x^3}{3} -\frac{2}{3}\ln^{}( \left\| x+2 \right\| ) -\frac{1}{3}\ln^{}( \left\| x-1 \right\| )+C \\
  \text{ or }\\
  \frac{1}{3}\left( x^3-2\ln^{}( \left\| x+2 \right\| )-\ln^{}( \left\| x-1 \right\| ) \right) +C
\end{gather*}
\newpage
\section{For which values of a will $\int_{0}^{\infty} e^{ax}dx$ converge?}%
\paragraph{First just evaluate the integral}
\[
\lim_{L \to \infty} \int_{0}^{L} e^{ax}dx \implies \lim_{L \to \infty} \left( \frac{1}{a}e^{La} - \frac{1}{a}\right) 
.\] 
\paragraph{Start by noticing that $a\neq 0$ and that $\frac{1}{a}e^{La}$ goes to infinity when $a>0$, but when $a<0$, $e^{La}$ goes to 0 as L approaches $\infty$, so we get the finite value of $-\frac{1}{a}$ when $a<0$.}

\section{Find the Laplace transform for $f\left( x \right) =\cos^{}( ax )$}%
these take WAY too long
\paragraph{Start with our general Laplace transform}
\[
\int_{0}^{\infty} e^{-sx}\cos^{}( ax )dx
.\] 
Do our first IBP by letting
\begin{align*}
  u &= \cos^{}( ax ) \\
  du &= -a\sin^{}( ax ) \\
  v &= -\frac{1}{s}e^{-sx} \\
  dv &= e^{-sx}dx \\
\end{align*}
Rewriting our integral we get
\[
\lim_{T \to \infty} \left(  -\frac{1}{s}e^{-sx}\cos^{}( ax )\big|^{T}_{0}-\int_{0}^{T} -\frac{1}{s}e^{-sx}a\sin^{}( ax )dx\right) 
.\] 
I don't want to clutter this so lets simplify the part before the integral and rewrite the integral by pulling out constants
\begin{align*}
  -\frac{1}{s}e^{-s\left( T \right) }\cos^{}( a\left( T \right) ) &\to \text{ DNE }\\
  -\frac{1}{s}e^{-s\left( 0 \right) }\cos^{}( a\left( 0 \right)  ) &\to \frac{1}{s}
\end{align*}
\[
  \lim_{T \to \infty} \left(  \frac{1}{s}+\int_{0}^{T}\frac{1}{s}e^{-sx}\sin^{}( ax )dx \right) 
.\] 
Again just go back to doing another IBP and rewrite
\begin{align*}
  u &= \sin^{}( ax ) \\
  du &= a\cos^{}( ax ) \\
  v &= -\frac{1}{s}e^{-sx} \\
  dv &= e^{-sx}dx 
\end{align*}
\[
\lim_{T \to \infty} \left( \frac{1}{s}-\frac{a}{s} \left( -\frac{1}{s}e^{-sx}\sin^{}( ax )\big|^{T}_{0}-\int_{0}^{T} -\frac{1}{s}e^{-sx}a\cos^{}( ax )dx \right)  \right) 
.\] 
Again, evaluate inner bounds to find
\[
-\frac{1}{s}e^{-sx}\sin^{}( ax )\big|^{T}_{0} = \text{ DNE }-0
.\] 
Simplify inner integral too and rewrite
\newpage
\[
  \lim_{T \to \infty} \left( \frac{1}{s}-\frac{a}{s}\left( \frac{a}{s}\int_{0}^{T} e^{-sx}\cos^{}( ax )dx \right)  \right) 
.\] 
As a last step simplify the inside and set equal to original expression.
\[
\lim_{T \to \infty}\int_{0}^{T} e^{-sx}\cos^{}( ax )dx=\lim_{T \to \infty} \left( \frac{1}{s}-\frac{a^2}{s^2}\int_{0}^{T} e^{-sx}\cos^{}( ax )dx \right) 
.\] 
Limits cancel and we are left with simple algebra (let $L$ be our integral I'm not writing it 10 times)
\begin{align*}
  &L = \frac{1}{s}-\frac{a^2}{s^2}L \\
  &L\left( 1+\frac{a^2}{s^2} \right) = \frac{1}{s} \\
  &L = \frac{s}{s^2+a^2}.
\end{align*}

\section{Gamma function question 1}%
\label{sec:Gamma function question 1}
\[
\Gamma\left( n \right) =\int_{0}^{\infty} t^{n-1}e^{-t}dt
.\] 
\subsection{Use IBP to prove $\Gamma\left( n+1 \right) =n\Gamma\left( n \right) $}%
Rewrite our integral with $n+1$
\[
  \lim_{p \to \infty} \int_{0}^{p} t^{n}e^{-t}dt
.\] 
Let our variables for IBP be 
\begin{align*}
  u =t^{n} &\implies du=nt^{n-1}dt \\
  v= -e^{-t}&\implies dv =e^{-t}dt \\
\end{align*}
so,
\[
-e^{-t}t^{n}-\int_{0}^{p} -e^{-t}nt^{n-1}dt
.\] 
Simplifying the left side and the integral gets us to
\[
0+n\int_{0}^{p}e^{-t}t^{n-1}dt 
.\] 
Which can be rewritten in terms of the gamma function as 
\[
\Gamma\left( n+1 \right) =n\Gamma\left( n \right) 
.\] 
\subsection{Show that if you assume $n$ is an integer,\\ then $\Gamma\left( n+1 \right) =n!$}%
Let's assume $n=1$.
\[
\Gamma\left( 2 \right) =1! = 1
\] 
\[
\Gamma\left( 2 \right) =\int_{0}^{\infty} t^{0}e^{-t}dt = \int_{0}^{\infty} e^{-t}dt \implies \left( -e^{-\infty} \right) -\left( -e^{0} \right) =1
\]

\section{Find the exact value of $\Gamma\left( \frac{1}{2} \right) $. Guess what power of $\pi$ matches this approximation.}%
We can approximate this at 100 for the upper limit to find the equation,
\[
  \lim_{L \to 0^{+}} \int_{L}^{100} \frac{1}{\sqrt{t}}e^{-t}dt
.\] 
Using some wizardry this becomes 
\[
\sqrt{\pi^{}}\int_{}^{} \frac{2e^{-u^2}}{\sqrt{\pi}} \to \sqrt{\pi}\text{erf}\left( 100\right) 
.\] 
Which when plugged into a calculator returns $\approx 1.772$ which is the exact value of $\sqrt{\pi}$ because the error function evaluates to 1 as it approaches infinity. 


\section*{Bonus question, generalization of type II integral.}%
Starting with the general integral
\[
\int_{0}^{a>0} \frac{dx}{x^{p}}
.\] 
We can find that our denominator cannot be 0 so, our lower bound needs to be evaluated as a limit and we need to rewrite the integral.
\[
\lim_{T \to 0^{+}} \int_{T}^{a} x^{-p}dx \implies \lim_{T \to 0^{+}} \left( \frac{x^{-p+1}}{-p+1} \right) \Big|_{T}^{a}
.\] 
Now we plug in values and evaluate our new expression
\[
\lim_{R \to 0^{+}} \left( \frac{a^{-p+1}}{1-p}- \frac{R^{-p+1}}{1-p} \right) 
.\] 
The limit cancels one of our terms and we are left with
\[
\frac{a^{-p+1}}{1-p}
.\] 

\end{document}
