\documentclass[a4paper]{article}

\usepackage[utf8]{inputenc}
\usepackage[T1]{fontenc}
\usepackage{textcomp}
\usepackage[english]{babel}
\usepackage{amsmath, amssymb}
\title{Quiz 4 questions}
\author{yawnbo}
\date{\today}

\maketitle

\pdfsuppresswarningpagegroup=1

\begin{document}
\section{Question 1}
\[
  \tan^{-1}(2x)dx
\]
Start by letting $u=\tan^{-1}(2x)$ and $dv=dx$, so $du=\frac{2}{1+4x^2}$ and $v=x$ \\
We can rewrite the integral as
\[
x\tan^{-1}(2x)-\int_{}^{} \frac{2}{1+4x^2}\cdot xdx
.\]
Where we can now work on the inner integral. Let
\begin{gather*}
  u=1+4x^2 \implies du=8xdx \\
  dx= \frac{1}{8x}du\\
\end{gather*}
The $x$ 's will cancel out and leave us with
\[
-\frac{2}{8}\int_{}^{} \frac{1}{u}du 
.\] 
Which simplifies to
\[
\frac{1}{4}(\ln(1+4x^2))+C
.\] 
At which point we can plug it back into the original expression
\[
x\tan^{-1}(2x)-\frac{1}{4}(\ln(1+4x^2))+C
.\]
\newpage
\section{Question 2}
\[
\int_{0}^{t} e^{8} \sin^{}(t-s)ds
.\]

Begin by simplifying the inside of sin and the exponent of e
\begin{gather*}
  w=t-s \implies dw=-ds \\
  s=t-w \\
  t-0=t \text{ & } t-t=0
\end{gather*}
We need to change the limits and rewrite as
\[
  \int_{t}^{0} e^{t-w}\sin^{}(w)\cdot -dw&= \int_{0}^{t} e^{t-w}\sin^{}(w)dw  \\ 
.\] 
Now we begin to integrate by parts
\begin{gather*}
  u=\sin^{}(w) \implies du=\cos^{}(w)dw \\
  dv=e^{t-w} \implies v=-e^{t-w} 
\end{gather*}
Which gets us to the equation
\[
-\sin^{}(t)-\int_{0}^{t}-e^{t-w}\cos^{}(w)dw
.\]
Doing another IBP will get us to the final step of cyclic equations, so,
\begin{gather*}
  \text{For }\int_{0}^{t} -e^{t-w}\cos^{}(w)dw \\
  u=\cos^{}(w)\implies du=-\sin^{}(w)dw \\
  dv=-e^{t-w}\implies v=e^{t-w} \\
  \cos^{}(t) - e^{t}-\int_{0}^{t} e^{t-w}\cdot -\sin^{}(w)dw =\cos^{}(t) - e^{t} +\int_{0}^{t} e^{t-w}\sin^{}(w)dw \\
\end{gather*}
Which we set equal to the original equation,
\begin{align*}
  \int_{0}^{t} e^{t-w} \sin^{}(w)dw &= -\sin^{}(t)-\cos^{}(t)+e^{t}-\int_{0}^{t} e^{t-w}\sin^{}(w)dw \\
  2\int^{t}_{0}e^{t-w}\sin^{}(w)dw &= -\sin^{}(t)-\cos^{}(t)+e^{t}
\end{align*}
Which finally gets us
\begin{gather*}
  \int_{0}^{t} e^{s}\sin^{}(t-s)ds= \frac{-\sin^{}(t)-\cos^{}(t)-e^{t}}{2} \\
  \text{Or} \\
\int_{0}^{t} e^{s}\sin^{}(t-s)ds= -\frac{\sin^{}(t)+\cos^{}(t)+e^{t}}{2}
\end{gather*}
\newpage
\section{Question 3}

Find the exact volume obtained by rotating the region bounded by $y=\sin^{2}(x),y=0, \text{ For } x \epsilon[0,\pi^{}]$about the $x-$ axis \\
\\
Start by writing the integral for the area, so,
\[
\pi \int_{0}^{\pi}r^2 \implies \pi \int_{0}^{\pi} \sin^{4}(x)dx 
.\] 
Split the large even power and apply power reduction formulas
\begin{gather*}
\pi^{}\int_{0}^{\pi^{}} \sin^{2}(x)\cdot \sin^{2}(x)dx \\
\pi^{}\int_{0}^{\pi^{}} \frac{1-\cos^{}(2x)}{2}\cdot \frac{1-\cos^{}(2x)}{2}dx
\end{gather*}

Bring constant to the outside and simplify
\[
  \frac{\pi}{4}\int_{0}^{\pi} 1-2\cos^{}(2x)+\cos^{2}(2x)dx
.\] 
Reduce powers again and bring out another $\frac{1}{2}$ 
\[
\frac{\pi}{8}\int_{0}^{\pi } 2-4\cos^{}(2x)+1+\cos^{}(4x)dx
.\]
We can now integrate nicely
\begin{align*}
  &= \frac{\pi}{8}\left( 3x-4\sin^{}(2x)+x+\sin^{}(4x) \right)^{4}_{0} \\
  &= \frac{\pi}{8}[3\pi-0-0] \\
 V &= \frac{3\pi^2}{8} 
.\end{align*}
\newpage
\section{}
The parabola $y=\frac{1}{2}x^2$ divides the disk $x^2+y^2\le 8$ into two regions, determine the exact value of each region. \\
\\
Start by finding the intersections of the functions
\[
x^2+\frac{1}{4}x^{4}=8 \to 4x^2+x^{4}=32
.\] 
Let $u=x^2$ 
\begin{gather*}
u^2+4u-32=0 \\
(u+8)(u-4) \\
x^2 = 4\text{ and } x^2=-8 \\
\end{gather*}
We find intersections at $x=2$ and $x=\sqrt{-8}$ which can be ignored as it's not a real number, so we set up the integral as
\begin{align*}
  R_1 &= \int_{-2}^{2} \left(\sqrt{8-x^2}-\frac{1}{2}x^2\right)dx  \\
  &= 2\left(\int_{0}^{2} \sqrt{8-x}dx -\int_{0}^{2} \frac{1}{2}x^2dx \right) \\
\end{align*}
Where we can now integrate the two separate integrals,

\subsection{Trig integral}%
\label{sub:Trig integral}
Using $\sqrt{a^2-x^2}$ where $x=a\sin^{}(\theta)$ and $dx=a\cos^{}(\theta)d\theta$ change limits to be 
\[
  \sqrt{8}\sin^{}(0) = 0 \text{ and } \sqrt{8}\sin^{}(2) \to \sqrt{8}\sin^{-1}(x)=2 \to \sin^{-1}(x)=\frac{2}{2\sqrt{2}}
.\] 
\[
\sin^{-1}(x)=\frac{1}{\sqrt{2}}\implies x=\frac{\pi}{4}
.\] 
rewrite the integral as, 
\begin{gather*}
\int_{0}^{\frac{\pi}{4}} \sqrt{8-8\sin^{}(\theta)}\sqrt{8}\cos^{}(\theta)d\theta \\
\end{gather*}
\subsubsection{}
Side work for the inside of the root, 
\[
  \sqrt{8(1-\sin^{2}(\theta))}=\sqrt{8\cos^{2}(\theta)}=\sqrt{8}\cos^{}(\theta)
.\] 
so,
\[
\int_{0}^{\frac{\pi}{4}} \sqrt{8}\cos^{}(\theta)\sqrt{8}\cos^{}(\theta)d\theta 
\]
\[
8 \int_{0}^{\frac{\pi}{4}} \cos^{2}(\theta)d\theta 
.\] 
\newpage
\begin{gather*}
8 \int_{0}^{\frac{\pi}{4}} \frac{1+\cos^{}(2\theta)}{2}d\theta \\
4 \int_{0}^{\frac{\pi}{4}} 1+\cos^{}(2\theta)d\theta \\
4\left[\theta + \frac{1}{2}\sin^{}(2\theta)\right]_{0}^{\frac{\pi}{4}} = \pi + 2
\end{gather*}
\subsubsection{Power integral}
\[
\int_{0}^{2} \frac{1}{2}x^2dx \to \frac{1}{6}(2)^3= \frac{4}{3} 
.\] 
\subsection{Main integral}%
\label{sec:Main integral}

This can now be brought back into the integral for $R_1$ so,
\[
2(\pi+2 - \frac{4}{3}) = 2(\pi+\frac{2}{3}) = 2\pi+\frac{4}{3}
.\] 
Now that we know $R_1=2\pi + \frac{4}{3}$ we can find $R_2$ by subtracting the area of the disk from $R_1$ 
\[
  R_2=8p\pi-\left(2\pi+\frac{4}{3}\right)=6\pi-\frac{4}{3}
.\] 
\end{document}
