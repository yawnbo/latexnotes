\section*{Bonus question, generalization of type II integral.}%
Starting with the general integral
\[
\int_{0}^{a>0} \frac{dx}{x^{p}}
.\] 
We can find that our denominator cannot be 0 so, our lower bound needs to be evaluated as a limit and we need to rewrite the integral.
\[
\lim_{T \to 0^{+}} \int_{T}^{a} x^{-p}dx \implies \lim_{T \to 0^{+}} \left( \frac{x^{-p+1}}{-p+1} \right) \Big|_{T}^{a}
.\] 
Now we plug in values and evaluate our new expression
\[
\lim_{R \to 0^{+}} \left( \frac{a^{-p+1}}{1-p}- \frac{R^{-p+1}}{1-p} \right) 
.\] 
The limit cancels one of our terms and we are left with
\[
\frac{a^{-p+1}}{1-p}
.\] 
This is the generalization of the type II integral for when we have a valid value of p. But what are those values?
We can now notice that $p\neq 1$ which creates two possibilities, $p>1$ or $p<1$.
Testing for $p<1$ we can use $1-p>0$ so,
\[
\lim_{R \to 0^{+}} \left( - \frac{R^{-p+1}}{1-p} \right) 
.\] 
Evaluates to 0 which gives us the finite answer of $\frac{a^{-p+1}}{1-p}$. But in the case of $p>1$, we instead find that 
\[
\lim_{R \to 0^{+}} \left( - \frac{R^{-p+1}}{1-p} \right) = \infty
.\] 
Which shows us that we can use our generalization for when $p<1$ but not $p>1$ where the integral will be divergent.
