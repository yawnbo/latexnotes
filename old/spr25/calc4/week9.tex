\section{Question 1}%
\label{sec: Question 1 }
\paragraph{Find an equation of the tangent plane to the given surface at the specified point.}
\[
	\left( 1 \right) z=\left( x+2 \right) ^2-2\left( y-1 \right) ^2-5, \left( 2,3,3 \right) 
.\] 
\[
f_x = 2\left( x+2 \right) \text{ and }f_y = -4\left( y-1 \right)
.\] 
\[
f_x\left( 2,3 \right) = 8 \text{ and }f_y\left( 2,3 \right) = -8\text{ and }f\left( 2,3 \right) = 3
.\] 
So because the equation of the tangent plane is,
\[
f_x\left( a,b \right) \left( x-a \right) +f_y\left( a,b \right) \left( y-b \right) +f\left( a,b \right) 
.\] 
We simplify and are left with,
\[
z=8x-8y+11
.\] 
\[
	\left( 2 \right) x^2+y^3+z^{ 4 }=2, \left( 1,0,1 \right) 
.\] 
Same thing here but we can skip writing out everything,
\[
\nabla f\left( x,y,z \right) = \left< 2x,3y^2, 4z^3 \right>
.\] 
\[
\nabla f\left( 1,0,1 \right) = \left< 2,0,4 \right>
.\] 
And our final equation is,
\[
2x+4z=6
.\] 
\section{Question 2}%
\label{sec: Question 2 }
\paragraph{Find the linear approximation to the function at the given point. Use the linear approximation to estimate the given function value.}
\[
f\left( x,y \right) = xy+x-y;\left( 2,3 \right) ;\text{ Estimate }f\left( 2.1,2.99 \right) 
.\] 
Find the $ \nabla f $,
\[
\nabla f\left( x,y \right) = \left< y+1, x-1 \right> 
.\] 
Eval at point,
\[
f_x\left( 2,3 \right) =4\text{ and }f_y\left( 2,3 \right) =1\text{ and }f\left( 2,3 \right) = 5
.\] 
\[
\implies L\left( x,y \right) = 4\left( x-2 \right) +1\left( y-3 \right) +5
.\] 
\[
\to L\left( 2.1,2.99 \right) = 4\left( 2.1-2 \right) +\left( 2.99-3 \right) +5 = 5.39
.\] 
\section{Question 3}%
\label{sec: Question 3 }
\paragraph{Find the local maximum and minimum values and saddle points of the function.}
\[
f\left( x,y \right) = x^3-3x+3xy^2
.\] 
First find critical points,
\[
f_x= 3x^2-3+3y^2\text{ and }f_y=6xy
.\] 
\[
3x^2-3+3y^2=0 \implies x^2+y^2=1 \implies y = \pm 1\text{ and }x= \pm 1
.\] 
Take second orders,
\[
f_{ xx }= 6x\text{ and }f_{ yy }=6x\text{ and }f_{ xy }=6y
.\] 
Plug into the general formua to get,
\[
6x\cdot 6x-\left( 6y \right) ^2=36x^2-36y^2
.\] 
Now test points again the above,
\[
D\left( 0,1 \right) = -36 < 0 \implies f\left( x,y \right) \text{ has a saddle point }D\left( 0,1 \right) 
.\] 
\[
D\left( 0,-1 \right) = -36 < 0 \implies f\left( x,y \right) \text{ has a saddle point }D\left( 0,-1 \right)
.\] 
\[
D\left( 1,0 \right) = 36 > 0\text{ and }f_{ xx }\left( 1,0 \right) = 6 > 0 
\] 
\[
\implies f\left( x,y \right) \text{ has a local minimum at }D\left( 1,0 \right)
.\] 
\[
D\left( -1,0 \right) = 36 > 0\text{ and }f_{ xx }\left( -1,0 \right) = -6 < 0 
\] 
\[
\implies f\left( x,y \right) \text{ has a local maximum at }D\left( -1,0 \right)
.\] 
\section{Question 4}%
\label{sec: Question 4 }
\paragraph{Find the absolute maximum and minimum values of $ f $ on the set D}
\[
f\left( x,y \right) x^2+2y^2-2x-4y+1
.\] 
\[
D=\left\{ \left( x,y \right) \big| 0 \le x \le 2, 0 \le y \le 3 \right\} 
.\] 
Take partials and set to zero,
\[
\frac{ \partial f}{\partial x} =2x-2 \implies 2x-2 = 0 \implies x=1
.\] 
\[
\frac{ \partial f}{\partial y} =4y-4 \implies 4y-4 = 0 \implies y=1
.\] 
Which gives the point $ \left( 1,1 \right) \in D $ where we eval at that point,
\[
f\left( 1,1 \right) = -2
.\] 
Now we check the boundaries starting with the left wall or $ x=0,0\le y \le 3 $
\[
f\left( 0,y \right) = 2y^2-4y+1 
.\] 
Take the derivative and set to zero,
\[
\frac{ d }{ dy } = 4y-4 \implies 4y-4 = 0 \implies y=1
.\] 
So eval at points,
\begin{align}
	f\left( 0,0 \right) &= 1 \\
	f\left( 0,1 \right) &= -1 \\
	f\left( 0,3 \right) &= 7
.\end{align}
Do the same thing for the other walls and floors, edge2, $ x=2,0\le y \le 3 $,
\[
f\left( 2,y \right) = 2y^2-4y+1
.\] 
Same as before min at $ y=1 $, edge3, $ 0\le x \le 2,y=0 $,
\[
f\left( x,0 \right) = x^2-2x+1 \implies 2x-2 = 0 \implies x=1
.\] 
\begin{align}
	f\left( 0,0 \right) &= 1 \\
	f\left( 1,0 \right) &= 0 \\
	f\left( 2,0 \right) &= 1
.\end{align}
and finally edge4, $ 0\le x \le 2,y=3 $,
\[
f\left( x,3 \right) = x^2-2x+7 \implies 2x-2 = 0 \implies x=1
.\] 
\begin{align}
	f\left( 0,3 \right) &= 7 \\
	f\left( 1,3 \right) &= 6 \\
	f\left( 2,3 \right) &= 7
.\end{align}
Now comparing all values that we got, we find that the absolute maximum is $ 7 $ at $ \left( 0,3 \right) $ and $ \left( 2,3 \right) $ and the absolute minimum is $ -2 $ at $ \left( 1,1 \right) $.
