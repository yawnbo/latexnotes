\section{Question 1}%
\label{sec:Question 1}

\paragraph{Which of the points A(-4,0,-1), B(3,1,-5), and C(2,4,6) is closest to the yz-plane? Which point lies in the xz-plane? \\ \\} 

a) A point closest to the yz-plane occurs when x is closest to 0 which occurs at point C when $ x=2 $. \\

b) A point lies in the xz-plane when y=0 which occurs at point A when $ y=0 $.

\section{Question 2}%
\label{sec: Question 2 }

\paragraph{Find the vector that has the same direction as $ < 6,2,-3 > $ but has length 4. \\ \\}

\[
	\left| \vec{v} \right| = \sqrt{ v_1^2+v_2^2+v_3^2 } = \sqrt{ 6^2+2^2+(-3)^2 } = \sqrt{ 49 } = 7
.\] 
Normalizing the vector gives us
\[
\hat{ v } = \frac{ \vec{{v}}}{ \left\| \vec{{ v }} {} \right\| } = < \frac{ 6 }{ 7 }, \frac{ 2 }{ 7 }, \frac{ -3 }{ 7 } >
.\] 
Now multiply the normalized vector by our length of 4 to find
\[
	\vec{{ v_f }} = < \frac{ 6 }{ 7 } \cdot 4, \frac{ 2 }{ 7 } \cdot 4, \frac{ -3 }{ 7 } \cdot 4 > = < \frac{ 24 }{ 7 }, \frac{ 8 }{ 7 }, \frac{ -12 }{ 7 } >
.\] 

\section{Question 3}%
\label{sec: Question 3 }
\paragraph{$ \vec{{ u }} {  }  $ is the vector from initial point (1,2) to terminal point (3,6); $ \vec{{ v }} {  } = 2 \vec{{ i }} {  } + \vec{{ j }} {  }$;  $ \vec{{ w }} {  } = \vec{{ i }} {  } - 3 \vec{{ j }} {  }  $}
;\\
(1) Find $ 2 \vec{ u } - 3\left( \vec{ v } + \vec{ w }  \right)  $ \\
(2) Find $ \left| 2 \vec{ v } + \vec{ w } \right| $ \\
(3) Find the unit vector in the direction of $ 2 \vec{ u } + 3 \vec{ v } $ \\
\[
\vec{ u } = < 3\hat{ i }-1\hat{ i },6\hat{ j }-2\hat{ j } > = < 2,4 > 
.\] 
1) 
\[
\vec{ v } + \vec{ w } = < 2,1 > + < 1,-3 > = < 3,-2 >
.\] 
\[
= 2 \vec{ u } - 3\left( \vec{ v } + \vec{ w }  \right) = 2 < 2,4 > - 3 < 3,-2 > = < 4,8 > - < 9,-6 > = < -5,14 >
.\] 
2)
\[
\left| 2 \vec{ v } + \vec{ w } \right| = \left| 2 < 2,1 > + < 1,-3 > \right| = \left| < 4,2 > + < 1,-3 > \right| = \left| < 5,-1 > \right| \to \sqrt{ 25+1 } = \sqrt{ 26 } 
.\] 
3)
\[
 2 \vec{ u } + 3 \vec{ v } = 2 < 2,4 > + 3 < 2,1 > = < 4,8 > + < 6,3 > = < 10,11 >
.\] 
Normalizing and finding the unit vector,
\[
\sqrt{ 100 + 121 } = \sqrt{ 221 } \implies \hat{ u } = \frac{ 2 \vec{ u } + 3 \vec{ v } }{ \left| 2 \vec{ u } + 3 \vec{ v } \right| } = < \frac{ 10 }{ \sqrt{ 221 } },\frac{ 11 }{ \sqrt{ 221 } }>
.\] 

\section{Question 4}%
\label{sec: Question 4 }
\paragraph{$ \vec{ u } $ is the vector from initial point $ \left( 1,2 \right)  $ to terminal point $ \left( 3,6 \right)  $; Given $ \vec{ v } = 2 \hat{ i } + \hat{ j } $ ; and $ \vec{ w }= \hat{ i }- 3 \hat{ j } $ Find $ Proj_{ \vec{ u } }\left( \vec{ v } + \vec{ w } \right)  $ \\ \\}
Again $ \vec{ u } = <2,4> $ and $ \vec{ v } + \vec{ w } = <3,-2> $ 
\[
Proj_{ \vec{ u } }\left( \vec{ v } + \vec{ w } \right) = \frac{ \vec{ u } \cdot \left( \vec{ v } + \vec{ w } \right) }{ \left\| \vec{ u } \right\|^2 } \cdot \vec{ u } = \frac{ < 2,4 > \cdot < 3,-2 > }{ 20 } < 2,4 > = \frac{ 6-8 }{ 20 } < 2,4 > = -\frac{ 1 }{ 10 } < 2,4 > = < -\frac{ 1 }{ 5 },-\frac{ 2 }{ 5 }>
.\] 

\section{Question 5}%
\label{sec: Question 5 }
\paragraph{A constant force $ F= < 4,3,2> $ (in newtons) moves an object from $ \left( 0,0,0 \right) $ to $ \left( 8,6,0 \right) $. Distance is measured in meters. Calculate the work done. \\ \\}

\[
\vec{ D } = < 8,6,0 >
.\] 
\[
\vec{ W } = \vec{ F } \cdot \vec{ D } = < 4,3,2 > \cdot < 8,6,0 > = 32 + 18 + 0 = 50 \text{ Joules }
.\] 
\section{Question 6}%
\label{sec: Question 6 }
\paragraph{For what value of a is the vector $ \vec{ v } = <4,-3,7> $ orthogonal to $ \vec{ w } = <a,8,3> $? \\ \\}
This happens when $ \vec{ v } \cdot \vec{ w } = 0 $, so solve,
\[
\vec{ v }\vec{ w } = <4,-3,7> \cdot <a,8,3> = 4a - 24 + 21 = 0 \implies 4a - 3 = 0 \implies a = \frac{ 3 }{ 4 }
.\] 
\section{Question 7}%
\label{sec: Question 7 }
\paragraph{Find the area of  a parallelogram that has two adjacent sides $ u=2i- j-2k $ and $ v=3i+2j-k $.}
\[
	A = \left\| \vec{ u } \cdot \vec{ v } \right\| = \begin{bmatrix} \vec{ i } & \vec{ j } & \vec{ k }\\ 2 & -1 & -2 \\ 3 & 2 & -1 \end{bmatrix} = \vec{ i } \begin{vmatrix} -1 & -2 \\ 2 & -1 \end{vmatrix} - \vec{ j } \begin{vmatrix} 2 & -2 \\ 3 & -1 \end{vmatrix} + \vec{ k } \begin{vmatrix} 2 & -1 \\ 3 & 2 \end{vmatrix} = \vec{ i } (1+4) - \vec{ j } (-2+6) + \vec{ k } (4+3) = \left\| < -5,4,7 > \right\|
.\] 
\[
A = \sqrt{ 25 + 16 + 49 } = \sqrt{ 90 } = 3\sqrt{ 10 }
.\] 

\section{Question 8}%
\label{sec: Question 8 }
\paragraph{Find a vector orthogonal to both $ \vec{ u } = <2,4,7> $ and $ \vec{ v } = <1,5,2> $. \\ \\}
\[
\vec{ u } \cdot  \vec{ v } = \begin{bmatrix} \vec{ i } & \vec{ j } & \vec{ k }\\ 2 & 4 & 7 \\ 1 & 5 & 2 \end{bmatrix} = \vec{ i } \begin{vmatrix} 4 & 7 \\ 5 & 2 \end{vmatrix} - \vec{ j } \begin{vmatrix} 2 & 7 \\ 1 & 2 \end{vmatrix} + \vec{ k } \begin{vmatrix} 2 & 4 \\ 1 & 5 \end{vmatrix} = \vec{ i } (8-35) - \vec{ j } (4-7) + \vec{ k } (10-4) = < -27,3,6 >
.\] 

