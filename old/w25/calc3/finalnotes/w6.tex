\section{11.9 Last examples}%
\label{sec:11.9 Last example}
\paragraph{Estimating a non-elementary integral}
\[
\int_{ 0 }^{ 0.3 } \frac{ x^2 }{ 1+x^{ 4 } }
.\] 
First make a power series for our function,
\[
x^2\left( \frac{ 1 }{ 1-\left( -x^{ 4 } \right)  }  \right) 
.\] 
\[
=\sum_{ n=0 } ^{ \infty } \left( -1 \right) ^{ n }x^{ 4n+2 }
.\] 
Now verify that $ \left[ 0,0.3 \right]  $ is in the IOC. So,
\[
\int_{ 0 }^{ 0.3 } \sum_{ n=0 } ^{ \infty } \left( -1 \right) ^{ n }x^{ 4n+2 }dx \to \lim_{ n \to \infty} \left| \frac{ x^{ 4n+6 } }{ x^{ 4n+2 } } \right|= \left| x^{ 4 } \right| < 1 \implies \left( -1,1 \right) 
.\] 
Evaluating our integral,
\[
\sum_{ n=0 } ^{ \infty } \left( -1 \right) ^{ n }\int_{ 0 }^{ 0.3 } x^{ 4n+2 }dx = \sum_{ n=0 } ^{ \infty } \left( -1 \right) ^{ n } \frac{ x^{ 4n+3 } }{ 4n + 3 } \bigg| _{ 0 }^{ 0.3 }
.\] 
\[
	=\sum_{ n=0 } ^{ \infty } \frac{ \left( -1 \right) ^{ n } \left( 0.3 \right) ^{ 4n+3 } }{ 4n+3 } - \cancel{\sum_{ n=0 } ^{ \infty } \frac{ \left( -1 \right) ^{ n }\left( 0 \right) ^{ 4n+3 } }{ 4n+3 }}
.\] 
Now we have an alternating series so we can easily estimate it with ASET to 6 decimal places. Knowing that our error will be $ \left| R_n \right|= \left| S-S_n \right| \le b_{ n+1 } $. So,
\[
\text{Error }\le b_{ n+1 } = \frac{ \left( 0.3 \right) ^{ 4\left( n+1 \right) +3 } }{ 4\left( n+1 \right) +3 } \le 5\cdot 10^{ -7 }
.\] 
There isn't a nice way of doing this so we can just manually compute it for this. Using a calculator we find that our error is $ 1.6\cdot 10^{ -7 } $ at $ n=2 $ but because of ASET having the $ b_{ n+1 } $ we want to add one to the final result of n. So our final value is going to be $ n=3 $. Without a calc this would look like,
\[
\sum_{ n=0 } ^{ 2 } \frac{ \left( -1 \right) ^{ n }\left( 0.3 \right) ^{ 4n+3 } }{ 4n+3 } = \frac{ \left( 0.3 \right) ^{ 3 } }{ 3 } - \frac{ \left( 0.3 \right) ^{ 7 } }{ 7 } + \frac{ \left( 0.3 \right) ^{ 11 } }{ 11 } \approx 0.0089689182
.\] 
\newpage
Now finding the actually value of our integral we find it to be $ \int_{ - }^{ 0.3 } \frac{ x^2 }{ 1+x^{ 4 } } = 0.00896892$. Which gives us an accuracy of $ 1.6\cdot 10^{ -7 } $. Side note, we chose $ 5\cdot 10^{ -7 } $ because of rounding reasons that gives us a middle ground for our accuracy. \\
We can also do it with a calculator and without brute force,
\[
y_1 = \frac{ \left( 0.3 \right) ^{ 4\left( n+1 \right) +3 } }{ 4\left( n+1 \right) +3 }< y_2 = 5\cdot 10^{ -7 }
.\] 
After this you can build your graph to have a $ y_{ \text{ max } }= 5\cdot 10^{ -6 } $ and a $ y_{ \text{ min } }= 5\cdot 10^{ -8 } $. Now looking at the picture. After being in intersect mode we find our $ n\approx 0.8 $ and we can test this,
\[
\sum_{ n=0 } ^{ 1 } \left( -1 \right) ^{ n } \frac{ \left( 0.3 \right) ^{ 4n+3 } }{ 4n+3 } = \frac{ 0.3^{ 3 } }{ 3 }- \frac{ 0.3^{ 7 } }{ 7 } \approx 0.0089687571
.\] 
Which gives us our desired accuracy of $ 5\cdot 10^{ -7 } $. This shows us that ASET is generally more generous and will overestimate unlike solving algebraically which will give us a more accurate result.
\section{11.10 Taylor \& Maclaurin Series}%
Let $ f\left( x \right)  $ be a function with power series $ \sum_{ k=0 } ^{ \infty } a_k\left( x-c \right) ^{ k } $ for $ \left| x-c \right|<R $. This is called a general power series. Our goal is to find what $ a_k $ would be. Just naming terms, $ \left| x-c \right| $ will be the center of our IOC, and $ R $ will be the radius of convergence. \\ \\

Looking at few terms of $ f\left( x \right)  $,
\[
f\left( x \right) =a_0 + a_1\left( x-c \right) + a_2\left( x-c \right) ^{ 2 } + a_3\left( x-c \right) ^{ 3 } + \ldots
.\] 
Taking a derivative,
\[
f'\left( x \right) = f^{ \left( 1 \right)  }\left( x \right) = a_1 + 2a_2\left( x-c \right) + 3a_3\left( x-c \right) ^{ 2 } + \ldots
.\] 
And another,
\[
f^{ \left( 2 \right)  }\left( x \right) = 2a_2 + 3\cdot 2a_3\left( x-c \right) + 4\cdot 3a_4\left( x-c \right) ^{ 2 } + \ldots
.\] 
With these, we want to find a pattern that would give us the $ f^{ \left( k \right) \left( x \right)  } $ derivative. Which we find to be,
\[
f^{ \left( k \right)  }\left( x \right) = k!a_k + \left( k+1 \right)! a_{ k+1 }\left( x-c \right) + \left( k+2 \right) ! a_{ k+2 }\left( x-c \right) ^2 + \ldots
.\] 
Now we can evaluate this at $ x=c $,
\begin{gather*}
f\left( c \right) =a_0 + 0 + 0 + 0\ldots \\
f'\left( c \right) = a_1 + 0 + 0 + 0 \ldots \\
f^{ 2 }\left( x \right) = 2a_2 \\
f^{ 3 }\left( x \right) = 6a_3 \\
\end{gather*}
So when going to the $ \frac{ d }{ dx } k^{ \text{th } } $ we can generalize it as $ f^{ k }\left( c \right) = k!a_k $. Now solving for $ a_k $, we find that $ a_k=\frac{ f^{ k }\left( c \right)  }{ k! } $. This is where the Taylor series comes from and we can now write our series as.
\[
\sum_{ k=0 } ^{ \infty } \frac{ f^{ k }\left( c \right)  }{ k! }\left( x-c \right) ^{ k }
.\] 
\subsection*{Taylor's theorem}%
\label{sub:Taylor's theorem}
\paragraph{If} $ f\left( x \right)  $ has a power series, then 
\[
f\left( x \right)=\sum_{ k=0 } ^{ \infty } \frac{ f^{ k }\left( c \right)  }{ k! }\left( x-c \right) ^{ k } \text{ for } \left| x-c \right|<R
.\] 
The maclaurin series is just a Taylor series with $ c=0 $. \\ \\

\subsection*{Usage}%
\label{sub:Usage}
Recall how we would make power series,
\[
\frac{ 1 }{ 1-x^2 } \to \sum_{ n=0 } ^{ \infty } \left( x^2 \right) ^{ n }
.\] 
This works for normal functions, but we can do this for functions like $ \sin^{  } \left( x \right)  $. So let's instead use a taylor series.\\ \\
\paragraph{Ex.}
	First let's choose a center and make a maclaurin series. So let's put the 0 in the center and make a series for $ \sin\left( x \right)  $.
	\[
	\sin^{  } \left( x \right) = \sum_{ k=0 } ^{ \infty } \frac{ f^{ k }\left( 0 \right)  }{ k! }\left( x-c \right) ^{ k }
	.\] 
	So let's start by writing some terms as we go through values of k. 
	\begin{gather*}
	f^{ 0 }\left( 0 \right) =\sin^{  } \left( 0 \right) =0 \\
	f^{ 1 }\left( 0 \right) =\cos^{  } \left( 0 \right) =1 \\
	f^{ 2 }\left( 0 \right) =-\sin^{  } \left( 0 \right) =0 \\
	f^{ 3 }\left( 0 \right) =-\cos^{  } \left( 0 \right) =-1 \\
	f^{ 4 }\left( 0 \right) =\sin^{  } \left( 0 \right) =0 \\
	\end{gather*}
	Now what happens if we throw in our factorial?
	\[
	=\frac{ 0 }{ 0 } , \frac{ 1 }{ 1! }, \frac{ 0 }{ 2! }, \frac{ -1 }{ 3! }, \frac{ 0 }{ 4! }
	.\] 
	Ignoring the zeroes we can start to make an explicit formula for our series as,
	\[
	\frac{ \left( -1 \right) ^{ k } }{ \left( 2k+1 \right) ! }
	.\] 
	Now with our function we can put it back into our power series as
	\[
	\sin^{  } \left( x \right) =\sum_{ k=0 } ^{ \infty } \frac{ \left( -1 \right) ^{ k } }{ \left( 2k+1 \right) ! }x^{ k }
	.\] 
	Now we want to find our IOC, so let's use ratio test and take the $ \lim_{ n \to \infty} \left| \frac{ a_{ k+1 } }{ a_k } \right| $ or,
	\[
	=\lim_{ n \to \infty} \left| \frac{ x^{ k+1 } }{ \left( 2\left( k+1 \right) +1 \right) ! }\cdot \frac{ \left( 2k+1 \right) ! }{ x^{ k } } \right| = \lim_{ k \to \infty} \frac{ 1 }{ \left( 2k+3 \right) \left( 2k+2 \right)  } \left| x \right|<1
	.\] 
	Now because $ \lim_{ k \to \infty} \frac{ 1 }{ \left( 2k+3 \right) \left( 2k+2 \right)  }  $ goes to 0 we know our interval of convergence will be $ \left( -\infty,\infty \right)  $.  \\ \\ 
	Just doing some proof work we estimate sin to be $ 1-\frac{ x }{ 3! } +\frac{ x^2 }{ 5! }  $ but this doesnt look right. Looking back at our series we only took our odd powers, but never accounted for it in our power series. So instead our series will be,
	\[
	\sum_{ k=0 } ^{ \infty } \frac{ \left( -1 \right) ^{ k } }{ \left( 2k+1 \right) ! }x^{ 2k+1 }
	.\] 

\subsection*{Maclaurin series of $ \cos^{  } \left( x \right)  $}%
	 $ f\left( x \right) =\cos^{  } \left( x \right)  $ and our sum $ \sum_{ k=0 } ^{ \infty } \frac{ f^{ k }\left( 0 \right)  }{ k! }\left( x-c \right) ^{ k } $ need to be solved. So writing out terms,
	 \begin{align*}
			f^{ 0 }\left( 0 \right) &= \cos^{  } \left( 0 \right) =1 \\
			f^{ 1 }\left( 1 \right) &= -\sin^{  } \left( 0 \right) =0 \\
			f^{ 2 }\left( 2 \right) &= -\cos^{  } \left( 0 \right) =-1 \\
			f^{ 3 }\left( 3 \right) &= \sin^{  } \left( 0 \right) =0 \\
			f^{ 4 }\left( 4 \right) &= \cos^{  } \left( 0 \right) =1 \\
	 .\end{align*}
	 Writing our sequence,
	 \[
		a_k=\frac{ 1 }{ 0! } ,\frac{ 0 }{ 1! } ,\frac{ -1 }{ 2! } ,\frac{ 0 }{ 3! } ,\frac{ 1 }{ 4! }\ldots
	 .\] 
	 We can find our sequence to be 
	 \[
	 a_k=\frac{ \left( -1 \right) ^{ k } }{ \left( 2k \right) ! }
	 .\] 
	 and our series to be 
	 \[
	 \sum_{ k=0 } ^{ \infty } \frac{ \left( -1 \right) ^{ k } }{ \left( 2k \right) ! }\left( x \right) ^{ 2k }
	 .\] 
		Note that we use $ 2k $ because we only want the evens in this series. If we instead kept it as just $ k $, we would have the odd powers and those would just zero. Now let's write a partial sum to test it,
		\[
		=\frac{ 1 }{ 0! } x^{ 0 }-\frac{ x^2 }{ 2! } + \frac{ x^{ 4 } }{ 4! }-\frac{ x^{ 6 } }{ 6! } + \frac{ x^{ 8 } }{ 8! } - \ldots
		.\] 

\section*{02/12/25}%
\label{sec:02/12/25}
\paragraph{Practice review is out and the review has an extra point problem about fractals worth 4 points. }

	\paragraph{Ex.}
	Find the maclaurin series for $ f\left( x \right) = \tan^{ -1 } \left( x \right)  $. Start by getting our derivatives,
	\begin{align*}
		f\left( x \right) &= \tan^{ -1 } \left( x \right) \\
		f^{ 1 }\left( x \right) &= \frac{ 1 }{ 1+x^2 } \\
		f^{ 2 }\left( x \right) &= \frac{ -1\left( 1+x^2 \right) ^{ -2 }\cdot 2x }{ \left( 1+x^2 \right) ^2 } \\
		f^{ 3 }\left( x \right) &= \frac{ -2\left( 1+x^2 \right)^{ -3 } -2x\left( 1+x^2 \right)^{ -2 }  }{ \left( 1+x^2 \right) ^2 } = 2\left( 1+x^2 \right) ^{ -3 }\left( 2x-\left( 1+x^2 \right) \right) \\
		f^{ 4 }\left( x \right) &=  \frac{ 24x\left( 1+x^2 \right)  }{ \left( 1+x^2 \right) ^{ 4 } }  \\
		f^{ 5 }\left( x \right) &= \frac{ 24\left( 1-10x^2+5x^{ 4 } \right)  }{ \left( 1+x^2 \right) ^{ 5 } } \\
	\end{align*}
	Because we want a mclaurin, we want to evaluate each function at 0,
	\begin{align*}
		f\left( 0 \right) &= \tan^{ -1 } \left( 0 \right) = 0 \\
		f^{ 1 }\left( 0 \right) &= \frac{ 1 }{ 1+0 } = 1 \\
		f^{ 2 }\left( 0 \right) &= \frac{ 0 }{ 1 } = 0 \\
		f^{ 3 }\left( 0 \right) &= = -2 \\
		f^{ 4 }\left( 0 \right) &= 0 \\
		f^{ 5 }\left( 0 \right) &= 24 \\
	\end{align*}
	We see that this gives us the pattern $ \underbrace{ 0 }_{ k=0 } ,1,0,-2,0,24,0,-720 $. Looking at these we see that we have a sequence of even factorials. This can now be built into a sum,
	\[
	\sum_{ k=0 } ^{ \infty } \frac{ \left( -1 \right) ^{ k }\left( 2k \right) ! }{ k! }
	.\] 
	Now because we only want the odds we can instead write it as,
	\[
	\sum_{ k=0 } ^{ \infty } \frac{ \left( -1 \right) ^{ k }\left( 2k \right) ! }{ \left(2k+1\right)! }x^{ 2k+1 }
	.\] 
	Now cancelling,
	\[
	\sum_{ k=0 } ^{ \infty } \frac{ \left( -1 \right) ^{ k } }{ 2k+1 }x^{ 2k+1 }
	.\] 
	\paragraph{The easier way with integration}
	Start by taking the $ \frac{ d }{ dx }  $,
	\[
	\frac{ d }{ dx } \tan^{ -1 } \left( x \right) = \frac{ 1 }{ 1+x^2 } \to \frac{ 1 }{ 1-\left( x^2 \right)  } = \sum_{ k=0 } ^{ \infty } \left( -x^2 \right) ^{ n }
	.\] 
	But this is just the sum of the derivative, so we can integrate both sides to find
	\[
	\int_{  }^{  } \frac{ d }{ dx } \tan^{ -1 } \left( x \right) =\int_{  }^{  } \sum_{ k=0 } ^{ \infty } \left( -1 \right) ^{ n }x^{ 2n } \to \sum_{ k=0 } ^{ \infty } \left( -1 \right) ^{ n }\frac{ x^{ 2n+1 } }{ 2n+1 }+C
	.\] 
	Now to find C we just plug in 0 for x,
	\[
	\underbrace{ \sum_{ k=0 } ^{ \infty } \frac{ \left( -1 \right) ^{ n }0^{ 2n+1 } }{ 2n+1 } }_{ 0 } \implies C = 0
	.\] 
	This again goes back to the same series we found earlier and shows that power series are unique up to their IOC, or there is only ONE power series for each IOC. 
	\paragraph{Ex.Making a new series from old series.}
	Knowing that \[
	e^{ x }=\sum_{ n=0 } ^{ \infty } \frac{ x^{ n } }{ n! }
	.\] 
	How can we make a power series for $ e^{ 5x+1 } $?
	We can just do this with subsitution where our new power series will look like 
	\[
=	\sum_{ n=0 } ^{ \infty } \frac{ \left( 5x+1 \right) ^{ n } }{ n! }
	.\] 
	Same thing for $ e^{ x^{ 3 } } $,
	\[
	=\sum_{ n=0 } ^{ \infty } \frac{ \left( x^3 \right) ^{ n } }{ n! }
	.\] 
	For something like $ \ln^{  } \left( x^2 \right)  $, we would have to set the inside of our known  maclaurin, $ \ln^{  } \left( 1+x \right) =\sum_{ k=1 } ^{ \infty } \left( -1 \right) ^{ k-1 }\frac{ x^{ k } }{ k } $ to $ x^2 $ so,
	\[
	\ln^{  } \left( x^2 \right) = \ln^{  } \left( 1+\left( x^2-1 \right)  \right) = \sum_{ k=1 } ^{ \infty } \left( -1 \right) ^{ k-1 }\frac{ \left( x^2-1 \right) ^{ k } }{ k }
	.\] 
	Also works for $ \frac{ 1 }{ x }  $ like
	\[
	\frac{ 1 }{ 1-\left( 1-x \right)  } = \sum_{ k=0 } ^{ \infty } \left( 1-x \right) ^{ n }
	.\] 
\paragraph{Find the taylor series for $ f\left( x \right) =x^{ -3 } $ centered at $ c=1 $}
Taking $ \frac{ d }{ dx }  $ 's we get,
\begin{align*}
	f\left( x \right) &= x^{ -3 } \to 1\\
	f^{ 1 }\left( x \right) &= -3x^{ -4 } \to -3 \\
	f^{ 2 }\left( x \right) &= 3\cdot 4 x^{ -5 } \to 3\cdot 4\\
	f^{ 3 }\left( x \right) &= -3\cdot 4\cdot 5 x^{ -6 } \to -3\cdot 4\cdot 5\\
	\ldots \\
\end{align*}
Giving us the sequence,
\[
f^{ n }\left( 1 \right) =\left( -1 \right) ^{ n }\frac{ \left( n+2 \right) ! }{ 2 }
.\] 
Taylor's theorem tells us that this will be
\[
x^{ -3 }= \sum_{k=0}^{\infty} \frac{ \left( -1 \right) ^{ k }\left( k+2 \right) ! }{ 2\cdot k! }\left( x-1 \right) ^{ k } = \sum_{ k=0 } ^{ \infty } \left( -1 \right) ^{ k } \frac{ \left( k+2 \right) \left( k+1 \right) \left( x-1 \right) ^{ k } }{ 2 }
.\] 
With this IOC can be done using ratio test, and should become $ \left| x -1 \right| < 1 $ or $ R=1 $. 
Friday's class will probably be the last lecture before the test and will be on sec11.11 or the $ T_n\left( x \right) : $ n-th degree Taylor polynomial. 
\paragraph{Quick Ex.}
\[
e^{ x }=\sum_{ k=0 } ^{ \infty } \frac{ x^{ k } }{ k! }= 1 + \frac{ x }{ 1 } +\frac{ x^2 }{ 2 } + \frac{ x^3 }{ 3! }+ \frac{ x^{ 4 } }{ 4! }\ldots
.\] 
If we take a partial sum $ S_0 $ then this would be called the $ T_0\left( x \right) 1 $ or the $ 0^{ \text{ th } } $-deg Taylor Polynomial. Basically just the estimation of the radical function in terms of a taylor polynomial. 
\section*{02/14/25}%
\label{sec:02/14/25}

\subsection*{Last lecture for chapter 11: Polynomials}%
\label{sub:Last lecture for chapter 11}
\subsection*{11.11 Taylor Polynomials}%
\label{sub:11.11 Taylor Polynomials}

\[
\sum_{k=0}^{\infty} \frac{ f^{ k }\left( c \right)  }{ k! }(x-c)^k , \left| x-c \right| < R
.\] 
Getting some terms would look like,
\[
= f\left( c \right) + f'\left( c \right) (x-c) + \frac{ f^{ 2 }\left( c \right)  }{ 2! }(x-c)^2 + \ldots
.\] 
Where taking each leading term and the sum of it's previous as a subsequence we can write it as $ T_n\left( x \right)  $. These can be called partial sums, but are more accurately defined as Taylor polynomials.

\paragraph{Maclaurin series example}
\[
\sum_{ n=0 } ^{ \infty } \frac{ \left( -1 \right) ^{ n } x^{ 2n+1 }}{ \left( 2n+1 \right) ! }= x - \frac{ x^3 }{ 3! } + \frac{ x^5 }{ 5! } - \ldots
.\] 
For this we DON'T have a 0th Taylor polynomial as it's centered at 0 and would just be 0. This means that our first significant one would be at $ T_1\left( x \right)  $. For things like this we cannot write $ T_2\left( x \right)  $ because our leading term would be non-existent. This would be the same as calling a quadratic cubic even though there is no leading term that contains a cube. 

\paragraph{Ex.}
\paragraph{Find the Taylor polynomial $ T_3\left( x \right)  $ for the function f centered at the number a. }
\[
f\left( x \right) =\cos^{  } \left( x \right) , a=\frac{ \pi }{ 2 } 
.\] 
With this we know our c value can be $ \frac{ \pi }{ 2 }  $ and we can write out some terms of the derivative to find our taylor,
\begin{align*}
	f^{ 0 }\left( x \right) &=\cos^{  } \left( x \right) = 0\\
	f^{ 1 }\left( x \right) &= -\sin^{  } \left( x \right) =-1 \\
	f^{ 2 }\left( x \right) &= -\cos^{  } \left( x \right) =0 \\
	f^{ 3 }\left( x \right) &= \sin^{  } \left( x \right) =1 
.\end{align*}
Where we now write it as a polynomial,
\[
T_3\left( x \right) = \frac{ -1 }{ 1! }\left( x-\frac{ \pi }{ 2 }  \right) ^{ 1 }+ \frac{ 1 }{ 3! } \left( x-\frac{ \pi }{ 2 }  \right) ^3
.\] 
\section*{Taylor's Inequality}%
\label{sec:Taylor's Inequality}
\paragraph{If $ \left| f^{ n+1 }\left( x \right)  \right|\le M $ for $ \left| x-c \right|\le d $ then $ R_n $ satisfies }
\[
\left| \text{ Error } \right|= \left| R_n\left( n \right)  \right|\le \frac{ M }{ \left( n+1 \right) ! } \left| x-c \right|^{ n+1 }\text{ for } \left| x-c \right|\le d
.\] 
\paragraph{Ex.}
Suppose $ x\epsilon\left( \frac{ \pi }{ 3 } ,\frac{ 2pi }{ 3 }  \right)  $ for our previous example taylor polynomial. \\
To estimate this we need to take the $ f^{ n+1 }\left( x \right)  $ which is the same as $ f^{ n }\left( x \right)  $ for cos and sin so we can say that $ f^{ 3+1 }\left( x \right) \approx \cos^{  } \left( x \right)  $. Now with the Taylor inequality we need to find the largest value of $ f^{ n+1 }\left( x \right)  $ over $ \left[ a,b \right]  $. So our max will be $ f^{ 4 }\left( x \right) =\frac{ 1 }{ 2 } =M $ and we can write our error to be 
\[
\left| \text{ Error } \right|\le \frac{ \frac{ 1 }{ 2 }  }{ \left( 3+1 \right) ! }
.\] 
We also need the last part of our inequality so we want to have 
\[
\frac{ \pi }{ 3 } <x<\frac{ 2pi }{ 3 } \to \left| x-c \right|\le R
.\] 
For this we already know we have a center at $ \frac{ \pi }{ 2 }  $ but if we didnt know it we would take the average,
\[
\frac{ \frac{ \pi }{ 3 } +\frac{ 2pi }{ 3 }  }{ 2 }= \frac{ \frac{ 3pi }{ 3 }  }{ 2 }=\frac{ \pi }{ 2 } =c
.\] 
Which also gives us what we want. Now that we would know the center we can find the $ R $ (radius) to be $ \frac{ \pi }{ 6 }  $. Where we can now write the second part as $ \left| x-\frac{ \pi }{ 2 }  \right|<\frac{ \pi }{ 6 }  $ and the whole thing as
\[
\left| \text{ Error } \right|\le \frac{ \frac{ 1 }{ 2 }  }{ \left( 3+1 \right) ! } \left|  x-\frac{ \pi }{ 2 } \right|^{ 3+1 }\le \frac{ 1 }{ 2\left( 4 \right) ! } \left( \frac{ \pi }{ 6 }  \right) ^{ 4 }
.\] 
This can be put into a calculator and we can find our error to be $ \approx 0.0015658612 $ which would give us an accuracy of around $ 10^{ -3 } $. 
\paragraph{What if  we instead wannted to find what values will $ T_3\left( x \right)  $ approx cosx with $ \left| \text{ Error } \right|\le 10^{ -4 } $?}
We start with the same starting inequality, $ \frac{ \frac{ 1 }{ 2 }  }{ 4! } \left| x-\frac{ \pi }{ 2 }  \right|^{ 4 }\le 10^{ -4 } $ but switching variables so we need to solve it with algebra. Rewriting and simplifying,
\[
\frac{ 1 }{ 48 } \left| x-\frac{ \pi }{ 2 }  \right|^{ 4 } \le 10^{ -4 } \to \sqrt[ 4 ]{ \left| x-\frac{ \pi }{ 2 }  \right|^{ 4 } } \le \sqrt[ 4 ]{ 48\cdot 10^{ -4 } } \to \frac{ \pi }{ 2 } - \sqrt[ 4 ]{ 48 } \cdot 10^{ -1 }< x < \sqrt[ 4 ]{ 48 } \cdot 10^{ -1 }+\frac{ \pi }{ 2 } 
.\] 
With a calculator,
\[
1.3075 < x < 1.8340
.\] 
Where we can expect our error to be in the range of $ 10^{ -4 } $ for this interval of x. It's very important to note that these are ESTIMATES and we can even go to something like $ 1.9 $ for the upper bound and still be within our range. \\ \\
Note the exam will be building power series, proving convergence/divergence and a few taylor series questions. Some maclaurin series will be given (common ones) and will be used a few times. 
