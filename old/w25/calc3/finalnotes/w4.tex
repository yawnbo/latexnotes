\section*{11.5 Ratio/root test \& absolute convergence}%
\label{sec:11.5 Ratio/root test}
\paragraph{Def. Absolute convergence}
A series $ \sum_{  } ^{  } a_n $ is absolutely convergent if $ \sum_{  } ^{  } \left\| a_n \right\| $ converges.
\paragraph{Ex.}
\[
\sum_{ n=1 } ^{ \infty } \frac{ -1^{ n } }{ n^{ 4 } } \to \sum_{  } ^{  } \left\| \frac{ -1^{ n } }{ n^{ 4 } } \right\| = \sum_{ n=1 } ^{ \infty } \frac{ 1 }{ n^{ 4 } } 
.\] 
The above will converge because of p-test so this series will be absolutely convergent. All the definition is that the series will converge if the absolute value of the series converges. \\

\paragraph{Def. Conditional convergence}
$ \sum_{  } ^{  } a_n $ is conditionally convergent if $ \sum_{  } ^{  } \left\| a_n \right\| $ converges but $ \sum_{ } ^{  } a_n $ diverges. 

\paragraph{Ex.}
\[
	\sum_{ n=1 } ^{ \infty } \frac{ \left( -1 \right) ^{ n+1 } }{ n } \text{ is conditionally convergent.  }
.\] 
This is because if we put an absolute to make it $ \sum_{ n=1 } ^{ \infty } \left\| \frac{ \left( -1 \right) ^{ n+1 } }{ n } \right\| $ will equal $ \sum_{ n=1 } ^{ \infty } \frac{ 1 }{ n }  $ which is the harmonic series.

\subsection*{11.5.1 Theorem}%
\label{sub:11.5.1 Theorem}
\paragraph{Def.}
If $ \sum_{  } ^{  } \left\| a_n \right\| $ converges, then $ \sum_{  } ^{  } a_n $ converges. With the contrapositive being, if $ \sum_{  } ^{  } a_n $ diverges, then $ \sum_{  } ^{  } \left\| a_n \right\| $ also diverges. 
\paragraph{Ex. Show that $ \sum_{  } ^{  } \frac{ \sin^{  } \left( n^2 \right)  }{ n^2 } $ converges or diverges.}
Let's use the previously established theorem, consider 
\[
\sum_{ n=1 } ^{ \infty } \left\| \frac{ \sin^{  } \left( n^2 \right)  }{ n^2 } \right\|
.\] As our new series and use squeeze theorem to show that it converges.

\[
\sum_{ n=1 } ^{ \infty } \left\| \frac{ \sin^{  } \left( n^2 \right)  }{ n^2 } \right\| \le \sum_{ n=1 } ^{ \infty } \frac{ 1 }{ n^2 } 
.\] 
Which can now be found using direct comparison because our new series is larger than our sin. Know that we have a convergent series that is larger, we know that the smaller series must also converge by absolute test and comparison. 
\subsection*{11.5.2 Examples}%
\label{sub:11.5.2 Examples}

\paragraph{For a) $ \sum_{ k=1 } ^{ \infty } \frac{ \left( -1 \right) ^{ k+1 } }{ \sqrt{ k} } \text{ b)  }\sum_{ n=1 } ^{ \infty } \frac{ \left( -1 \right) ^{ n+1 } }{ \sqrt{ n^3} } \text{ c)  } \sum_{ n=1 } ^{ \infty } \frac{ \left( -1 \right) ^{ k }k }{ k+1 } $ }
\paragraph{Which series converge absolutely, converge conditionally, or diverge?}
Each of these become
\begin{align*}
a = \sum_{  } ^{  } \frac{ 1 }{ \sqrt{ k} } \text{ which diverges by p-test } \\
b = \sum_{  } ^{  } \frac{ 1 }{ \sqrt{ n^3} } \text{ which converges by p-test } \\
c = \sum_{  } ^{  } \frac{ k }{ k+1 } \text{ which diverges } \\
.\end{align*}
\paragraph{a will be conditionally convergent by AST because $ \frac{ 1 }{ \sqrt{ k} } $ is decreasing and $ \lim_{ k \to \infty } \frac{ 1 }{ \sqrt{ k} } = 0 $. b will be absolutely convergent because $ \frac{ 1 }{ \sqrt{ n^3} } $ is decreasing and $ \lim_{ n \to \infty } \frac{ 1 }{ \sqrt{ n^3} } = 0 $. c will diverge because $ \frac{ k }{ k+1 } $ is increasing and $ \lim_{ k \to \infty } \frac{ k }{ k+1 } = 1 $.}

\subsection*{11.5.3 The ratio test}%
\label{sub:11.5.3 The ratio test}
(i) If $ \lim_{ n \to \infty} \left\| \frac{ a_{ n+1 } }{ a_n } \right\|<1 $ but not 0, then $ \sum_{  } ^{  } \left\| a_n \right\| $ converges. \\ \\
(ii) If $ \lim_{ n \to \infty} \left\| \frac{ a_{ n+1 } }{ a_n } \right\| >1 $ then $ \sum_{  } ^{  } a_n $ is divergent. \\
(iii) If $ \lim_{ n \to \infty} \left\| \frac{ a_{ n+1 } }{ a_n } \right\|=1 $ then the test is inconclusive
\paragraph{Ex.}
\[
\text{ Test }\sum_{ n=1 } ^{ \infty } \frac{ \left( -1 \right) ^{ n-1 }n }{ e^{ n } } \text{ for convergence or divergence }
.\]
Start by plugging in $ a_{ n+1 } $ as our value so,
\[
a_{ n+1 }= \frac{ \left( -1 \right) ^{ \left( n+1 \right) -1 } n+1}{ e^{ n+1 } } 
.\] 
And write it over the absolute to get rid of the $ -1^{ n } $,
\[
\left\| \frac{ a_{ n+1 } }{ a_n } \right\| = \frac{ \frac{ n+1 }{ e^{ n+1 } } }{ \frac{ n }{ e^{ n } }  } = \lim_{ n \to \infty} \frac{ n+1 }{ n }\cdot \frac{ e^{ n } }{ e^{ n } } = \frac{ e^{ n } }{ e^{ n }\cdot e } = \frac{ 1 }{ e } = 1\cdot \frac{ 1 }{ e } = \frac{ 1 }{ e } <1
.\] 
Which solves our (i) case and we know that the series converges. This can also be done with the integral test using
\[
\int_{ 1 }^{ \infty } \frac{ x }{ e^{ x } } dx
.\] 
and showing that it is finite. End of monday class.

\section*{Tests cont.}%
\label{sec:Tests}
\paragraph{Note that quiz 3 is open by tomorrow and due in a week I assume.}

\paragraph{Ex.}
Apply the ratio test to 
\[
\sum_{ n=0 } ^{ \infty } \left( -1 \right) ^{ n }\frac{ \sqrt{ n} }{ n+1 }
.\] 
Note that our sequence can be called $ a_n $. Now if we sub n for n+1 we can now take the limit,
\[
\lim_{ n \to \infty} \left\| \frac{ a_{ n+1 } }{ a_n } \right\| = \lim_{ n \to \infty} \frac{ \sqrt{ n+1} }{ n+2 }\cdot \frac{ n+1 }{ \sqrt{ n} }= \sqrt{ \frac{ n+1 }{ n }} \cdot \frac{ n+1 }{ n+2 } = 1\cdot 1 = 1
.\] 
Which makes our test inconclusive. This just means that we have to use another test to determine convergence or divergence. \\
Instead looking at our sequence, we can use AST to find that it's monotonically decreasing and that the limit is 0. This means that the series converges by AST. So because $ \frac{ \sqrt{ n} }{ n+1 } $ is monotonically decreasing and $ \lim_{ n \to \infty} \frac{ \sqrt{ n} }{ n+1 } = 0 $, we can say that the series converges by AST.

\paragraph{What if we also wanted to find if the absolute also converges?}
\[
\sum_{ } ^{  } |a_n| = \sum_{  } ^{  } \frac{ \sqrt{ n} }{ n+1 }
.\] 
Comparing to a smaller function $ \sum_{  } ^{  } \frac{ 1 }{ n+1 } $ shows that the series converges by p-test. This means that the series conditionally converges.
\subsection*{11.5.4}%
\label{sub:11.5.4}
\paragraph{The root test }
(i) If $ \lim_{ n \to \infty} \sqrt[ n ]{ \left\| a_n \right\| } <1 $, then $ \sum_{  } ^{  } \left\| a_n \right\| $ converges \\
(ii) If $ \lim_{ n \to \infty} \sqrt[ n ]{ \left\| a_n \right\| } >1 $, then $ \sum_{  } ^{  } a_n $ diverges \\
(iii) If $ \lim_{ n \to \infty} \sqrt[ n ]{ \left\| a_n \right\| } =1 $, then inconclusive. \\ \\ 
\paragraph{Ex.}
\[
\sum_{ n=1 } ^{ \infty } \left( \frac{ n+1 }{ 2n } \right) ^{ n }
.\] 
We commonly use this test when we have something complex to the power of n, so,
\[
\sqrt[ n ]{ \left\| a_n \right\| } =\sqrt[ n ]{ \frac{ n+1 }{ 2n }^{ n } } = \lim_{ n \to \infty} \frac{ n+1 }{ 2n }=\frac{ 1 }{ 2 } < 1
.\] 
Which proves that our sum $ \sum_{  } ^{  } a_n $ converges because its less than 1.

\paragraph{Ex.}
\[
\sum_{ k=1 } ^{ \infty } \left( 1+\frac{ 3 }{ k }  \right) ^{ k^2 }
.\] 
Start by applying the root,
\[
\sqrt[ k ]{ \left\| a_n \right\| } = \sqrt[ k ]{ \left( 1+\frac{ 3 }{ k }  \right) ^{ k^2 } } = \lim_{ n \to \infty}  \left( 1+\frac{ 3 }{ k }  \right) ^{ k }
.\] 
Solving this at the above point makes it indecisive. So we can use the ratio test to find that the series converges. Let our limit equal L, then,
\[
k \ln^{  } \left( 1+\frac{ 3 }{ k }  \right) = \ln^{  } \left( L \right) 
.\] 
Now as $ k \to \infty $ we get an indeterminate form so we can use L'Hopital's rule to get,
\[
k\ln^{  } \left( 1+\frac{ 3 }{ k }  \right) = \frac{ \frac{ 1 }{ 1+\frac{ 3 }{ k }  }\cdot \left( -\frac{ 3 }{ k^2 }  \right)  }{ -\frac{ 1 }{ k^2 }  } = \frac{ 3 }{ 1+\frac{ 3 }{ k }  } \to 3 = \ln^{  } \left( L \right) 
.\] 
So $ L = e^{ 3 } $, which means that the series diverges because $ e^{ 3  }>1 $.
\paragraph{Everything above is done up to section 11.6 and 11.7 is just review on how to actually use these tests}

\paragraph{Example list to do if wanted (compare first to second to prove divergence or convergence)}
\begin{align*}
	1.& \sum_{  } ^{  } \frac{ 1 }{ 5^{ n } } , \sum_{  } ^{  } \frac{ 1 }{ 5^{ n }+n } \\
2.&\sum_{  } ^{  } \frac{ \left( -1 \right) ^{ n } }{ n^{ \frac{ 3 }{ 2 }  } }, \sum_{  } ^{  } \frac{ 1 }{ n^{ \frac{ 3 }{ 2 }  } } \\
3.&\sum_{  } ^{  } \frac{ n }{ r^{ n } } , \sum_{  } ^{  } \frac{ 3^{ n } }{ n } \\
4.&\sum_{  } ^{  } \frac{ n+1 }{ n }, \sum_{  } ^{  } \left( -1 \right) ^{ n } \frac{ n+1 }{ n } \\
5.&\sum_{ n=1 } ^{  } \frac{ n }{ n^2+1 } , \sum_{  } ^{  } \left( \frac{ n }{ n^2+1 }  \right) ^{ n } \\
6.&\sum_{  } ^{  } \frac{ \ln^{  } \left( n \right)  }{ n }, \sum_{ n=10 } ^{  } \frac{ 1 }{ n\ln^{  } \left( n \right)  } \\
7.&\sum_{  } ^{  } \frac{ 1 }{ n+n! } , \sum_{  } ^{  } \left( \frac{ 1 }{ n } +\frac{ 1 }{ n! }  \right) \\
8.&\sum_{  } ^{  } \frac{ 1 }{ \sqrt{ n^2+1} } , \sum_{ } ^{  } \frac{ 1 }{ n\sqrt{ n^2+1} } \\
\end{align*}
\section*{Power series}%
\label{sec:Power series}

\paragraph{Def.}
A power series is an infinite series of the form 
\[
\sum_{ n=0 } ^{ \infty } \underbrace{ c_n }_{ \text{ sequence } } \cdot  \underbrace{ x^{ n } }_{ \text{ var to an index power } } = c+ c_1x+ c_2x^2+ c_3x^3+ \ldots
.\] 
\paragraph{Ex. Power series}
For what values x will $ \sum_{ n=0 } ^{ \infty } \frac{ nx^{ n } }{ 4^{ n } }$ converge? \\
After defining our $ a_n $ to be $ \frac{ nx^{ n } }{ 4^{ n } } $, we can use the ratio test to find the convergence of the series, so,
\[
\lim_{ n \to \infty} \left\| \frac{ a_{ n+1 } }{ a_n } \right\| < 1 \to \lim_{ n \to \infty} \frac{ \left( n+1 \right) |x^{ n+1 }| }{ 4^{ n+1 } }\cdot \frac{ 4^{ n } }{ \left( n \right) |x|^{ n } }
.\] 
\[
\lim_{ n \to \infty} \frac{ n+1 }{ n }\cdot \frac{ 1 }{ 4 } \cdot \left| x \right| < 1
.\] 
which becomes
\[
1\cdot \frac{ 1 }{ 4 } \cdot \left| x \right| < 1 \to -4 < x < 4
.\] 
However, even though our inequality is solved, we still need to test our values of 4 and -4. So first for -4, our sum becomes,
\[
\sum_{ n=0 } ^{ \infty } \frac{ n\left( -4 \right) ^{ n } }{ 4^{ n } }= \sum_{ n=0 } ^{ \infty } \left( -1 \right) ^{ n }\cdot n
.\] 
Which we can take the limit of and find that $ \lim_{ n \to \infty} a_n = \pm \infty $. Now testing for 4, 
\[
\sum_{  } ^{  } \frac{ n\left( r \right) ^{ n } }{ 4^{ n } }= \sum_{  } ^{  } n  = \infty
.\] 
which also diverges. Now we know our interval of convergence is $ -4 < x < 4 $ and our radius of convergence will be 4.

\paragraph{Ex.}
\[
\sum_{ n=0 } ^{ \infty } \left( -1 \right) ^{ n } \frac{ x^{ 2n+1 } }{ \left( 2n+1 \right) ! }
.\] 
\paragraph{Find the domain/interval of convergence.}
Start by applying our ratio test to find the interval of convergence.
\[
\lim_{ n \to \infty} \left| \frac{ a_{ n+1 } }{ a_n } \right|= \lim_{ n \to \infty} \left| \frac{ x^{ 2\left( n+1 \right) +1 } }{ \left( 2\left( n+1 \right) +1 \right) ! } \right| * \frac{ \left( 2n+1 \right) ! }{ \left| x^{ 2n+1 } \right| }
.\] 
\[
\lim_{ n \to \infty} \left| x^2 \right|\cdot \frac{ \left( 2n+1 \right) !}{ \left( 2n+3 \right) \left( 2n+2 \right) \left( 2n+1 \right) ! } < 1 \implies \left| x^2 \right|\cdot 0< 1 
.\] 
Which means that our domain is all real numbers and our radius will be $ \infty $. Expect to commonly use ratio test

\paragraph{Ex.}
\[
\sum_{ n=1 } ^{ \infty } \frac{ x^{ n } }{ n }
.\] 
Instead using root test, we take the $ n^{ th }$ root and take the limit as its less than 1
\[
\lim_{ n \to \infty} \sqrt[ n ]{ \left| a_n \right| } < 1
.\] 
\[
=\lim_{ n \to \infty} \sqrt[ n ]{ \left| \frac{ x^{ n } }{ n } \right| } = \lim_{ n \to \infty} \frac{ \left| x \right| }{ \sqrt[ n ]{ n }  } = \lim_{ n \to \infty} \frac{ 1 }{ \sqrt[ n ]{ n }  } \cdot \left| x \right| < 1
.\] 
Start by taking the limit of our root,
\[
\lim_{ n \to \infty} \sqrt[ n ]{ n } =L
.\] 
\[
\lim_{ n \to \infty} \frac{ 1 }{ n } \ln^{  } \left( n \right) = \ln^{  } \left( L \right)
.\] 
And by L'hopital,
\[
\frac{ \frac{ 1 }{ n }  }{ 1 }\to 0
.\] 
\[
0=\ln^{  } \left( L \right) \implies e^{ 0 }= L = 1
.\] 
Now going back to our original limit,
\[
	\underbrace{\lim_{ n \to \infty} \frac{ 1 }{ \sqrt[ n ]{ n }  } }_{1} \cdot \left| x \right|
.\] 
Meaning our domain will be $ \left( -1,1 \right)  $. Now testing our values of -1 and 1,
\[
\sum_{ n=1 } ^{ \infty } \frac{ \left( -1 \right) ^{ n } }{ n }
.\] 
Will converge by AST because $ b_n $ is monotonic decreasing and $ \lim_{ n \to \infty} b_n = 0 $. Now for 1,
\[
\sum_{ n=1 } ^{ \infty } \frac{ 1 }{ n } 
.\] 
Which will be the harmonic series and will be divergent because of p test, leading our interval to be $ \left[ -1,1 \right) $

\paragraph{Ex.}
\[
\sum_{ n=0 } ^{ \infty } \frac{ x^{ 2n } }{ \left( -9 \right) ^{ n } }
.\] 
Start with root test,
\[
\lim_{ n \to \infty} \sqrt[ n ]{ \left| \frac{ x^{ 2n } }{ \left( -9 \right) ^{ n } } \right| } \to \lim_{ n \to \infty} \left| \frac{ x^2 }{ 9 }  \right| < 1
.\] 
\[
\left| x^2 \right|<9 \to \left| x \right| < 3 \to -3< x < 3
.\] 
Where we can now test our endpoints. For -3,
\[
\sum_{ } ^{  } \frac{ \left( -3 \right)^{ 2n }  }{ \left( -9 \right) ^{ n } }= \sum_{  } ^{  } \left( -1 \right) ^{ n }\left( 1 \right) 
.\] 
Which will diverge by divergence test because the limit will not go to 0. Now for 3,
\[
\sum_{  } ^{  } \frac{ \left( -3 \right) ^{ 2n } }{ \left( -9 \right) ^{ n } }= \sum_{  } ^{  } \left( -1 \right) ^{ n }\left( 1 \right) 
.\] 
Which will also diverge leading our interval of convergence to be $ -3 < x < 3 $.
\section*{11.8 Power series cont}%
\label{sec:11.8 Power series cont}
\subsection*{11.8.1 Derivatives in calc 1}%
\label{sub:11.8.1 Derivatives in calc 1}
\paragraph{When we had} some thing like \[
\frac{ d }{ dx } \left( f\left( x \right) \pm g\left( x \right)  \right) 
.\] 
we could distribute the $ \frac{ d }{ dx }  $ which shows linearity of the distribution function. So what happens when we take a $ \frac{ d }{ dx }  $ of the power series $ \sum_{ n=0 } ^{ \infty } C_n \left( x \right) ^{ n } $? This will become something like
\[
=\frac{ d }{ dx } \left( C_1 \cdot x^{ 0 }+c_2x^{ 1 }+c_2x^2+c_3x^3 \ldots c_nx^{ n }\right) 
.\] 
Where we can simplify it to 
\[
= c_1+2c_2x+3c_3x^2+4c_4x^3 \ldots nc_nx^{ n-1 } \implies \sum_{ n=1 } ^{ \infty } c_nx^{ n-1 }
.\] 
Or,
\[
\sum_{ n=1 } ^{ \infty } c_nn\left( x \right) ^{ n-1 }
.\] 
Note that this sholdn't start at $ n=0 $ because the first term will just be 0 thanks to our n multiplyer. 
\paragraph{Power Series for the function $ e^{ x } $ \\}
Start with the function $ y' = y\text{, } y\left( 0 \right) =1 $ whose solution will be $ y=e^{ x } $. Now we can call $ f'\left( x \right) = f\left( x \right)  $ or $ \sum_{ n=0 } ^{ \infty } a_nx^{ n } = a_{ 0 }+a_1x + a_2x^2 + a_3x^3 \ldots$ This sum can now be derived to get 
\[
f'\left( x \right) = \sum_{ n=1 } ^{ \infty } a_n \cdot nx^{ n-1 }= a_1\left( 1 \right) +a_2\left( 2x \right) +a_3\left( 3x^2 \right) \ldots
.\] 
Where we can now set terms equal to each other,
\begin{align*}
a_0 &= a_1 \\
a_1x &=  a_2\left( 2x \right)  \\
a_2x^2 &= a_3 3x^2 \\
\ldots
.\end{align*}
Now we can use algebra on each term to find
\[
a_2 = \frac{ a_1 }{ 2 }, a_3 = \frac{ a_2 }{ 3 } = \frac{ a_1 }{ 2\cdot 3 }, a_4 = \frac{ a_3 }{ 4 } = \frac{ a_1 }{ 2\cdot 3\cdot 4 } \ldots a_n = \frac{ a_1 }{ n\cdot \left( n-1 \right) \ldots 3 \cdot 2\cdot 1 }
.\] 
which gives us the power series but we still want to find $ a_1 $, so we can find the sum to cancel all terms other than $ a_0 $ to find that our $ a_n = \frac{ 1 }{ n! }  $. Which gives us the final equation,
\[
e^{ x }= \sum_{ n=0 } ^{ \infty } \frac{ 1 }{ n! } x^{ n }
.\] 
Now that we have the summation we can look at our expansion, 
\[
\sum_{ n=0 } ^{ \infty } \frac{ x^{ n } }{ n! }= 1 + \frac{ x^{ 1 } }{ 1! }+ \frac{ x^{ 2 } }{ 2! }+ \frac{ x^{ 3 } }{ 3! }+ \ldots
\] 
Which can be graphed to approximate the function $ e^{ x } $. Questions can appear using this on the next quiz where we have to find an interval of x values that will give a certain range of error like $ 10^{ -3 } $. Note that these are called transedental functions ($ e^{ x }$) and we are converting them to simple ones with our summation. A common one in physics is using x to approximate $ \sin^{  } \left( x \right)  $ for values close to 0. 

\paragraph{Ex.}
\paragraph{Prove that}
\[
\ln^{  } \left( 1+x \right) = \sum_{ n=1 } ^{ \infty } \left( -1 \right) ^{ n-1 }\frac{ x^{ n } }{ n } \text{ when } \left| x \right|<1
.\] 
Start with the function $ \sum_{ n=0 } ^{ \infty } x^{ n } $ which converges to $ \frac{ 1 }{ 1-x }  $ for the values of $ \left| x \right|< 1 $. We essentially want to build the function in order to get from $ \frac{ 1 }{ 1-x }  $ to the $ \ln^{  } \left( 1-x \right)  $. As our first step let's make the x negative,  $ \sum_{ n=0 } ^{ \infty } \left( -x \right) ^{ n } =\frac{ 1 }{ 1+x }  $. This doesn't really get us anywhere so what if we instead try something else. \\
Let's instead make it $ \frac{ 1-x^2 }{ 1-x }= 1+x $. To get us here we need to simplify $ \left( 1-x^2 \right) \left( \sum_{ n=0 } ^{ \infty } x^{ n } \right)  $. So,
\[
=\left( 1-x^2 \right) \left( 1+x+x^2+x^3\ldots \right) 
.\] 
Now looking at our sequence we can see that each multiplication by one will give us a term in the sequence. So we can simplify this to 
\[
\sum_{ n=0 } ^{ \infty } x^{ n }+\left( -x^2-x^3-x^{ 4 }-x^{ 5 }\ldots \right) = 1+x
.\] 
This gives us the inside of our $ \ln^{  } \left( 1+x \right)  $, but we still need the rest which we will come back to later. 

\subsection*{11.9}%
\label{sub:11.9}
\paragraph{The geometric series identity}
\[
\sum_{ n=0 } ^{ \infty } x^{ n }= \frac{ 1 }{ 1-x } 
.\] 
This is one of the most useful identities for making a power series because we can use it to build them easily. For example,
\[
\frac{ 1 }{ 1+2x } =\frac{ 1 }{ 1-\left( -2x \right)  } = \sum_{ n=0 } ^{ \infty } \left( -2x \right) ^{ n } = \sum_{ n=0 } ^{ \infty } \left( -1 \right) ^{ n }\left( 2x \right) ^{ n }
.\] 
\paragraph{Ex.}
\paragraph{Make the below into a power series.}
\[
\frac{ 1 }{ 2+x^2 } 
.\] 
So let's start by factoring out our two,
\[
\frac{ 1 }{ 2\left( 1+\frac{ x^2 }{ 2 } \right)  } = \frac{ 1 }{ 2\left( 1-\left( -\frac{ x }{ 2 }  \right) ^2 \right)  } = \frac{ 1 }{ 2 } \left[ \frac{ 1 }{ 1-\left( -\frac{ x^2 }{ 2 }  \right)  }  \right] = \frac{ 1 }{ 2 } \sum_{ n=0 } ^{ \infty } \left( -\frac{ x^2 }{ 2 }  \right)^{ n } 
.\] 
\[
= \frac{ 1 }{ 2 } \sum_{ n=0 } ^{ \infty } \left( -1 \right) ^{ n }\left( \frac{ x^2 }{ 2}  \right) ^{ n }
.\] 
What if we wanted to find our interval of convergence? We can start with the ratio test,
\begin{gather*}
\lim_{ n \to \infty} \left| \frac{ a_{ n+1 } }{ a_n } \right| < 1\\
\left| \frac{ \left( \frac{ x^2 }{ 2 }  \right) ^{ n+1 } }{ \left( \frac{ x^2 }{ 2 }  \right) ^{ n } } \right|<1 \to \left| \left( \frac{ x^2 }{ 2 }  \right)  \right|<1 \to \left| x^2 \right|<2\\
\end{gather*}
Which gives the interval $ -\sqrt{ 2}<x<\sqrt{ 2} $ where we can now test our intervals and easily see that they will diverge because the function doesn't go to 0. 
\paragraph{Ex.}
\[
\frac{ x }{ 16+2x^3 } 
.\] 
Lets start by removing an x and a 16,
\[
x\left( \frac{ 1 }{ 16+2x^3 }  \right) =\frac{ x }{ 16 } \left( \frac{ 1 }{ 1+\frac{ x^3 }{ 8 }  }  \right) = \frac{ x }{ 16 } \left( \frac{ 1 }{ 1-\left( -\frac{ x^3 }{ 8 }  \right)  }  \right) = \frac{ x }{ 16 } \sum_{ n=0 } ^{ \infty } \left( -\frac{ x^3 }{ 8 }  \right) ^{ n }
.\] 
Which can now be written as the alternating series,
\[
\frac{ x }{ \underbrace{ 16 }_{ 2^{ 4 } }  } \sum_{ n=0 } ^{ \infty } \frac{ \left( -1 \right) ^{ n }x^{ 3n } }{ \underbrace{ 8^{ n } }_{ 2^{ 3n } }  } = \sum_{ n=0 } ^{ \infty } \frac{ \left( -1 \right) ^{ n } x^{ 3n+1 }}{ 2^{ 3n+3 } }
.\] 
The IOC (interval of convergence) here is (-2,2) but wasn't proven in class because it would take a while. Just use ratio test if you want to prove it on our snow days. His quote not mine. Actually one more,\\
\paragraph{Ex. $ \tan^{ -1 } \left( x \right)  $. }
Using $ \int_{  }^{  } \frac{ 1 }{ 1+x^2 } dx= \tan^{ -1 } \left( x \right)  $, how can we make a power series of,
\[
\frac{ 1 }{ 1+x^2 } = \frac{ 1 }{ 1 - \left( -x^2 \right)  } = \sum_{ n=0 } ^{ \infty } \left( -1 \right) ^{ n }x^{ 2n }
.\] 
Now we want to integrate both sides, remembering that integrals are linear as well,
\[
\int_{  }^{  } \frac{ 1 }{ 1+x^2 } = \sum_{ n=0 } ^{ \infty } \left( -1 \right) ^{ n }\int_{  }^{  } x^{ 2n }dx
.\] 
Note that this happens because n is constant with respect to x so there is no point in keeping it inside the integral in this case. Now,
\[
\int_{  }^{  } \left( 1-x^2 + x^{ 4 }- x^{ 6 } + x^{ 8 }\ldots\right) 
.\] 
Which becomes,
\[
C + x-x^3 + \frac{ x^5 }{ 5 } - \frac{ x^7 }{ 7 } + \frac{ x^9 }{ 9 } \ldots
.\] 
Which can be simplified to,
\[
C+ \sum_{ n=1 } ^{ \infty } \left( -1 \right) ^{ n-1 }\frac{ x^{ 2n-1 } }{ 2n-1 } = \tan^{ -1 } \left( x \right) 
.\] 
And to find our initial condition we can plug in 0 to find that $ C=0 $ because $ \tan^{ -1 } \left( 0 \right) =0 $. Which proves that 
\[
\tan^{ -1 } \left( x \right) = \sum_{ n=1 } ^{ \infty } \left( -1 \right) ^{ n }\frac{ x^{ 2n-1 } }{ 2n-1 }
.\] 
