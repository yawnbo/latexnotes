\ex{
Make a Taylor Polynomial (4 terms) for $ \cos^{  } \left( x \right)  $ with $ c=-\frac{ \pi }{ 2 }  $
}
{
	Start with $ f\left( x \right) =\cos^{  } \left( x \right)  $ and write out some derivatives while evaluating at $ -\frac{ \pi }{ 2 }  $.
	\begin{align*}
		f^{ 0 }\left( x \right) &=\cos^{  } \left( x \right) \to 0 \\
		f^{ 1 }\left( x \right) &= -\sin^{  } \left( x \right) \to 1 \\
		f^{ 2 }\left( x \right) &=-\cos^{  } \left( x \right) \to 0 \\
		f^{ 3 }\left( x \right) &= \sin^{  } \left( x \right) \to -1 \\
		f^{ 4 }\left( x \right) &= \cos^{  } \left( x \right) \to 0
	.\end{align*}
	With this we can write our polynomial as
	\[
	\frac{ x+\frac{ \pi }{ 2 }  }{ 1 }-\frac{ x+\frac{ \pi }{ 2 }  }{ 3! }+\frac{ x+\frac{ \pi }{ 2 }  }{ 5! }-\frac{ x+\frac{ \pi }{ 2 }  }{ 7! }
	.\] 
	Now that we have our taylor, whats the max error to expect $ T_7\left( x \right)  $ of $ \cos^{  } \left( x \right)  $ to have over $ \left[ -\pi ,\frac{ \pi }{ 4 }  \right]  $?
	Using Taylor's inequality we can write our starting equation as
	\[
	\left| T_7\left( x \right)  \right|\le \frac{ M }{ \left( n+1 \right) ! } \left| x-c \right|^{ n+1 }
	.\] 
	Getting our center we find, $ c= \frac{ -\pi + \frac{ \pi }{ 4 }  }{ 2 } = -\frac{ 3\pi }{ 8 }  $ and our max value as $ max\left( f^{ n+1 }\left( x \right) \text{ over } \left[ -\pi , \frac{ \pi }{ 4 }  \right] \right) = f^{ 8 }\left( x \right) =\cos^{  } \left( M \right)  $ and because $ \cos^{  } \left( x \right)  $ has a max at $ M=1 $ over $ \left[ -\pi , \frac{ \pi }{ 4 }  \right]  $ we can write our max error as
	\[
	\frac{ 1 }{ 8! } \left| x+\frac{ \pi }{ 8 }  \right|^{ 8 } \to \frac{ 1 }{ 8! } \left( \frac{ 5\pi }{ 8 }   \right) ^{ 8 }
	.\] 

} 
\section*{Exam review for midterm 1}%
\label{sec:Exam review for midterm 1}
\subsection*{1.f}%
\label{sub:1.f}
Done with the theorem that $ \lim_{ n \to \infty} \sqrt[ n ]{ a_n } = \lim_{ n \to \infty} \frac{ a_{ n+1 } }{ a_n } $, so
\[
\lim_{ n \to \infty} \sqrt[ n ]{ \frac{ n }{ 2 }  } = \lim_{ n \to \infty} \frac{ \frac{ n+1 }{ 2 } }{ \frac{ n }{ 2 }  }= \lim_{ n \to \infty} \frac{ n+1 }{ n }=1
.\] 
Instead using natural logs we can do it as
\[
\lim_{ n \to \infty} \sqrt[ n ]{ \frac{ n }{ 2 }  } = L \to \frac{ 1 }{ n } \ln^{  } \left( \frac{ n }{ 2 }  \right) =\ln^{  } \left( L \right) 
.\] 
Taking L'Hopital's rule we get
\[
\frac{ \frac{ 1 }{ \frac{ n }{ 2 } } \cdot \frac{ 1 }{ 2 }  }{ 1 } = \frac{ 1 }{ n } = 0 \implies L = u ^{ 0 }=1
.\] 
Note that this can help with using root test a lot because at most points you will get to something in terms of the $ n $ th root. 

\subsection*{q3 question 7}%
\label{sub:q3 question 7}
\[
\sum_{ n=1 } ^{ \infty } \frac{ \left( -1 \right) ^{ k+1 }k }{ 3k+1 }
.\] 
Checking for absolute convergence we use divergence thm and see that for the absulte, $ b_n $ will not approach 0.\\
For conditional we can use something that if $ \lim_{ n \to \infty} a_{ 2k }\neq \lim_{ n \to \infty} a_{ 2k+1 } $. 
