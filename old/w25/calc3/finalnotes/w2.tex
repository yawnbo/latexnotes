
\section{Limits of recursive sequences}%
\label{sec:Limits of recursive sequences}

\paragraph{Ex.}%
\label{par:Ex.}

\[
a_1 = \sqrt{ 3}, a_2 = \sqrt{ 3 \cdot \sqrt{ 3}}, a_3 = \sqrt{ 3 \cdot \sqrt{ 3 \cdot \sqrt{ 3}}}, \ldots
.\] 

\paragraph{Definition\\}

If we let $ a_1 = \sqrt{ 3} $ Then for $ n>1 $: $ a_n = \sqrt{ 3\cdot a_{ n-1 }} $.
This creates our nesting and we can begin to solve for our limit. 

\paragraph{First \\}
We want to prove that $ a_n $ has a limit. For this, we need to prove that the sequence is monotonic (increase/decrease) and that we are bounded somewhere along the sequence. When these two parts are proven together, we know that we have convergence at some point. \\

For this we can just write out a few terms, so,
\[
a_1 = \sqrt{ 3} \approx 1.77, a_2 = \sqrt{ 3\cdot \sqrt{ 3}} \approx 2.28, a_3 = \sqrt{ 3\cdot \sqrt{ 3\cdot \sqrt{ 3}}} \approx 2.62, a_{ 4 } \approx 2.8 , a_{ 5 } \approx 2.90
.\] 
This differs from how we would find the limit with a function because we can't take a solid $ \frac{ d }{ dx }  $ so we can just do this for now to find an approximation.\\

Due to the above, we can prove that we have an increasing sequence so we can define some upper limit/bound to prove. If we write out a few more terms, $ a_6 \approx 2.95 a_7 \approx 2.97 $ which seems to approach 3 which we can label as our upper bound. \\

This is all that is needed to prove that a limit EXISTS but we still need to find the limit. Let's assume that our limit is $ L $. Now we should know that,
\[
\lim_{ n \to \infty} a_n = L
.\] 
\[
\lim_{ n \to \infty} a_{ n-1 }= L 
.\] 
\newpage
Knowing that these should be equal, we can write out and solve for our limits
\[
a_n = \sqrt{ 3\cdot a_{ n-1 }}
.\] 
\[
\lim_{ n \to \infty} a_n = \sqrt{ 3\cdot \lim_{ n \to \infty} \cdot a_{ n-1 }}
.\] 
\[
L = \sqrt{ 3\cdot L}
.\] 
\[
L^2=3L \to l^2-3l = 0 \to L(L-3) = 0
.\] 
Which shows that $ L = 0 \text{ or }3 $, but we ignore the 0 because our sequence is non-zero. 

\section{Geometric sequences}%
\label{sec:Geometric sequences}
\paragraph{Recall}
that an arithmetic sequence is one that increases in the pattern of $ a_1 + m = a_2 $ where we add by a factor of m while a geometric sequence is one that increases in the pattern of $ a_1 \cdot r, a_1 \cdot r^2, a_1 \cdot r^{ 3 }$ where we instead multiply by a factor of r. Basically just common difference vs common ration. 

\paragraph{Convergence of $ \left\{ r^{ n } \right\}_{ n=1 }^{ \infty }  $.}
If we were to take the limit ($ \lim_{ n \to \infty} r^{ n }$), then we have a few cases to look at, first, if $ r=1 $ then our limit will just approach 1 because our sequence will look like $ 1^{ 1 }, 1^{ 2 },1^{ 3 },1^{ 4 }\ldots $ which will forever be one. At the case of $ r=-1 $, we will have a sequence that looks like $ -1^{ 1 }, -1^{ 2 }, -1^{ 3 }, -1^{ 4 }\ldots $ which will oscillate between 1 and -1 giving us a divergent sequence. If we instead have $ -1<r<1 $ then we will have a convergent sequence that will approach 0. If we have $ r>1 $ then we will have a divergent sequence that will approach positive $ \infty $ while for $ r<1 $ we would have $ \pm \infty $, but for each of these cases, the limits don't exist. 

\paragraph{Summary}
To end, $ r^{ n } \text{ converges for } r\epsilon\left( -1,1 \right] $.

\section{Sec 11.2 Series}%
\label{sec:Sec 11.2 Series}

\paragraph{Definition\\}
For $ \left\{ a_n \right\} ^{ \infty }_{ n=1 } $, the sum of its terms $ a_1 + a_2 + a_3 \ldots = \sum_{ n=1 } ^{ \infty } a_n$ is referred to as an infinite series or just a series. \\

Our goal with this section is to find out when our sum will equal a finite, infinite, or indeterminate value. Basically just convergence and divergence with fancy terms now. \\

An important thing to note is that,
\[
\sum_{ i=1 } ^{ n } a_{ i } \text{ is always finite}
.\] 

\paragraph{Ex.}
\[
\sum_{ i=1 } ^{ n } i= 1+2+3+4+5\ldots + \left( n-1 \right) + n = \frac{ n\left( n+1 \right)  }{ 2 }
.\] 

\newpage
\paragraph{Definition again\\}
For $ \sum_{ i=1 } ^{ \infty } a_i $, 
\[
\sum_{ i=1 } ^{ n } a_i = S_n \text{ which is the nth partial sum of }\sum_{ n=1 } ^{ \infty } a_n
.\] 

\paragraph{Ex. we can have a series like,}
\begin{align*}
&\sum_{ i=1 } ^{ n } a_i \\
S_1 &= \sum_{ i=1 } ^{ 1 } a_i = a_1 \\
S_2 &= \sum_{ i=1 } ^{ 2 } a_i = a_1 + a_2 \\
S_3 &= \sum_{ i=1 } ^{ 3 } a_i = a_1 + a_2 + a_3 \\
\ldots
S_n &= \sum_{ i=1 } ^{ n } a_i = a_1 + a_2 + a_3 + \ldots + a_n
\end{align*}
which is really just a sequence of partial sums. Going back to our definition we know that $ \lim_{ n \to \infty} \left\{ S_n \right\} =\sum_{ i=1 } ^{ \infty } a_i $. But if our limit DNE then our sum will diverge instead. 

\subsection{Going back to Geometric Series}%
\label{sub:Going back to Geometric Series}

This is a case where we would use our sum.
\[
\sum_{ n=1 } ^{ \infty } ar^{ n-1 }=a + ar^{ 1 } + ar^{ 2 } + ar^{ 3 } + \ldots = a\left( 1+r+r^{ 2 }+r^{ 3 }+\ldots \right)
.\] 
We can find our common ratio to be $ \frac{ ar^3 }{ ar^2 }=r $.\\

Now lets say we want to find when does $ \sum_{  } ^{  } ar^{ n-1 } $ converge? Lets try with some cases,
\begin{gather*}
r=1 \\
\sum_{ i=1 } ^{ \infty } ar^{ i-1 } = a + a + a + a + \ldots \\
\sum_{ i=1 } ^{ n } ar^{ i-1 }= a + a + a \ldots + a = na = \left\{ S_{ i } \right\}^{ \infty }_{ i=1 } 
\end{gather*}

Now taking the limit, 
\[
\lim_{ i \to \infty} \left\{ S_{ i } \right\} =\lim_{ n \to \infty} \sum_{ i=1 } ^{ \infty } a\left( 1 \right) = \lim_{ n \to \infty} na = \pm \infty
.\] 
We find that this diverges, unlike $ \left\{ a\left( 1 \right) ^{ n } \right\}  $ which converges. So adding our sum makes our series diverge. \\

Now lets try,
\begin{gather*}
r\neq 1: -1<r<1\\
\sum_{ i=1 } ^{ n } ar^{ i-1 }=a+ar+ar^2+ \ldots + ar^{ n-1 }= S_{ n }
\end{gather*}
But this doesn't get us anywhere so lets multiply the whole thing by r,
\[
\sum_{ i=1 } ^{ n } ar^{ i-1 }= ar + ar^2+ ar^3+ \ldots + ar^{ n }= S_n \cdot r
.\] 
Lets call the equation before introducing the r $ eq_1 $ and the second one $ eq_2 $. Subtracting $ eq_1 $ from $ eq_2 $ we get,
\[
a-ar^{ n } = S_n - S_n r = S_n\left( 1-r \right) \to
.\] 
\[
S_n=\frac{ a-ar^{ n } }{ 1-r } = \frac{ a\left( 1-r^{ n } \right)  }{ 1-r }
.\] 
Which proves that 
\[
\sum_{ i=1 } ^{ n } ar^{ n-1 }=S_n = \frac{ a\left( 1-r^{ n } \right)  }{ 1-r }
.\] 
\paragraph{Example where this is useful}
\[
\sum_{ i=1 } ^{ 100 } 2\left( \frac{ 3 }{ 7 }  \right) ^{ i-1 }
.\] 
With our formula we can plug in all our values to find that
\[
	\frac{ 2\left( 1-\frac{ 3 }{ 7 }  \right)^{ 100 } }{ 1-\frac{ 3 }{ 7 }  }
.\] 	
Typo on quiz, Question 6 should be 
\[
a_{ n+1 }=\sqrt{ 2+a_n} \text{ (same thing just without the root) }
.\] 
\section*{Sec 11.2 Series cont.}%

\paragraph{Geometric: Partial sums}
\[
\sum_{ i=1 } ^{ \infty } ar^{ n-1 }=\frac{ a\left( 1-r^{ n } \right)  }{ 1-r }
.\] 
Now our goal is to find the answer to 
$\lim_{ n \to \infty} \sum_{ i=1 } ^{ \infty } ar^{ n }=\lim_{ n \to \infty} S_n$
\subsection{Ex.}%
\label{sub:Ex.}
\paragraph{Calculate}
\[
\sum_{ i=1 } ^{ 20 } 3\left( \frac{ 1 }{ 2 }  \right) ^{ n-1 }=\frac{ 3\left( 1-\frac{ 1 }{ 2 }^{ 20 }  \right)  }{ 1-\frac{ 1 }{ 2 }  }	\approx 5.999994278
.\] 

\section{Result of a geometric series}%
\[
\sum_{ i=1 } ^{ \infty } ar^{ n-1 }=\lim_{ n \to \infty} S_n=\lim_{ n \to \infty} \frac{ a\left( 1-r^{ n } \right)  }{ 1-r }=\frac{ a }{ 1-r } ,r\epsilon\left( -1,1 \right) 
.\] 

\section{Examples}%
\label{sec:Examples}
\paragraph{Find the sum of }
\[
5-\frac{ 5 }{ 4 } +\frac{ 5 }{ 16 } -\frac{ 5 }{ 64 } + \ldots
.\] 
Our test should be $ a_{ n+1 } = r $ where r is our common ratio. In this case we would get $ \frac{ -\frac{ 5 }{ 4 }  }{ 5 } = -\frac{ 1 }{ 4 } $. So our sum would be

\[
\sum_{ i=1 } ^{ \infty } 5\left( -\frac{ 1 }{ 4 }  \right) ^{ n-1 }
.\] 

so we use our formula to get
\[
\frac{ a }{ 1-r } =\frac{ 5 }{ 1-\left( -\frac{ 1 }{ 4 }  \right)  } =4
.\] 
This is also one of the few infinite sums that we can actually compute with no issues. 

\newpage
\subsection{Ex}%
\label{sub:Ex}
\[
3-4+\frac{ 16 }{ 3 } -\frac{ 64 }{ 9 } +\ldots \text{ is geometric, and Does it converge? }
\] 

Our ratio can be found to be $ -\frac{ 4 }{ 3 }  $ and if we check this value, we find it's outside our range of $ r\epsilon\left( -1,1 \right)  $. This means that sum will diverge and we will not have a finite value. 

\subsection{Ex}%
\label{sub:Ex}
\[
\text{ Given the sequence }-1\left( 2+\frac{ 2 }{ 3 } +\frac{ 2 }{ 9 } +\frac{ 2 }{ 27 } +\ldots \right) 
.\] 
We can find our r value to be $ \frac{ 1 }{ 3 }  $ and we can plug this into our equation of $ \frac{ a }{ 1-r } $ to find
\[
-\frac{ 2 }{ 1-\frac{ 1 }{ 3 }  } =-3
.\]

\subsection{Ex}%
\label{sub:Ex}
\paragraph{Compute}
\[
\sum_{ i=0 } ^{ \infty } \frac{ 1 }{ r^{ n+1 } }
.\] 
For this we can use something called a Reindex. This is just getting a dummy variable so we get our desired starting value. So for this we can change our equation to be $ m=n+1 $ and change our sum to be $ \sum_{ m=1 } ^{ \infty } \frac{ 1 }{ r^{ m } }  $. But now we need to adjust it to have an exponent that matches our formula so we take out a constant to leave us with
\[
\frac{ 1 }{ 3 } \cdot \frac{ 1 }{ 3^{ m-1 } } =\frac{ 1 }{ 3 } \sum_{ m=1 } ^{ \infty } \frac{ 1 }{ 3^{ m-1 } } =\frac{ 1 }{ 3 } \left( \frac{ 1 }{ 1-3 }  \right) 
.\] 
Now computing this we are left with
\[
\frac{ 1 }{ 3 } \left( \frac{ 3 }{ 2 }  \right) =\frac{ 1 }{ 2 } \text{ as our final limit. }
\] 

\subsection{Ex.}%
\label{sub:Ex.}
\paragraph{What if we instead started at $ i=5 $?}
\paragraph{Compute}
\[
\sum_{ n=5 } ^{ \infty } 2\left( \frac{ 1 }{ 4 }  \right) ^{ n+3 }
.\] 
So we take our reindex to be $ m=n-4 $ and rewrite our sum. We need to remember to solve for $ n $ and replace it in our equation, so we get $ n=m+4 $. So,
\[
=\sum_{ m=1 } ^{ \infty } 2\left( \frac{ 1 }{ 4 }  \right) ^{ m+7 }=\sum_{ m=1 } ^{ \infty } 2\left( \frac{ 1 }{ 4 }  \right) ^{ m+7 }
.\] 
Now we need to adjust our exponent to be left with $ m-1 $, so we take out a $ \left( \frac{ 1 }{ 4 }  \right) ^{ 8 } $ and be left with 
\[
	\left( \frac{ 1 }{ 4 }  \right) ^{ 8 }\sum_{ m=1 } ^{ \infty } 2\left( \frac{ 1 }{ 4 }  \right) ^{ m-1 }=\left( \frac{ 1 }{ 4 }  \right) ^{ 8 }\cdot \frac{ 2 }{ 1-\frac{ 1 }{ 4 }  } = \left( \frac{ 1 }{ 4 }  \right) ^{ 8 }\cdot \frac{ 2\cdot 4 }{ 3 }=\frac{ 2 }{ 3\cdot 4^{ 7 } } \text{ or }\frac{ 2 }{ 49152 } 
.\] 

\section{Decimal system}%
\label{sec:Decimal system}
The Decimal System is an application of a geometric series. For example, given the number 542, we can think of this as $ 2\cdot 10^{ 2 }+4\cdot 10^{ 1 }+2\cdot 10^{ 0 } $.

\paragraph{Ex. Write 0.345 as a sum of powers of 10. }

For this we would just use negative powers so we would write 
\[
0.346 = 3\cdot 10^{ -1 }+4\cdot 10^{ -2 }+6 \cdot 10^{ -3 } \text{ which can also just be thought of as fractions }= \frac{ 3 }{ 10 } +\frac{ 4 }{ 10^2 } \ldots
.\] 

\subsection{Repeated decimals}%
\label{sub:Repeated decimals}
\paragraph{Write $ 3.1454545\ldots $ as a fraction. }
Let's just tear it apart into power of -10
\[
=3.1 + \frac{ 45 }{ 10^{ 3 } } +\frac{ 45 }{ 10^{ 5 } } +\frac{ 45 }{ 10^{ 7 } } \ldots
.\] 

The last part of our fraction is what we can consider a geometric series, so let's set up our sum.
\[
3.1+\sum_{ n=1 } ^{ \infty } 45\left( \frac{ 1 }{ 10 }  \right) ^{ 2n+1 }
.\] 
Now we need to normalize our current sum to be in the form of $ \frac{ a }{ 1-r } $ so, we can find our exponent to be 
\[
\frac{ 1 }{ 10 } ^{ 2n+1 }=\left( \frac{ 1 }{ 10 }  \right) ^{ 2n }\cdot \left( \frac{ 1 }{ 10 }  \right) ^{ 1 }=\left( \frac{ 1 }{ 10^2 }  \right) ^{ n }\cdot \frac{ 1 }{ 10 } =\frac{ 1 }{ 10^3 } \left( \frac{ 1 }{ 10^2 }  \right)^{ n-1 }
.\] 
Now we rewrite our sum to be 
\[
3.1+\frac{ 1 }{ 10^3 } \sum_{ n=1 } ^{ \infty } 45\left( \frac{ 1 }{ 10^2 }  \right) ^{ n-1 }
.\] 
Time for awesome simplifications
\[
3.1+\frac{ 1 }{ 10^3 } \left( \frac{ 45 }{ 1-\frac{ 1 }{ 10^2 }  }  \right)=3.1+ \frac{ 45 }{ 990 } = \frac{ 173 }{ 55 }  
.\] 

\section{Power Series}%
\label{sec:Power Series}
\[
\sum_{ n= 0 } ^{ \infty } x^{ n } = 1 + x + x^2+x^3+ \ldots
.\] 
The above is what we call a power series. Because this geometric we can write our new identity as 
\[
\sum_{ n=0 } ^{ \infty } x^{ n }=\sum_{ m=1 } ^{ \infty } 1\left( x \right) ^{ m-1 }=\frac{ 1 }{ 1-x } ,x\epsilon\left( -1,1 \right) 
.\] 

\subsection{Example}%
\label{sub:Example}
\[
\sum_{ n=1 } ^{ \infty } \left( -3 \right) ^{ n }x^{ n }
.\] 
\paragraph{For what values, $ x $, will the series converge?}
Because our power is too big on the x, we want to reindex and call $ m=n+1 $. With this we can rewrite our sum to be 
\[
\sum_{ m=0 } ^{ \infty } \left( -3x \right) ^{ m+1 }=\left( -3x \right) \sum_{ m=0 } ^{ \infty } \left( -3x \right) ^{ m }
.\] 
where we can now use our previous identity to find our sum to be equal to
\[
	\left( -3x \right) \cdot \frac{ 1 }{ 1-\left( -3x \right)  } = \frac{ -3x }{ 1+3x }
.\] 
Now that we found our final equation, we still need to rewrite our range. 
\[
-1<-3x<1 \implies \frac{ 1 }{ 3 } >x>-\frac{ 1 }{ 3 } 
.\] 
As a whole, this is just showing that if $ \frac{ 1 }{ 3 } >x>-\frac{ 1 }{ 3 }  $, we can use the equation $ \frac{ -3x }{ 1+3x } $ to find what our series would converge to. 

\section{Short class 01/15}%
\label{sec:Short class 01/15}

\subsection{Geometric and power series review}%
\label{sub:Geometric and power series review}

Geometric happens when we are using 
\[
\sum_{ n=1 } ^{ \infty } ar^{ n-1 }=\frac{ a }{ 1-r } 
.\] 
Power happens when we have 
\[
1+x+x^2+x^3+\ldots=\sum_{ n=0 } ^{ \infty } x^{ n }=\frac{ 1 }{ 1-x } ,\left\| x \right\|<1 \text{ or }-1<x<1
.\] 

\paragraph{Ex}%
\label{par:Ex}
\[
\text{ For }\sum_{ n=0 } ^{ \infty } \left( -2x \right) ^{ n }\text{ Find the result, and the x-values it converges for }
.\] 
Just plug in values into our formula to get
\[
\frac{ 1 }{ 1-\left( -2x \right) ^{ n } }, \left\| -2x \right\|<1 \text{ or }\frac{ 1 }{ 2 } >x>-\frac{ 1 }{ 2 } \text{ which is our interval of convergence}
.\] 
This is basically the same thing as the geometric sequence but, we are solving for our r value instead of knowing it off the bat.

\paragraph{Ex.}
\[
\text{ For }\sum_{ n=0 } ^{ \infty } \left( \frac{ 2 }{ x }  \right) ^{ n }
.\]
With our identity we get
\[
\frac{ 1 }{ 1-\frac{ 2 }{ x }  } =\frac{ x }{ x-2 } 
.\] 
For this we get a technical detail that $ x\neq 2 $. Now to find our interval of convergence we can find that we have $ \left\| \frac{ 2 }{ x }  \right\|<1 $ which gives us $ \left| \frac{ 1 }{ x }  \right|<\frac{ 1 }{ 2 }   $ which is basically saying that we can have an x value anywhere that isn't between -2 and 2 inclusive. 

\section{Telescoping series}%
\label{sec:Telescoping series}
\paragraph{Ex.}%
\label{par:Ex.}
\[
\sum_{ n=1 } ^{ \infty } \frac{ 1 }{ n\left( n+1 \right)  } 
.\] 
If we were to just start getting some terms we would have
\[
\frac{ 1 }{ 2 } +\frac{ 1 }{ 6 } +\frac{ 1 }{ 12 } +\ldots
.\] 
But what happens if we instead do a partial decomposition of our function?
\begin{gather*}
\frac{ 1 }{ n\left( n+1 \right)  } =\frac{ A }{ n } +\frac{ B }{ n+1 } \\
1=A\left( n+1 \right) +B\left( n \right) \implies B = -1 \text{ and }A=1\\
\end{gather*}

Going back to the sum we can now rewrite it as
\begin{gather*}
\sum_{ n=1 } ^{ \infty } \left( \frac{ 1 }{ n } -\frac{ 1 }{ n+1 }  \right) \\
=\left( \frac{ 1 }{ 1 } -\frac{ 1 }{ 2 }  \right) +\left( \frac{ 1 }{ 2 } -\frac{ 1 }{ 3 }  \right) +\left( \frac{ 1 }{ 3 } -\frac{ 1 }{ 4 }  \right) +\ldots+\left( \frac{ 1 }{ n-1 } -\frac{ 1 }{ n }  \right)  +\left( \frac{ 1 }{ n } -\frac{ 1 }{ n+1 }  \right) 
\end{gather*}
Looking at our terms, we can see that all of them cancel other than 
\[
1-\frac{ 1 }{ n+1 } 
.\] 
Which holds for all telescope series. Now we can easily take the limit to find 
\[
\lim_{ n \to \infty} S_n = \lim_{ n \to \infty} \left( 1-\frac{ 1 }{ n+1 }  \right) =1
.\] 
For all telescopic series' we can easily find the answer if we break the function into partials. While it's possible to do with the original function it's a lot harder and should usually be done with partials. 

Harmonic series will be done tomorrow. 
\section{01/16}%
\label{sec:01/16}
\subsection{Examples from the worksheet}%
\label{sub:Examples from the worksheet}

\subsubsection*{Ex 9b}
\[
\sum_{ n=0 } ^{ \infty } \frac{ 3\left( -2 \right) ^{ n }-5^{ n } }{ 8^{ n } }
.\] 
Because we see that our sum starts at 0 we should probably break it into $ \sum_{ m=0 } ^{ \infty } ar^{ m } $.

\[
\sum_{ n=0 } ^{ \infty } \left( \frac{ 3\left( -2 \right) ^{ n } }{ 8^{ n } }-\frac{ 5^{ n } }{ 8^{ n } } \right) 
.\] 
\[
=\sum_{ i=0 } ^{ \infty } 3\left( -\frac{ 1 }{ 4 }  \right) ^{ n }-\sum_{ i=0 } ^{ \infty } \left( \frac{ 5 }{ 8 }  \right) ^{ n } 
.\] 
Now that we have our 2 geometrics we can just find it to be
\[
\frac{ 3 }{ 1+\frac{ 1 }{ 4 }  } -\frac{ 1 }{ 1+\frac{ 5 }{ 8 }  } 
.\] 

\section{Harmonic series 11.2}%
\label{sec:Harmonic series 11.2}

These can be found as something like 
\[
1+\frac{ 1 }{ 2 } +\frac{ 1 }{ 3 } +\frac{ 1 }{ 4 } +\frac{ 1 }{ 5 } 
.\] 
These are generally infinite but we should still prove it. So,
\[
\text{ Prove } \sum_{ n=1 } ^{ \infty } \frac{ 1 }{ n } \text{ diverges }
.\] 
We can start by looking at our partial sums so lets start with
\begin{align*}
	S_2 &= 1+ \frac{ 1 }{ 2 } \\
	s_4 &= 1+ \frac{ 1 }{ 2 } +\frac{ 1 }{ 3 } +\frac{ 1 }{ 4 } \\
	S_8 &= 1+\frac{ 1 }{ 2 } +\frac{ 1 }{ 3 } +\frac{ 1 }{ 4 } +\frac{ 1 }{ 5 } +\frac{ 1 }{ 6 } +\frac{ 1 }{ 7 } +\frac{ 1 }{ 8 } 
\end{align*}
So we can rewrite some of our terms to find what value they are larger than. 
\begin{align*}
S_2 &= \frac{ 1 }{ 4 } +\frac{ 1 }{ 4 }  \\
S_4 &= 1+\frac{ 2 }{ 2 }  \\
S_8 &= 1+\frac{ 3 }{ 2 }  \\
.\end{align*}
Because we have a pattern of $ S_{ 2^{ n } }>1+\frac{ n }{ 2 }  $ we can start to find our limit. So,
\[
\lim_{ n \to \infty} S_{ 2^{ n } }=\sum_{ i=1 } ^{ n } \frac{ 1 }{ n } 
.\] 
Becuase our $ \lim_{ n \to \infty} 1+\frac{ n }{ 2 } = \infty $ we say that our sum is infinite because we found a smaller function that diverges which proves that everything above it will also diverge. 

\subsection{Theorem}%
\label{sub:Theorem}
If $ \sum_{  } ^{  } a_n $ converges, then $ a_n \to 0 $. However, this can be misunderstood with the converse which is if $ a_n \to 0$, then the $ \sum_{  } ^{  } a_n $ converges. This is wrong because this wouldn't be true for the general case. One such case is the harmonic series $ \sum_{ n=1 } ^{ \infty } \frac{ 1 }{ n }  $. Our $ a_n $ goes to 0, but the sum of this will go to infinity and diverge.

Instead we can write the contrapositive of the theorem as, if $ \lim_{ n \to \infty} a_n \neq  0 $ then $ \sum_{  } ^{  } a_n $ diverges. 

\paragraph{Ex.}%
\label{par:Ex.}
\[
\text{ Prove }\sum_{  } ^{  } 2^{ n }\text{ diverges }
.\] 
For this it's simply enough to show that $ \lim_{ n \to \infty} 2^{ n }=\infty $ which counters our orignal theorem. The part where people go wrong is thinking that if we prove $ \lim_{ n \to \infty} a_n = 0 $ then our sum must converge. However, this is not enough and must have other theorems along with it to properly prove what we are trying to show. 
\paragraph{Ex.}
\[
\text{ Determine divergence or convergence on }\frac{ 1 }{ 3 } +\frac{ 1 }{ 6 } +\frac{ 1 }{ 9 } +\frac{ 1 }{ 12 } +\frac{ 1 }{ 15 } +\ldots
.\]
This can be done by writing our sum as 
\[
\sum_{ n=1 } ^{ \infty } \frac{ 1 }{ 3n } = \frac{ 1 }{ 3 } \sum_{ n=1 } ^{ \infty } \frac{ 1 }{ n } 
.\] 
Which is just our harmonic series so this would diverge.

\subsection{Theorem}%
\label{sub:Theorem}
If $ \sum_{  } ^{  } a_n $ and $ \sum_{  } ^{  } b_n $ each converge then so does $ k\sum_{  } ^{  } a_n $ and $ \sum_{  } ^{  } \left(   a_n \pm b_n \right)$. 
\paragraph{Ex.}
This can be used for the following
\[
\text{ Show }\sum_{ n=1 } ^{ \infty } \left( \frac{ 1 }{ e^{ n } } +\frac{ 1 }{ n\left( n+1 \right)  }  \right) \text{ converges/diverges }
.\] 

So,
\[
\sum_{ n=1 } ^{ \infty } \frac{ 1 }{ e^{ n } } +\sum_{ n=1 } ^{ \infty } \frac{ 1 }{ n\left( n+1 \right)  } 
.\] 
We can look at our first sum and find that it will be a geometric series that can be written as
\[
\sum_{ n=1 } ^{ \infty } \left( \frac{ 1 }{ e }  \right) ^{ n }
.\] 
Now because our base of $ \frac{ 1 }{ e }  $ is less than 1, we find that this is convergent. For our second sum we can find this as a telescoping series and can find that this is convergent. Because both functions are proven convergent we know that the original sum is also convergent based on our new theorem.

\paragraph{Tomorrow will be Improper integrals again from calc 2. Pull out the notes from calc 2}

\section{01/17}%
\label{sec:01/17}

\subsection{Common mistakes on the quiz}%
\label{sub:Common mistakes on the quiz}
\paragraph{Question 1}
Some people are showing that the absolute value of the limit is less than 1, but we need to show that the limit itself is less than 1. One thing that you can do is use subsequences such as doing an even or odd n to prove it.

\paragraph{NOTE}
Theorems will be given on the exams and do not need to be remembered. 

\paragraph{Question 5 part 1}
Show that $ \ln^{  } \left( b_n \right) = \frac{ 1 }{ n } \sum_{ k=1 } ^{ n } \ln^{  } \left( \frac{ k }{ n }  \right) $ \\

Specifically, during the process of simplifying $ \frac{ 1 }{ n } \ln^{  } \left( n! \right) -\ln^{  } \left( n \right) $ gets messed up when trying to combine the terms into a single sum. This needs to be proven using the sum of $ \ln^{  } \left( n \right)  $.


\subsection{P integrals}%
\label{sub:P integrals}
\[
\int_{ a }^{ \infty } \frac{ dx }{ x^{ p } } \text{ or } \int_{ 0 }^{ 1 } \frac{ dx }{ x^{ p } }
.\] 
where
\begin{align*}
\int_{ a }^{ \infty } \frac{ dx }{ x^{ p } } \text{ converges if p > 1 to } \frac{ a^{ 1-p } }{ p_1 }\\
\int_{ 0 }^{ 1 } \frac{ dx }{ x^{ p } } \text{ converges if p < 1 to } \frac{ 1 }{ 1-p }
.\end{align*}

\paragraph{Ex.}
\[
\int_{ 9 }^{ \infty } \frac{ dx }{ x^{ 10 } } =\frac{ n_1-10 }{ 10-1 }=\frac{ n^{ -9 } }{ 9 } = \frac{ 1 }{ 9n^{ 9 } } 
.\] 

\paragraph{Ex.}
\[
\int_{ n+1 }^{ \infty } \frac{ dx }{ x^{ 10 } } = \frac{ \left( n+1 \right) ^{ 1-10 } }{ 10-1 }=\frac{ \left( n+1 \right) ^{ -9 } }{ 9 }=\frac{ 1 }{ 9\left( n+1 \right) ^{ 9 } } 
.\] 
This is the only type that will be used in this class and we will never(?) see the 0 to 1

\section{11.3 The integral Test}%
\label{sec:11.3 The integral Test}

Prof gave a paper worksheet that showed
\[
\sum_{ k=2 } ^{ n } a_k<\int_{ 1 }^{ n } f\left( x \right) dx<\sum_{ k=1 } ^{ n-1 } a_k
.\] 
When we assume $ f\left( x \right)  $ is monatonic (how do you spell it) decreasing over $ \left( 0,\infty \right)  $ or $ f''\left( x \right) >0 $ we look at this because it will approach 0 and we may have a convergent sum, or $ \lim_{ x \to \infty} f\left( x \right) =0 $. This is shown in some sketches that the integral will be the absolute area and our sum on the left will underestimate while the right one will overestimate. \\

With this we show that (1) if our integral is finite then the sum on the left will also be finite. This is just basic comparison, because if we have a bigger function, then anything under it should be finite. This also goes if we show the right sum to be finite, then our integral and other sum will also be finite. \\

We also have (2) where the sum $ \sum_{ k=1 } ^{ \infty } a_k $ converges to a value S, then given $ \int_{ 1 }^{ n } f\left( x \right) dx <\sum_{ k=1 } ^{ n-1 } a_k<\sum_{ k=1 } ^{ \infty } a_k=S<\infty$. This is when the sequence $ \left\{ \int_{ 1 }^{ n } f\left( x \right) dx \right\}  $ is monotonic increasing (b/c f(x) > 0) and bounded above by S. Thus $ \int_{ 1 }^{ \infty } f\left( x \right) dx\le S $.

\paragraph{Ex. The harmonic series}
\[
\sum_{ n=1 } ^{ \infty } \frac{ 1 }{ n } =\infty
.\] 
This can be compared to our improper integral
\[
\int_{ 1 }^{ \infty } \frac{ 1 }{ x } dx= \text{ finite because p=1 }
.\] 

\paragraph{Ex. Determine conv. or div.}
\[
\sum_{ k=1 } ^{ \infty } \frac{ k }{ k^2+1 } 
.\] 
Remembering our theorem of that if $ \sum_{  } ^{  } a_k $ converges then $ a_k\to 0 $, we can use the converse and see what $ \int_{ 1 }^{ \infty } \frac{ x }{ x^2+1 } dx $ is. Before using this we should make sure that we match our requirements, (1) we need our function to be monotonic decreasing and (2) that $ \lim_{ x \to \infty} f\left( x \right) =0 $. \\

Our second req. is clearly passed using dominance theory and isn't a problem so now we just check if our function is monotonic decreasing. The easiest way would just be to derive our top,
\[
\left( x^2+1 \right) \left( 1 \right) -\left( x \right) \left( 2x \right) \to -x^2+1<0
.\] 

To find that we have a decreasing function as it's less than 0 and, now we can just use u sub on the integral to find what our p value is,
\begin{align*}
u&= x^2+1 \\
du&= 2xdx \\
\to \frac{ du }{ 2 } &= xdx \\
\int_{ 2 }^{ \infty } \frac{ du }{ 2u } &= \infty
.\end{align*}

\paragraph{Ex.}
\[
\sum_{ k=3 } ^{ \infty } \frac{ 1 }{ \sqrt{ 2k-5} } 
.\] 
Rewrite using integral,
\[
\int_{ 3 }^{ \infty } \frac{ 1 }{ \sqrt{ 2x-5} } dx
.\] 
and we can see that this clearly goes to 0 with dominance so our second condition is met aand we check our first,
\[
f'=d\left( 2x-5 \right) ^{ -\frac{ 1 }{ 2 }  }=-\frac{ 1 }{ 2 } \left( 2x-5 \right) ^{ -\frac{ 3 }{ 2 }  }\left( 2 \right) <0
.\] 
Which we find to be true and that our function is decreasing. This lets us directly take the integral using u sub now,
\begin{gather*}
u=2x-5 \\
du = 2dx
\end{gather*}
\[
\int_{ 3 }^{ \infty } \frac{ dx }{ \sqrt{ 2x-5} }=\int_{ 1 }^{ \infty } \frac{ du }{ 2\sqrt{ u} } =\infty
.\] 
This also goes to infinity b/c p=1/2.

\paragraph{Ex.}
\[
\sum_{ k=0 } ^{ \infty } \frac{ 1 }{ k^2+4 } 
.\] 
Rewrite using integral,
\[
\int_{ - }^{ \infty } \frac{ dx }{ x^2+4 } 
.\] 
Again, it's obvious this goes to 0, so we take the d/dx
\[
f'=-\frac{ 2x }{ \left( x^2+4 \right) ^2 } <0
.\] 
Which matches our other condition and we can integrate.

\[
\int_{ 0 }^{ \infty } \frac{ dx }{ x^2+4 } 
.\] 
This requires us to use trig sub so let
\begin{gather*}
x=2\tan^{  } \left( \theta \right) \\
dx=2\sec^{ 2 } \left( \theta \right) d\theta
\end{gather*}
so,
\[
\int_{  }^{ } \frac{ 2\sec^{ 2 } \left( \theta \right) d\theta }{ 4\sec^{ 2 } \left( \theta \right)  }
.\] 

Where we can use $ \theta = \arctan^{  }\left( \frac{ x }{ 2 }  \right) $ to change our limits to $ \int_{ 0 }^{ \frac{ \pi }{ 2 }  }  $, so
\[
\frac{ 1 }{ 2 } \theta \bigg|_{ 0 }^{ \frac{ \pi }{ 2 }  } =\frac{ \pi }{ 4 }
.\] 
Now because this integral is a finite value we find that our sum will converge. However this won't be the limit of the sum and this will be talked about on monday i think.
