\documentclass{report}

\input{../../../pkgs/preamble}
\input{../../../pkgs/macros}
\input{../../../pkgs/letterfonts}

\title{\Huge{Calc3q6}\\  }
\author{\huge{yawnbo}}
\date{\today}

\begin{document}

\maketitle
\newpage% or \cleardoublepage
% \pdfbookmark[<level>]{<title>}{<dest>}
\pdfbookmark[section]{\contentsname}{toc}

\pagebreak
\section{Question 2}%
\label{sec:Question 2}
Finding the zeros of our r,
\[
1-2\sin^{  } \left( \theta \right) \implies \sin^{  } \left( \theta \right) =\frac{ 1 }{ 2 } 
.\] 
\[
\theta = \frac{ \pi }{ 6 } ,\frac{ 5\pi }{ 6 }
.\] 
This means the area of our inner region will take the limits of $ \beta=\frac{ 5\pi }{ 6 }  $ and $ \alpha = \frac{ \pi }{ 6 }  $. Knowing this we can assume our larger area will be from $ \frac{ \pi }{ 6 }  $ to whenever the next time the small circle will start at which will happen at $ \frac{ \pi }{ 6 } +2\pi = \frac{ 13\pi }{ 6 }  $. This means we can integrate our area as 
\[
\frac{ 1 }{ 2 } \left( \int_{ \frac{ 5\pi }{ 6 }  }^{ \frac{ 13\pi }{ 6 }  } \left( 1-2\sin^{  } \left( \theta \right)  \right) ^2d\theta - \int_{ \frac{ \pi }{ 6 }  }^{ \frac{ 5\pi }{ 6 }  } \left( 1-2\sin^{  } \left( \theta \right)  \right) ^2d\theta \right)
.\] 
Solving the generic integral,
\[
\int_{  }^{  } 1 - 4\sin^{  } \left( \theta \right) +4\sin^{ 2 } \left( \theta \right) d\theta = \theta + 4\cos^{  } \left( \theta \right) + \int_{  }^{  } 4\left( \sin^{ 2 } \left( \theta \right)  \right) d\theta
.\] 
\[
4 \int_{  }^{  } \left( \frac{ 1-\cos^{  } \left( 2\theta \right)  }{ 2 } \right) d\theta = 2\theta - \frac{ \sin^{  } \left( 2\theta \right) }{ 2 } 
.\] 
So each integral will be,
\[
\left[ 3\theta + 4\cos^{  } \left( \theta \right) - \frac{ \sin^{  } \left( 2\theta \right)  }{ 2 }\right]_{ \alpha }^{ \beta }
.\] 
and plugging in our values,
\[
\frac{ 1 }{ 2 } \left( \left[ 3\theta + 4\cos^{  } \left( \theta \right) -\frac{ \sin^{  } \left( 2\theta \right)  }{ 2 } \right]_{ \frac{ 5\pi }{ 6 } }^{ \frac{ 13\pi }{ 6 } } - \left[ 3\theta + 4\cos^{  } \left( \theta \right) -\frac{ \sin^{  } \left( 2\theta \right) }{ 2 }  \right]_{ \frac{ \pi }{ 6 }  }^{ \frac{ 5\pi }{ 6 } }\right)
.\] 

Evaluating the first block,
\[
	\left( \frac{ 39\pi }{ 6 } + 4\cos^{  } \left( \frac{ 13\pi }{ 6 }  \right) - \frac{ \sin^{  } \left( \frac{ 26\pi }{ 6 }  \right) }{ 2 }  \right) - \left( \frac{ 15\pi }{ 6 } + 4\cos^{  } \left( \frac{ 5\pi }{ 6 }  \right) - \frac{ \sin^{  } \left( \frac{ 10\pi }{ 6 }  \right) }{ 2 }  \right) = 4\pi + 4\sqrt{ 3 } 
.\] 
and the second,
\[
-\left(  \left( \frac{ 15\pi }{ 6 } + 4\cos^{  } \left( \frac{ 5\pi }{ 6 }  \right) - \frac{ \sin^{  } \left( \frac{ 10\pi }{ 6 }  \right) }{ 2 }  \right) - \left( \frac{ \pi }{ 2 } +4\cos^{  } \left( \frac{ \pi }{ 6 }  \right) - \frac{ \sin^{  } \left( \frac{ \pi }{ 3 } \right) }{ 2 }  \right) \right) = -\left( 2\pi - 2\sqrt{ 3 }  \right) 
.\] 
Combining these we get our final area of $ A = \pi + 3\sqrt{ 3 } $ 

\section{Question 3}%
\label{sec:Question 3}
\[
r = \frac{ 2 }{ \cos^{  } \left( \theta \right) -\sin^{  } \left( \theta \right)  } \to r\left( \cos^{  } \left( \theta \right) -\sin^{  } \left( \theta \right)  \right) = 2 \to x - y = 2 \to y = x-2
.\] 
\newpage
\section{Question 4}%
\label{sec:Question 4}
First finding the intersection point we know that $ \sin^{  } \left( \theta \right) =\cos^{  } \left( \theta \right)  $ at $ \theta= \frac{ \pi }{ 4 }  $ and we can find our tangents using the formula 
\[
\frac{ dy }{ dx } = \frac{ \frac{ dr }{ d\theta }\sin^{  } \left( \theta \right) +r\cos^{  } \left( \theta \right)  }{ \frac{ dr }{ d\theta }\cos^{  } \left( \theta \right) -r\sin^{  } \left( \theta \right)  }
.\] 
So,
\[
\frac{ dy }{ dx } a\sin^{  } \left( \theta \right)  = \frac{ a\cos^{  } \left( \theta \right) \sin^{  } \left( \theta \right) +a\sin^{  } \left( \theta \right) \cos^{  } \left( \theta \right)  }{ a\cos^{  } \left( \theta \right) \cos^{  } \left( \theta \right) -a\sin^{   } \left( \theta \right)  \sin^{  } \left( \theta \right)} = \frac{ 2\sin^{  } \left( \theta \right) \cos^{  } \left( \theta \right)  }{ \cos^{  } \left( 2\theta \right)  }
.\] 
\[
\frac{ dy }{ dx } a\cos^{  } \left( \theta \right) = \frac{ -a\sin^{  } \left( \theta \right) \sin^{  } \left( \theta \right) +a\cos^{  } \left( \theta \right) \cos^{  } \left( \theta \right)  }{ -a\sin^{  } \left( \theta \right) \cos^{  } \left( \theta \right) -a\cos^{  } \left( \theta \right) \sin^{  } \left( \theta \right)  }= \frac{ \cos^{  } \left( 2\theta \right)  }{ -2\sin^{  } \left( \theta \right) \cos^{  } \left( \theta \right)  }
.\] 
Now evaluating at $ \frac{ \pi }{ 4 }  $, 
\[
r=a\sin^{  } \left( \frac{ \pi }{ 4 } \right) \implies \frac{ 2\sin^{  } \left( \frac{ \pi }{ 4 }  \right) \cos^{  } \left( \frac{ \pi }{ 4 }  \right)  }{ \cos^{  } \left( \frac{ \pi }{ 2 }  \right)  }= \frac{ 1 }{ 0 } = \text{ undefined } = \text{ vertical tangent}
.\] 
\[
r=a\cos^{  } \left( \frac{ \pi }{ 4 } \right) = \frac{ \cos^{  } \left( \frac{ \pi }{ 2 }  \right)  }{ -\sin^{  } \left( \frac{ \pi }{ 4 }  \right) \cos^{  } \left( \frac{ \pi }{ 4 }  \right)  }= 0 = \text{ horizontal tanget}
.\] 
Since these tangents are perpendicular we have proved that the intersection is at a right angle.
\section{Question 5}%
\label{sec:Question 5}
Solving for the zeroes we find that $ \theta=\frac{ \pi }{ 4 } + \frac{ \pi }{ 2 } k $. Since we want the loop of one curve we can use \[
2 \int_{ 0 }^{ \frac{ \pi }{ 4 }  } \sqrt{ \cos^{ 2 } \left( 2\theta \right) + \left( 4\sin^{ 2 } \left( 2\theta \right)  \right)  } d\theta
.\] 
because we have a max at $ 0 $ and another 0 at $ -\frac{ \pi }{ 4 }  $ we know of symmetry and can just double the integral. Our inside can be rewritten as 
\[
\cos^{ 2 } \left( 2\theta \right) +4\left( 1-\cos^{ 2 } \left( \theta \right)  \right)  = 4-3\cos^{ 2 } \left( 2\theta \right) 
.\] 
and our integral as 
\[
2 \int_{ 0 }^{ \frac{ \pi }{ 4 }  } \sqrt{  4-3\cos^{ 2 } \left( 2\theta \right)  } 
.\] 
which doesnt look any easier to solve so I'm going to estimate it to $ L \approx 2.422 $. 

\end{document}
