\documentclass[a4paper]{article}

\usepackage[utf8]{inputenc}
\usepackage[T1]{fontenc}
\usepackage{textcomp}
\usepackage[english]{babel}
\usepackage{amsmath, amssymb}
\title{More impropers}
\author{yawnbo}
\date{\today}

\maketitle

\pdfsuppresswarningpagegroup=1

\begin{document}
\section{7.8 Improper Integrals p.2}%
\label{sec:7.8 Improper Integrals p.2}
\subsection{Ex.}%
\label{sub:Ex.}
\[
\int_{2}^{\infty} xe^{-3x}dx
.\] 
\paragraph{Apply integration by parts}
\paragraph{Let $u=x$ and $du=dx$}
\paragraph{Then let $dv=e^{-3x}dx$ and $v=-\frac{1}{3}e^{-3x}$}
\paragraph{Go back to lims from type 1 yesterday}

\[
\lim_{R \to \infty} -\frac{1}{3}e^{-3x}x |_{2}^{R}+\int_{2}^{R} \frac{1}{3}e^{-3x}dx
.\] 
\paragraph{Which we simplify to }
\[
\lim_{R \to \infty} -\frac{1}{3}xe^{-3x}-\frac{1}{9}e^{-3x}|_{2}^{R}
.\] 
\paragraph{Now just integrate}
\[
  \lim_{R \to \infty} \left[-\frac{x}{3e^{3R}}-\frac{1}{9}\cdot \frac{1}{e^{3R}}-\left(-\frac{1}{3}(2)e^{-3(2)} - \frac{1}{9}e^{-3(2)}\right)\right]
.\] 

\paragraph{Evaulating the first half, we find that $-\frac{R}{3e^{3R}}$ is interminate so we take the derivative of the top and bottom with respect to R. This leaves us with}
\[
\lim_{R \to \infty} -\frac{1}{9e^{3R}}=0
.\] 
\paragraph{Now we look at the right side to just simplify constants. }
\[
-\frac{1}{3}(2)e^{-3(2)} - \frac{1}{9}e^{-3(2)} = \frac{7}{9e^{6}}
.\] 
\newpage

\subsection{Another ex.}%
\label{sub:Another ex.}
\paragraph{Double improper integral.}
\[
\int_{-\infty}^{\infty} 5xe^{-x^2}
.\] 
\paragraph{This is a symmetric Interval. Something like, $\left[-a,a\right]$ and we can simplify the limits in the function is even or odd. So try,}
\[
5(-x)e^{-x^2}
.\] 
\paragraph{Which shows us that this function is odd because $f(-x) =-f(x)$}
\paragraph{But when we integrate an odd function over a symmetric interval it is just 0 because both sides will cancel each other out. }
\subsection{Even functions ex.}%
\label{sub:Even functions ex.}
\[
\int_{-\infty}^{\infty} 5x^2e^{-x^2}dx
.\] 
\paragraph{Test if its an even function}%
\label{par:Test if its an even function}
\paragraph{Which it is but im not writing that}

\paragraph{Because this is even we can rewrite as}
\[
  2 \int_{0}^{\infty} 5x^2e^{-x^2}
.\]
\paragraph{For this we cant integrate because $e^{-x^2}$ is not possible to evaluate. }

\section{Convergent Vs. Divergent.}%
\label{sec:Convergent Vs. Divergent.}
\paragraph{This is just saying that divergent can be infinite or doesn't go to a single value. }
\paragraph{This can happen for something like }

\[
\int_{0}^{\infty} \sin^{2}(x)dx
.\] 
\paragraph{Forgot to write down steps just know this goes to }
\[
\int_{0}^{\infty} \left( \frac{1}{2}-\frac{1}{2}\cos^{}(2x) \right) dx
.\] 
\[
\lim_{R \to \infty} \frac{1}{2}R-\frac{1}{4}\sin^{}(2R)
.\] 
\newpage
\paragraph{This creates a problem because the graph of sin fluctuates between -1 and 1. So the limit is not going to a single value. }
\paragraph{This creates a weird situation because the first eq because infinite but the second part DNE. This just lets it be infnite because one of them goes to infinity and calcels the DNE. }%
\label{par:This creates a weird situation because the first eq because infinite but the second part DNE. This just lets it be infnite because one of them goes to infinity and calcels the DNE. }

\section{Convergence and divergence without doing integrals (Comparison Theorum)}%
\label{sec:Convergence and divergence without doing integrals (Comparison Theorum)}

\subsection{Ex.}%
\label{sub:Ex.}
\[
\int_{1}^{\infty} \frac{e^{-x}}{x}dx
.\] 
\paragraph{This integral is one that cannot be done using any technique. This is called a non-elementary function/integral.}
\paragraph{Even though we can't solve it we can still figure out if its convergent or divergent. }
\paragraph{For this one he drew and graph and told us to find a function that sits above the original. }

\paragraph{Start with finding your interval as $[1,\infty)$ where $x\ge 1$, then rewrite the condition as $\frac{1}{x}\le 1$}
\paragraph{Now we can multiply both sides by $e^{-x}$ to find that $\frac{e^{-x}}{x}\le e^{-x}$}
\paragraph{This tells us that }
\[
\int_{1}^{\infty} \frac{e^{-x}}{x}dx \le \int_{1}^{\infty} e^{-x}dx
.\] 
\paragraph{Which tells us that}
\[
=\lim_{R \to \infty} -e^{-x}|_{1}^{R}=\lim_{R \to \infty} 0+e^{-1}=\frac{1}{e}
.\] 

\paragraph{This just proved to us that the answer for the original integral has to be less than or equal to $\frac{1}{e}$}
\paragraph{This is just everything that comparison does and for this example it proves the }
\[
\int_{1}^{\infty} \frac{e^{-x}}{x}dx
.\] 
\paragraph{Converges due to our comparison function. }

\paragraph{One more lecture on monday}

\end{document}
