\documentclass{report}

\input{../../pkgs/preamble}
\input{../../pkgs/macros}
\input{../../pkgs/letterfonts}

\title{\Huge{Lin alg}\\ Notes }
\author{\huge{yawnbo}}
\date{\today}

\begin{document}

\maketitle
\section{mat help}%
\label{sec: mat help }

\paragraph{Let $ A = \begin{bmatrix} 2 & -4 \\ -2 & 4 \end{bmatrix}  $. Construct a $ 2 \times 2 $ matrix B such that AB is the zero matrix. Use two different nonzero columns for b. \\ \\}
We essentially want to set each entry in the resultant vector to zero, so we simplify into our matrix with variables and set each one equal to zero. 
\[
	\begin{bmatrix} 2 & -4 \\ -2 & 4 \end{bmatrix} \begin{bmatrix} a & b \\ c & d \end{bmatrix} = \begin{bmatrix} 0 & 0 \\ 0 & 0 \end{bmatrix} 
.\] 
Simplifying,
\[
	\begin{bmatrix} 2a - 4c & 2b -4d \\ -2a + 4c & -2b +4d \end{bmatrix} = \begin{bmatrix} 0 & 0 \\ 0 & 0 \end{bmatrix}
.\] 
Writing this out will give us something we are more used to,
\begin{align*}
2a + 0b - 4c + 0d &= 0 \\
0a + 2b + 0c - 4d &= 0 \\
-2a + 0b + 4c + 0d &= 0 \\
0a + -2b + 0c + 4d &= 0 \\
.\end{align*}
Where now you can either just use algebra or find the rref, im showing the rref because I don't wanna do algebra,
\[
	\begin{bmatrix} 2 & 0 & -4 & 0 & \big| & 0 \\ 0 & 2 & 0 & -4 & \big| & 0 \\ -2 & 0 & 4 & 0 & \big| & 0 \\ 0 & -2 & 0 & 4 & \big| & 0 \\ \end{bmatrix} \approx \begin{bmatrix} 1 & 0 & -2 & 0 & 0 \\ 0 & 1 & 0 & -2 & 0 \\ 0 & 0 & 0 & 0 & 0 \\ 0 & 0 & 0 & 0 & 0 \end{bmatrix} 
.\] 
Here we now know that c and d (or $ x_1 $ and $ x_2 $) are free and $ a= 2c $ and $ b=2d $ so we write our mat as
\[
	\begin{bmatrix} 2c & 2d \\ c & d \end{bmatrix} 
.\] 
Where we can now choose some values like 1 and 2 to check if it's true but I don't wanna write it out so take my word for it and submit any solution matrix that fits the above. I submitted using $ c=1 $ and $ d=2 $ but up to you, good luck!
\newpage
\section{Rotation matrix stuff 06/07}%
\label{sec: Rotation matrix stuff 06/07 }
The rotation matrix is given as,
\[
	\begin{bmatrix} \cos^{  } \left( \theta \right) & -\sin^{  } \left( \theta \right) \\ \sin^{  } \left( \theta \right) & \cos^{  } \left( \theta \right)  \end{bmatrix} 
.\] 
Which can just be used to rotate a vector $ \begin{bmatrix} x \\ y \end{bmatrix} $ by an angle $ \theta $ in the counterclockwise (the normal way to spin on the unit circle) direction. I think it's easier to think of it when you multiply them out,
\[
	\begin{bmatrix} x \\ y \end{bmatrix} \cdot \begin{bmatrix} \cos^{  } \left( \theta \right) & -\sin^{  } \left( \theta \right) \\ \sin^{  } \left( \theta \right) & \cos^{  } \left( \theta \right)  \end{bmatrix} = \begin{bmatrix} x \cos^{  } \left( \theta \right) - y \sin^{  } \left( \theta \right) \\ x \sin^{  } \left( \theta \right) + y \cos^{  } \left( \theta \right)  \end{bmatrix} = \begin{bmatrix} x' \\ y' \end{bmatrix}
.\] 
The reason that's done is because we cannot rotate a vector by an angle without moving to the polar plane. In the polar plane how would we draw a vector along the x axis and rotate it by an angle $ \theta $? \\\\ 
Think of plotting the unit circle onto a normal plane and imagine you're at $ \theta=0 $. You know that the point would be at $ \left( 1,0 \right)  $ which is all the way on the right of the circle. Now, how is that represented in polar? ie. what trig functions get us those points at 0. Well, we know that the polar plane can be defined in normal coordinates as $ \left( r \cos^{  } \left( \theta \right) , r \sin^{  } \left( \theta \right)  \right)  $, and since we are looking for $ \left( 1,0 \right)  $ we just set them equal,
\[
\cos^{ -1 } \left( \theta \right) =1 \implies \theta = 0 
.\] 
\[
\sin^{ -1 } \left( \theta \right) = 0 \implies \theta = 0
.\] 
and this tells us that we can write the normal coordinate point $ \left( 1,0 \right)  $ as $ \left( \cos^{  } \left( \theta \right) ,\sin^{  } \left( \theta \right)  \right)  $. But notice that our matrix is literally just written as $ \left( x,y \right)  $ on it's side, so we can just write the polar plane in the same way,
\[
\begin{bmatrix} x \\ y \end{bmatrix} = \begin{bmatrix} \cos^{  } \left( \theta \right) \\ \sin^{  } \left( \theta \right)  \end{bmatrix} 
.\] 
Which gives us the first column of the rotation matrix, but the second vector is where it's kind of weird. I would imagine the y axis and how it differs from the x on the cartesian plane. The y axis is just the x axis rotated by $ \frac{ \pi }{ 2 } $ radians, or 90 degrees which would throw off our original polar vector because we no longer start at $ \left( 1,0 \right)  $ and instead start at $ \left( 0,1 \right)  $, but this can be fixed by rotating the same equation we have for the x, by 90 degrees or as $ \theta + 90 $, so let's do exactly that,
\[
y=\begin{bmatrix} 0 \\ 1 \end{bmatrix} = \begin{bmatrix} \cos^{  } \left( \theta + 90 \right) \\ \sin^{  } \left( \theta + 90 \right)  \end{bmatrix}
.\] 
Which is exactly what the y vector is! The only problem with this is that it looks horribly ugly, so we use trig identities to clean it up. 
\[
\cos^{  } \left( \theta + 90 \right) = \cos^{  } \left( \theta \right) \underbrace{ \cos^{  } \left( 90 \right) }_{ 0 }  - \sin^{  } \left( \theta \right) \underbrace{ \sin^{  } \left( 90 \right) }_{ 1 }  = 0 - \sin^{  } \left( \theta \right) = -\sin^{  } \left( \theta \right)
.\] 
\[
\sin^{  } \left( \theta + 90 \right) = \sin^{  } \left( \theta \right) \underbrace{ \cos^{  } \left( 90 \right) }_{ 0 }  + \cos^{  } \left( \theta \right) \underbrace{ \sin^{  } \left( 90 \right) }_{ 1 }  = 0 + \cos^{  } \left( \theta \right) = \cos^{  } \left( \theta \right)
.\] 
and we can write the second column as 
\[
\begin{bmatrix} -\sin^{  } \left( \theta \right) \\ \cos^{  } \left( \theta \right)  \end{bmatrix} 
.\]
Now since our vector shouldn't be changed when we rotate by $ \theta = 0$, we line up our columns to the identity matrix and are left with the rotation matrix,
\[
	\begin{bmatrix} 1 & 0 \\ 0 & 1 \end{bmatrix} = \begin{bmatrix} x \\ y \end{bmatrix} = \begin{bmatrix} \cos^{  } \left( \theta \right) & -\sin^{  } \left( \theta \right) \\ \sin^{  } \left( \theta \right) & \cos^{  } \left( \theta \right)  \end{bmatrix}
.\] 
\newpage
\section{Example q}%
\label{sec: Example q }
\textbf{Example 5.} Suppose A, B, and X are $ n\times n $ matricies with A, X, and $ A-AX $ invertible, and suppose 
\[
	\left( A-AX \right) ^{ -1 } = X^{ -1 }B
.\] 
Solve the equation for X. If you need to invert a matrix, explain why that matrix is invertible. \\\\
Start by just solving this for x. Since we have an x on both sides we don't want to split the B and $ X^{ -1 } $ yet because that'll just be reversed later in the process so we instead take the inverse of both sides,
\[
	\left( \left( A-AX \right) ^{ -1 } \right) ^{ -1 } = \left( X^{ -1 }B \right) ^{ -1 }
.\] 
The inverse of an inverse is just the orignal matrix so that handles the left side, and the right side is just the inverse of a product which is the product of the inverses in reverse order (basically just switch multiplying order and invert each one respectively),
\[
	A-AX = B^{ -1 }X
.\] 
Note that the X on the right was inverted before but became the origial because we inverted it. Since we originally assumed that A, X, and $ A-AX $ were invertible we find that we don't know if B is invertible, even though our equation wants us to invert it. This is fine for now, but we need to check this later. \\\\
Now we want to add AX to both sides so we have the X's on the right, and we can drag out the X, but make sure to write it on the right side as $ B^{ -1 } $ and A were both being right multiplied,
\[
A=B^{ -1 }X +AX \implies A= \left( B^{ -1 }+ A \right) X
.\] 
Now we want to cancel everything that isn't the X so we need to take the inverse our multiplicant of x, $ \left( B^{ -1 }+A \right) ^{ -1 } $ and we also will need to chck this later because we still aren't sure if we can invert this whole thing. \\\\
Now we left multiply by the inverse, assuming it exists,
\[
X=\left( B^{ -1 }+A \right) ^{ -1 }A
.\] 
And that's all that X is, but we need to prove that B is invertible and that $ \left( B^{ -1 }+A \right)  $ is also invertible. We prove inversion by showing that what we are looking for is a product of two known invertibles, so we can solve our original problem for B instead of X,
\[
	\left( A-AX \right) ^{ -1 }=X^{ -1 }B \implies B=x\left( A-AX \right) ^{ -1 }
.\] 
The question stated that A, X and $ A-AX $ were invertible, so we know that $ \left( A-AX \right) ^{ -1 } $ is invertible, and since we are multiplying two invertible matricies, we know that B is invertible. \\\\
Last step is to prove that $ \left( B^{ -1 }+A \right)  $ is invertible, which is easiest from the last step before we needed to invert it,
\[
A=\left( B^{ -1 }+A \right) X \implies AX^{ -1 }= B^{ -1 }+A
.\] 
And because we know that A and $ X^{ -1 } $ are invertible, we know that the product is invertible, and since the sum of two invertible matricies is also invertible, we know that $ \left( B^{ -1 }+A \right)  $ is invertible. \\\\
\end{document}
