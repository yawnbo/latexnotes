\section*{01/10/25 Proving things (last for this pdf)}%
\paragraph{Pre class notes}%
Quiz is open and is due in a week (1/17/25).

\section{Examples of sequences}%
\label{sec:Examples of sequences}

\paragraph{Ex. show that $ \frac{ n }{ n+1 }  $ is increasing using $ \frac{ d }{ dx }  $ and algebra. \\}

 We start by remembering that a positive $ \frac{ d }{ dx }  $ means that a function is increasing so, we take its derivative

 \[
 \frac{ \left( n+1 \right) \left( 1 \right) -\left( n \right) \left( 1 \right)  }{ \left( n+1 \right) ^2 } = \frac{ 1 }{ \left( n+1 \right) ^2 } >0
 .\] 

If we instead wanted to do this algebraically we could show that $ a_{ n+1 }>a_n $ for all n. So we,
\[
\frac{ \left( n+1 \right)  }{ \left( n+1 \right) +1 }>\frac{ n }{ \left( n+1 \right)  } \implies \left( n+1 \right) ^2 > n\left[ \left( n+2 \right)  \right] \to n^2+2n+1> n^2+2n \text{ or } 1>0
.\] 
Both of these methods get to the same place and neither is prefered over the other, but one might be easier to do than the other in certain situations.

\paragraph{Again using $ \left\{ \frac{ n }{ n+1 }  \right\}  $, show that this sequence is bounded above. \\}

To do this we can make an assumption that we are bounded somewhere like 1 in this case. We can then write out some of the terms of the sequence to see if we are proven correct, like in this case,
\[
\frac{ 1 }{ 2 } <1, \frac{ 2 }{ 3 } <1, \frac{ 3 }{ 4 } <1 \ldots
.\] 

This can also be proven using algebra by stating that $ n<n+1 $ which will always be valid for each term in the sequence proving that we are bounded at 1. 

Going back to one of our theories from a few days ago, we know that a sequence converges if it's bounded above and increasing so we can say that this is convergent, but we still need to find the limit.

First lets do the hard way of finding it by using algebra. 
\[
\lim_{ n \to \infty} \frac{ n }{ n+1 } \cdot \frac{ \frac{ 1 }{ n }  }{ \frac{ 1 }{ n }  }= \lim_{ n \to \infty} \frac{ 1 }{ 1+\frac{ 1 }{ n }  }= \frac{ 1 }{ 1+0 } =1
.\] 

The easy way can be done just by using Dominance Theory. This states that whenever we are dealing with a variable to the power of some constant ($ n^{ p } $ ) we can just ignore the constant and take the limit of the variable. So in this case we can just ignore the 1 and take the limit of $ \frac{ n }{ n } =1$.
\newpage
\paragraph{Question 24 worksheet}
\[
\lim_{ n \to \infty} \frac{ 2n }{ \sqrt[ 4 ]{ 256n^{ 4 }+81n^2+49 }  }
.\] 
With this we can just use dominance to rewrite our whole expression to be a lot simpler. 
\[
\lim_{ n \to \infty} a_n \approx \lim_{ n \to \infty} \frac{ 2n }{ \sqrt[ 4 ]{ 256n^{ 4 } }  } =\lim_{ n \to \infty} \frac{ 2n }{ 4n } =\frac{ 1 }{ 2 } 
.\] 

We take only the term with the highest power p when using dominance so we just worry about the $ n^{ 4 } $ for this case.

\paragraph{Question 25 worksheet}
\[
\lim_{ n \to \infty} \int_{ \frac{ 2 }{ n }  }^{ \frac{ 1 }{ n }  } \frac{ x+1 }{ x^2+1 }dx
.\] 

Lets break the integral first using u sub where $ u=x^2+1 \text{ and }du=xdx $ 
\[
 \frac{ du }{ 2u } +\tan^{-1}\left( x \right) \Big|_{ \frac{ 2 }{ n }  }^{ \frac{ 1 }{ n }  } =\frac{ 1 }{ 2 } \ln\left( \left\| u \right\| \right) +\tan^{-1}\left( x \right) \Big|_{ \frac{ 2 }{ n }  }^{ \frac{ 1 }{ n }  }=
.\] 
Which simplifies to,
\[
\frac{ 1 }{ 2 } \ln^{  } \left( \left\| \left( \frac{ 1 }{ n }  \right) ^2 +1\right\| \right) + tan^{ -1 }\left( \frac{ 1 }{ n }  \right) -\frac{ 1 }{ 2 } \ln^{  } \left( \left\| \left( \frac{ 2 }{ n }  \right) ^2 +1\right\| \right) - tan^{ -1 }\left( \frac{ 2 }{ n }  \right) =0
.\] 
Looking at this we can see that each term just goes to 0 so we know that our whole thing would go to 0. 

\paragraph{Next week will be geometric sequences}

