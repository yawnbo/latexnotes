\section*{01/08/25 Seqs. cont}%
\label{sec:01/08/25 Seqs. cont}

\subsection{Recursive sequence}%
\label{sub:Recursive sequence}
\paragraph{A recursive sequence is one that needs to use its previous terms to define the next term.}

\subsection{Explicit sequence}%
\label{sub:Explict sequence}

\paragraph{An explicit sequence is one that can be defined by a function.}

\paragraph{Ex. Given}

\[
	\left\{\frac{n+2}{n}\right\}_{n=1}^{\infty}
.\] 
\paragraph{Find the 5th term.}
\paragraph{Because this isn't a recursive sequence, we can just plug in a given $n$ value into our sequence to find our term which is $\frac{7}{5}$. }

\paragraph{This sequence can also be written as a set like,}

\[
	\left\{\left( 1,3 \right), \left( 2,2 \right), \left( 3,\frac{5}{3} \right) \ldots \right\} 
.\] 

\paragraph{Looking at this set, we also can find a commmon theory known as Set Theory, which states that we cannot have a set that repeats. This happens because the sets each have different values of $n$ and an output by the function. }

\paragraph{Ex. Write out the first 5 elements/terms for the sequence}

\[
	\left\{\frac{\left( -1 \right) ^{n}\left( n-2 \right) }{2^{n}}\right\}_{n=0}^{\infty}
.\] 

\paragraph{We can now find the terms leading up to 5.}

\[
\frac{\left( -1 \right) ^{0}\left( 0-2 \right) }{2^{0}}=-2, \frac{\left( -1 \right) ^{1}\left( 1-2 \right) }{2^{1}}=\frac{1}{2}, \frac{\left( -1 \right) ^{2}\left( 2-2 \right) }{2^{2}}=0 \ldots
.\] 
\paragraph{We keep doing the above until we get to using 5 as n.}

\paragraph{This function also shows us an important factor for sequences in $\left( -1 \right) ^{n}$ which creates a function that alternates between negative and positive.}

\newpage
\paragraph{Ex. Make a formula for}
\[
	\left\{-2,4,-6,8,-10 \ldots \right\}
.\] 
\paragraph{Lets start with $2n$ because we are counting even values. Because we also know that -2 is our starting point we can decide to start at $n=1$. but we also need our function to fluctuate between being positive and negative so we make it $\left\{\left( -1 \right) ^{n}\cdot 2n\right\}_{n=1}^{\infty}$. We can also write this with the lower bound being $n=0$ but this becomes more complex because we need to avoid having $n^{n}$ as this can cause issues with a computer. }

\paragraph{For this function we would have $\left\{\left( -2n-2 \right) \left( -1 \right) ^{n}\right\}_{n=0}^{\infty}$. The general strategy for finding such a function is taking the lower bound and matching it with what our starting value of the set is. }

\paragraph{Ex. come up with the function for the sequence}

\[
	\left\{0,1,0,-1,0,1,0,-1 \ldots \right\}
.\]

\paragraph{This is a sequence that is defined as $\left\{\sin^{}( \frac{n\pi^{}}{2} )\right\}_{n=0}$. }

\paragraph{But this can also be defined with cos as $\left\{\cos^{}( \frac{n\pi^{}}{2} )\right\}_{n=-1}$.}
\paragraph{Sequences that are fluctuating between 3 values like this are generally defined by sin or cos.}



\section{Arithmetic sequences}%
\label{sec:Arithmetic sequences}

\paragraph{Ex.}
\[
	\left\{ 142,146,150,154 \ldots\right\}
.\] 
\paragraph{These are usually present in IQ tests for children. In this case we can define the next number in the sequence as 158. If we are using addition to get to the next number, we can define this as an arithmetic sequence that can be defined with a linear equation in the form $y=mx+b$. For this case we would have $\left\{y=4x+142\right\}_{n=0}$. This can also be proven by just graphing the points, which can be often helpful in finding the equation for a sequence.}

\paragraph{Ex.}

\[
\left\{8,3,-2,-7 \ldots \right\}
.\] 

\paragraph{Which can be defined with $\left\{-5n+8\right\}_{n=0}$. We can also start it at $n=1$ using the equation $\left\{-5n+13\right\}_{n=1}$.}

\subsection{Trick for substitution}%
\label{sub:Trick for substitution}

\paragraph{If we have our function$\left\{-5n+8\right\}_{n=0}$ and we want to switch our $n=0$ to $n=1$, we can use a dummy variable such as $m$ and define it as $m=n-1$. We can now sub it in for our old function to get $\left\{-5\left( m-1 \right) +8\right\}_{m=1}$. This is essentially u sub from calc 2 but brought into our functions.  }

\section{Introduction of calculus into sequences}%
\label{sec:Introduction of calculus into sequences}
\paragraph{One of the goals of this course is to be able to find the limits of a sequence. }
\paragraph{Definition: A sequence $\left\{a_{n}\right\}$ has limit L when $\lim_{n \to \infty} a_n = L$. If $\lim_{n \to \infty} a_n$ exists, we can say that the sequence converges to a point $L$. If the limit doesn't exist, we can instead say that the series will diverge. }

\paragraph{Ex. Find $\lim_{n \to \infty} 1+\frac{1}{n}$.}

\paragraph{In calc 1 we would have written it as $\lim_{x \to \infty} 1+\frac{1}{n}$ because we weren't only talking about integers. }

\paragraph{Because we distribute the lim we can get}

\[
\lim_{n \to \infty} 1 + \lim_{n \to \infty} \frac{1}{x}
.\] 
Which equates to 
\[
1+0=1
.\] 

\subsection{Theorem}%
\label{sub:Theorem}
\[
\lim_{n \to \infty} \frac{1}{n^{P}}=0 \text{ when }p>0
.\] 
This should be used whenever we have a similar situation. 

\subsubsection{Remember to use the limit laws when applying them, such as}

\begin{gather*}
\text{ Sum Law: } \lim_{n \to \infty} \left( a_n+-b_n \right) = \lim_{n \to \infty} a_n +- \lim_{n \to \infty} b_n\\
\text{ Constant Multiple Law } \lim_{n \to \infty} c\cdot a_n = c\cdot \lim_{n \to \infty} a_n\\
\text{ Product Law } \lim_{n \to \infty} a_n\cdot b_n = \lim_{n \to \infty} a_n \cdot \lim_{n \to \infty} b_n\\
\text{ Quotient Law } \lim_{n \to \infty} \frac{a_n}{b_n} = \frac{\lim_{n \to \infty} a_n}{\lim_{n \to \infty} b_n} \text{ when } \lim_{n \to \infty} b_n \neq 0\\
\text{ Power Law } \lim_{n \to \infty} a_n^{p} = \left( \lim_{n \to \infty} a_n \right) ^{p} \text{ when } p>0\\
\text{ Squeeze Theorem } \text{ if } a_n \leq b_n \leq c_n \text{ and } \lim_{n \to \infty} a_n = \lim_{n \to \infty} c_n = L \text{ then } \lim_{n \to \infty} b_n = L
\end{gather*}
\newpage
\paragraph{Ex.}
\[
\lim_{n \to \infty} \frac{\left( -1 \right) ^{n}}{n}
.\] 

\paragraph{Using limit laws we could say that }

\[
	\left( \lim_{n \to \infty} \frac{1}{n} \right) \left( \lim_{n \to \infty} \left( -1 \right) ^{n} \right) 
.\] 

Which simplifies to $0 \cdot \text{ Indeterminate }$. Which just leaves it undetermined. 

\paragraph{Luckily we have a theorem that states that if we have a sequence $\lim_{n \to \infty} a_n = 0$, we can rewrite it as $\lim_{n \to \infty} \|a_n\| = 0$ which we can take to our last equation and find}

\[
\lim_{n \to \infty} \|\frac{\left( -1 \right) ^{n}}{n}\|=\lim_{n \to \infty} \frac{1}{n}=0
.\] 
\paragraph{Which proves our original question to be 0.}

