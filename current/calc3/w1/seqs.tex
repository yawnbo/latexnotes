\documentclass[a4paper]{article}

\usepackage[utf8]{inputenc}
\usepackage[T1]{fontenc}
\usepackage{textcomp}
\usepackage[english]{babel}
\usepackage{amsmath, amssymb}
\title{Day 1 sequences}
\author{yawnbo}
\date{\today}

\maketitle

\pdfsuppresswarningpagegroup=1

\begin{document}
\paragraph{Office hours will be held at L218 on the given dates of the front page of canvas}%
\label{par:Office hours will be held at L218 on the given dates of the front page of canvas}
\paragraph{Room that class meets in is subject to change and will be announced through canvas }%
\label{par:Room that class meets in is subject to change and will be announced through canvas }

\section{Intro to sequences chapter 11.1}%
\label{sec:Intro to sequences}

\paragraph{Think about functions. A function can be defined as something that takes an input and assigns at most one output. We usually have things like $f\left( x \right) = x^2$ which gives us the continuous graph of $x^2$. }

\paragraph{When talking about sequences we are essentially breaking the function into parts that we can give another function. This can be done by looking at the domain of the function to start. Looking at $x^2$ we see a dense set of points that has infinitely many values. This implies that if we take the set $\left[ 0,1 \right] $ we get a dense set of points that has the same amount of points as the whole set. This is proven by Cantor. }

\paragraph{If instead have a number line of integers, we have something that is countibly infinite instead of uncountably infinite like the first example with a dense set of points. We can use the symbol Aleph to represent the smallest infinite cardinal number. This basically serves as a way to scale certain infinities.}

\paragraph{Going back to the graph of $x^2$, if we throw away all values between our integers, we are left with a discrete graph that changes our domain to real integers. This is exactly what a sequence is. }

\subsection{Definition of a sequence}%
\label{sub:Definition of a sequence}
\paragraph{A sequence is a function with a COUNTABLE domain. This is usually Integers or natural/whole numbers which is what we will usually see. (natural numbers are just positive integers). The last one is Q, or all real/natural fractions. (this will not be used during this course due to complexity) }

\newpage
\subsection{Coming up with a sequence}%
\label{sub:Coming up with a sequence}
\paragraph{Lets take the sequence $\left( 0,2,4,6,8 \ldots \right) $. This is just a sequence of even numbers. We can assign index numbers to each of these (ordinals) so, $n_1 = 0, n_2 = 2, n_3 = 4 \ldots$. Now our goal is to find some $f\left( n \right) $ so that we find the $n^{th}$ even number. We are essentially just taking some countable domain and giving it a function. For this we can define this as, $f\left( n \right) =2n-2$. We should avoid using $x$ as our dummy variable as it is commonly referring to uncountable doamins, but we want to look at a discrete domain so we can use n. }

\paragraph{Another notation for this can be using $a$. This would look like $a_0 = 0, a_1 = 2, a_2 = 4 \ldots$. This is generally the common notation for it so this should be the focus. However, notice that $a_n$ starts counting at 0. This means that our function should change to $f\left( n \right) =2n$. Which can be tested as so, $f\left( 3 \right) =2\left( 3 \right) =6$. But we also need to replace our function with our a notation as so, $a_n=2n$. }

\subsection{Set notation }%
\label{sub:Set notation }

\paragraph{For set notation we generally use something like $\{a_n\} = \{2n\}_{n=0}^{\infty}$. This is generally more useful as we can define our starting and ending points. In this example, we would start at 0 and end at infinity (infinity is implied and doesnt have to be written). For another set like $\{0,2,4,6,8\} $ we would use  $\{2n\}_{n=0}^{8}$. Another quirk of this is that we can't use $-\infty$ as our starting point because its not a real value. If we wanted to go to $-\infty$ we can just write it as $\{2n\}_{n = 0}^{-\infty}$. }

\subsection{Explicitly defined sequences}%
\label{sub:Explicitly defined sequences}

\paragraph{The above example is a explicitly defined sequence that can fit in something nice like $f\left( n \right) $, but we can also have recursive sequences like the Fibonacci Sequence. This sequence works by having the inputs $\left( 1,1 \right) $ and the sequence would add those two together. This sequence just keeps taking the previous 2 outputs in order to get the next one. So this would continue as, $\left( 1,2 \right) , \left( 2,3 \right) , \left( 3,5 \right) \ldots$ and so on. }

\paragraph{For this we don't have a nice function to define it so we use Recursive definitions such as, $a_0 = 1 \text{ and } a_1 = 1 $ for $n>1$ $a_n=a_{n-1}+a_{n-2}$. This definition works such that if we were to pick a number $n$ we can recursively backtrack through our list to find our answer. (without a given list we would just use the final sum as we count up to n)}



\end{document}
