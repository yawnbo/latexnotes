\section{Thursday}%
\label{sec:Thursday}
This started with quiz questions that I came late to, so starting with 7, 
\subsection{Question 7}%
\label{sub:Question 7}
Instead of using ast on the absolute we can take subsequences to prove the the whole $ a_n $ diverges. 

\subsection{Question 3}%
\label{sub:Question 3}
Use ASET with integrals and remember to use the p-test to solve the integral,
\[
\int_{ n }^{ \infty } \frac{ 1 }{ x\left( \ln^{  } \left( x \right)  \right) ^2 } 
.\] 

\section*{11.9 Building power series}%
\label{sec:Building power series}
Using the identity
\[
\sum_{ n=0 } ^{ \infty } x^{ n } = \frac{ 1 }{ 1-x } , \left| x \right| < 1
.\] 
\paragraph{Ex.}
Make 
\[
\frac{ 1 }{ 1+x^{ 4 } } \text{ into a power series. }
.\] 
To start, we know we need a negative so we flip it using a double negative and build the series,
\[
\frac{ 1 }{ 1+x^{ 4 } } = \frac{ 1 }{ 1- \left( -x^{ 4 } \right) } = \sum_{ n=0 } ^{ \infty } \left( -x^{ 4 } \right) ^{ n } = \sum_{ n=0 } ^{ \infty } \left( -1 \right) ^{ n } x^{ 4n }
.\] 
\paragraph{Ex.}
\[
\frac{ 5x }{ 2-x } 
.\] 
Know that we want $ \frac{ 1 }{ 1-\left( \text{  } \right)  }  $ so lets move our 5x out,
\[
=5x\left( \frac{ 1 }{ 2-x }  \right) = \frac{ 5x }{ 2 } \left( \frac{ 1 }{ 1-\left( \frac{ x }{ 2 }  \right)  }  \right) = \frac{ 5x }{ 2 } \sum_{ n=0 } ^{ \infty } \left( \frac{ x }{ 2 }  \right) ^{ n }
.\] 
We have our series, but we want to put our x back in, so we multiply by 5x,
\[
=\sum_{ n=0 } ^{ \infty } \frac{ 5x }{ 2 } \cdot \frac{ x^{ n } }{ 2^{ n } }= \sum_{ n=0 } ^{ \infty } \frac{ 5x^{ n+1 } }{ 2^{ n+1 } }= \sum_{ n=0 } ^{ \infty } 5\left( \frac{ x }{ 2 }  \right) ^{ n+1 }
.\] 

\subsection*{Integrating and Differentiating Power Series}%
\label{sub:Integrating and Differentiating Power Series}
For any power series $ \left( \sum_{ n=0 } ^{ \infty } a_n\left( x-c \right) ^{ n } \right)  $ you can take the derivative, 
\[
\left( \frac{ d }{ dx } \left[ \left( \sum_{ n=0 } ^{ \infty } a_n\left( x-c \right) ^{ n } \right) \right]  = \sum_{ n=0 } ^{ \infty } \frac{ d }{ dx } \left( a_n\left( x-c \right) ^{ n } \right) \right) 
.\] 
or take the integral, 
\[
\int_{  }^{  } \left[ \sum_{ n=0 } ^{ \infty } a_n\left( x-c \right) ^{ n } \right] = \sum_{ n=0 } ^{ \infty } a_n\left( x-c \right) ^{ n }
.\] 
\paragraph{Ex.}
Show 
\[
\frac{ 1 }{ \left( 1-x \right) ^2 } = 1+ 2x+ 3x^2 + 4x ^3 + \ldots
.\] 
We're trying to find what $ f\left( x \right)  $ has $ f'\left( x \right) \frac{ 1 }{ \left( 1-x \right) ^2 }  $? So we integrate,
\[
\int_{  }^{  } \frac{ dx }{ \left( 1-x \right) ^2 } = \int_{  }^{  } -\frac{ du }{ u^2 } = -\frac{ u^{ -2+1 } }{ -2+1 } = \frac{ 1 }{ u } = \frac{ 1 }{ 1-x } 
.\] 
This means we can think of $ \frac{ 1 }{ \left( 1-x \right) ^2 } = f'\left( x \right)  $ where $ f\left( x \right) = \frac{ 1 }{ 1-x } = \sum_{ n=0 } ^{ \infty } x^{ n } $ or $ \frac{ d }{ dx } \sum_{ n=0 } ^{ \infty } x^{ n } = \sum_{ n=0 } ^{ \infty } nx^{ n-1 }=\frac{ 1 }{ \left( 1-x \right) ^2 }  $. \\ \\
The reason we have to do it this way is because there is no way to turn $ \frac{ 1 }{ \left( 1-x \right) ^2 }  $ into $ \frac{ 1 }{ 1- \left( \text{  } \right)  }  $ normally, so we instead use the integral for it and differentiate it to get the series.

\paragraph{Another ex.}
Find a power series for 
\[
f\left( x \right) = \ln^{  } \left( 1+x^{ 4 } \right) 
.\] 
The first step would be to try and get $ f\left( x \right)  $ into our form of $ \frac{ 1 }{ 1- \left( \text{  } \right)  }  $.  For this we can use $ \frac{ d }{ dx }  $ because we know that taking the derivative of ln will be $ \frac{ 1 }{ x }  $. So,
\[
f'\left( x \right) = \frac{ 4x^3 }{ 1+x^{ 4 } } = 4x\left( \frac{ 1 }{ 1-\left( x^{ 4 } \right)  }  \right) = 4x^3\sum_{ n=0 } ^{ \infty } \left( -x^{ 4 } \right) ^{ n }
.\] 
And combining,
\[
\sum_{ n=0 } ^{ \infty } 4\left( -1 \right) ^{ n }x^{ 4n+3 }
.\] 
We know have a power series, but it's for $ f'\left( x \right)  $ so we integrate to get it back to $ f\left( x \right)  $. 
\[
\int_{  }^{  } f'\left( x \right) dx = \int_{  }^{  } \sum_{ n=0 } ^{ \infty } 4\left( -1 \right) ^{ n }x^{ 4n+3 }dx= \sum_{ n=0 } ^{ \infty } 4\left( -1 \right) ^{ n }\int_{  }^{  } x^{ 4n+3 }dx
.\] 
\[
=\sum_{ n=0 } ^{ \infty } 4\left( -1 \right) ^{ n }\frac{ x^{ 4n+4 } }{ 4n+4 }+C= \ln^{  } \left( 1+x^{ 4 } \right) 
.\] 
Now to find our C value we can pick a point on our $ f\left( x \right)  $ so try $ x=0 $ to make $ y=0 $ and find that our whole sum will also be 0 meaning that $ C=0 $ and we have our power series for $ \ln^{  } \left( 1+x^{ 4 } \right)  $ to be $ \sum_{ n=0 } ^{ \infty } 4\left( -1 \right) ^{ n } \frac{ x^{ 4n+4 } }{ 4n+4 } $. \\ \\
As an extra, we can find our IOC using ratio test.
\[
\lim_{ n \to \infty} \left| \frac{ a_{ n+1 } }{ a_n } \right| = \lim_{ n \to \infty} \frac{ 4 \left|  x^{ 4\left( n+1 \right) +4 } \right|}{ 4\left( n+1 \right) +4 }\cdot \frac{ 4n+4 }{ 4 \left| x^{ 4n+4 } \right| } = \lim_{ n \to \infty} \frac{ n+1 }{ n+2 }\cdot \left| x^{ 4 } \right|<1
.\] 
So our ratio is $ -1<x<1 $ and we now test -1 and 1. When we test -1, we find that our numerator is always positive 1 so our sum is basically $ \sum_{ n=0 } ^{ \infty } \left( -1 \right) ^{ n }\frac{ 1 }{ n+1 }  $ and by AST we are convergent because $ \frac{ 1 }{ n+1 }  $ goes to 0 and we have a decreasing function. The same thing happpens for $ x=1 $ so we know our IOC is $ \left[ -1,1 \right]  $. 
