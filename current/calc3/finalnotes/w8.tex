\section{Review of exam 1}%
\label{sec:Review of exam 1}
\subsection{Question 4}%
\label{sub:Question 4}
\[
\sum_{ n=1 } ^{ \infty } \frac{ \left( -1 \right) ^{ n+1 } }{ n+\sqrt{ n} }
.\] 
Using an absolute you can get rid of the absolute and use limit comparison to show 
\[
\lim_{ n \to \infty} \frac{ \frac{ 1 }{ n }  }{ \frac{ 1 }{ n+\sqrt{ n} }  }=\lim_{ n \to \infty} \frac{ n+\sqrt{ n} }{ n }=1
.\] 
Which means that both will either converge or diverge, but because $ \frac{ 1 }{ n }  $ is the harmonic we can see that both diverge and we can prove that it's conditional because AST shows that it's decreasing and limit is 0.
\subsection{Question 5}%
\label{sub:Question 5}
\[
\sum_{ n=2 } ^{ \infty } \frac{ 1 }{ n\left( n+2 \right)  } 
.\] 
Using partials we can deconstruct the series into 
\[
\sum_{ n=2 } ^{ \infty } \left( \frac{ 1 }{ 2n  } - \frac{ 1 }{ 2\left( n+2 \right)  }  \right) = \sum_{ n=2 } ^{ \infty } \frac{ 1 }{ 2 } \left( \frac{ 1 }{ n } -\frac{ 1 }{ n+2 }  \right) 
.\] 
Writing out terms you can see it becomes the telescoping
\[
\frac{ 1 }{ 2 } \left[ \left( \frac{ 1 }{ 2 } -\frac{ 1 }{ 4 }  \right) + \left( \frac{ 1 }{ 3 } -\frac{ 1 }{ 5 }  \right) + \ldots \right] 
.\] 
which cancels and leaves us with
\[
\frac{ 1 }{ 2 } \left[ \frac{ 1 }{ 2 } +\frac{ 1 }{ 3 }  \right] = \frac{ 5 }{ 12 } 
.\] 
\subsection{Question 7 friday}%
\label{sub:Question 7 friday}
\[
\ln^{  } \left( \sqrt{ 1-x^2} \right) 
.\] 
This can become $ \frac{ 1 }{ 2 } \ln^{  } \left( 1-x^2 \right)  $ and taking a $ \frac{ d }{ dx }  $ this becomes
\[
\frac{ 1 }{ 2 } \cdot \frac{ 1 }{ 1-x^2 } \cdot -2x= \frac{ -x }{ 1-x^2 } = -x\left( \frac{ 1 }{ 1-x^2 }  \right) =-\sum_{ n=0 } ^{ \infty } x^{ 2n+1 }
.\] 
Now taking the integral to reverse or derivative,
\[
\int_{  }^{  } -\sum_{ n=0 } ^{ \infty } x^{ 2n+1 }=-\sum_{ n=0 } ^{ \infty } \frac{ x^{ 2n+2 } }{ 2n+2 }
.\] 
Proving IOC can be done with ratio test as
\[
	\lim_{ n \to \infty} \left| \frac{ x^{ 2n+4 } }{ \cancel{ 2n+4 }} \cdot \frac{ 2n+2 }{ x^{ 2n+2 } } \right| = \left| x^2 \right| <1 \to -1<x<1 
.\] 

\section{Make a taylor without taylors thm }%
\label{sec:Make a taylor without taylors thm}
For $ e^{ x } $,
\[
1-x\le e^{ -x } \le 1
.\] 
\[
\int_{ 0 }^{ x } \left( 1-t \right) dt \le \int_{ 0 }^{ x } e^{ -t }dt \le \int_{ - }^{ x } 1dt
.\] 
\[
x-\frac{ x^2 }{ 2 } \le -e^{ -x }+1 \le x
.\] 
Integrate again,
\[
\int_{ 0 }^{ x } \left( t-\frac{ t^2 }{ 2 }  \right) dt\le \int_{ 0 }^{ x } \left( -e^{ -t }+1 \right) dt \le \int_{ 0 }^{ x } tdt
.\] 
\[
\frac{ x^2 }{ 2 } -\frac{ x^3 }{ 6 } \le e^{ x }+x \le 1 + \frac{ x^2 }{ 2 } 
.\] 
\[
1-x + \frac{ x^2 }{ 2 } -\frac{ x^3 }{ 6 } \le e^{ x } \le 1 -x +\frac{ x^2 }{ 2 } 
.\] 
Now we can use the squeeze theorem to show that $ e^{ x } $ is equal to the taylor series of $ e^{ x } $, or 
\[
e^{ x }= \sum_{ n=0 } ^{ \infty } \frac{ x^{ n } }{ n! }
.\] 
and
\[
e^{ -x }=\sum_{ n=0 } ^{ \infty } \frac{ \frac{ \left( -x \right) ^{ n } }{ \left( -1 \right) ^{ n }x^{ n } } }{ n! } 
.\] 
Which matches our created terms and shows that we made the MacLaurin series for $ e^{ x } $ without using the Taylor formula.
\section{Eulers formula}%
\label{sec:Eulers formula}
\[
e^{ i \theta } \cos^{  } \left( \theta \right) + i \sin^{  } \left( \theta \right) 
.\] 
