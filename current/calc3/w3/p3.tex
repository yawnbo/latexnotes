\section{01/24/25 Alt limit and comparison last}%
\label{sec:01/24/25 Alt limit and comparison last}

\paragraph{Ex.}
\paragraph{Determine Conv/div using limit comp. Using limit comp test}
\[
\sum_{ n=1 } ^{ \infty } \frac{ 1 }{ \sqrt[ 3 ]{ 3n^{ 4 }+ 1}  } 
.\] 

Start with dominance and take our dominant terms to find
\[
\sum_{ n=1 } ^{ \infty } \frac{ 1 }{ \sqrt[ 3 ]{ 3n^{ 4 }+1 }  } \approx \sum_{  } ^{  } \frac{ 1 }{ \sqrt[ 3 ]{ 3n^{ 4 } }  } =\sum_{  } ^{  } \frac{ 1 }{ \sqrt[ 3 ]{ 3 } n^{ \frac{ 4 }{ 3 }  } } 
.\] 
and because $ p > 1 $ we will assume convergence and test for it. Now we need verift using $ 0<\lim_{ n \to \infty} \frac{ a_n }{ b_n } <\infty $. So, 
\[
\lim_{ n \to \infty} \frac{ \frac{ 1 }{ \sqrt[ 3 ]{ 3n^{ 4 }+1 }  }  } {\frac{ 1 }{ \sqrt[ 3 ]{ 3n^{ 4 } }  }}  = \lim_{ n \to \infty} \sqrt[ 3 ]{ \frac{ 3n^{ 4 } }{ 3n^{ 4 }+1 } }= 1
.\] 

\subsection*{11.5 Alternating series test (AST)}%
\label{sub:11.5 Alternating series test (AST)}

\paragraph{Definition\\}
$ b_n > 0 $ for all n.
\paragraph{Form:}
\[
b_1 - b_2 + b_3 - b_4 + b_5 - b_6 + \ldots \text{ or }\sum_{ n=1 } ^{ \infty } \left( -1 \right) ^{ n+1 }b_n
.\] 
or
\[
-b_1 + b_2 - b_3 + b_4 - b_5 + b_6 - \ldots \text{ or }\sum_{ n=1 } ^{ \infty } \left( -1 \right) ^{ n }b_n
.\] 
is an alternating series.

\paragraph{Theorem.} AST, we start with the same conditions, so $ b_n > 0 $ for all n and assume $ \lim_{ n \to \infty} b_n=0 $ and our series is monotonic decreasing ($ b_1 > b_2 > b_3 > \ldots $) (which just states  that the terms are decreasing to 0) then $ \sum_{ n=1 } ^{ \infty } \left( -1 \right)^{ n+1 }b_n  $

\paragraph{Theorem} Alternating series Estimation Theorem. Let $ S=\sum_{ n=1 } ^{ \infty } \left( -1 \right) ^{ n+1 }b_n $ with the same conditions on $ b_n $ as AST. Then the $ \left\| \text{ error } \right\|=\left\| S-S_n \right\| \le b_{ n+1 } $. \\
\label{par:Theorem}
This is just saying that if we stop at a definite value $ n $ then our error will be less than what our next b value will be. 

\paragraph{Ex.}
\[
\sum_{ n=1 } ^{ \infty } \left( -1 \right) ^{ n+1 }\frac{ 1 }{ n^2 } =S
.\] 
First we can look at our AST theorem, which, (1) being that $ b_n $ is monotnoic decreasing and that $ \lim_{ n \to \infty} b_n = 0 $, now find 
\paragraph{How close is $ S_4 $ to $ S $?\\}
So we start with our error, 
\[
\left\| S-S_{ 4 } \right\|\le b_5 = \frac{ 1 }{ 25 } 
.\] 

\paragraph{Ex.}
\[
\text{ For }\sum_{ n=1 } ^{ \infty } \left( -1 \right) ^{ n+1 }\frac{ 1 }{ n! } 
.\] 
\paragraph{Determine $ n $ so that $ S_n $ is within $ 0.001 $ of $ S $.\\}

Start with error, 

\[
\left\| S-S_n \right\|\le 0.001
.\] 
So we know we need $ b_{ n+1 }\le 0.001 $. So,
\begin{gather*}
b_n = \frac{ 1 }{ n! } \to n=1:1 n=2:\frac{ 1 }{ 2! } =\frac{ 1 }{ 2 } n=3:\frac{ 1 }{ 3! } =\frac{ 1 }{ 6 } \\
\ldots n=7:\frac{ 1 }{ 7! } =\frac{ 1 }{ 5040 } <\frac{ 1 }{ 1000 } 
\end{gather*}

This is just one way of solving something like this, but we can also do it with algebra if we have a better looking function like
\[
\frac{ 1 }{ \left( n+1 \right) ^2 } \le 0.0001 \to \frac{ 1 }{ 0.0001 } \le \left( n+1 \right) ^2 \to \sqrt{ \frac{ 1 }{ 0.0001 } }\le n+1 \implies \sqrt{ \frac{ 1 }{ 0.0001 } -1}\le n
.\] 
Which would come out to $ n\ge 99 $ which means that we need to find the sum of 99 terms to get close enough of the absolute sum of the series $ S $. 

\section{End of week 3}%
\label{sec:End of week 3}

