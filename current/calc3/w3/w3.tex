\documentclass[a4paper]{article}

\usepackage[utf8]{inputenc}
\usepackage[T1]{fontenc}
\usepackage{textcomp}
\usepackage[english]{babel}
\usepackage{amsmath, amssymb}

\title{Second half of integral tests and w3}
\author{yawnbo}
\date{\today}

\pdfsuppresswarningpagegroup=1

\begin{document}
\maketitle
\paragraph{Note that quiz 2 is due next Tuesday. }
\section{Day 1 week3}%
\label{sec:Day 1 week3}

\section*{11.3 Convergent series with the integral test}%
\label{sec:11.3 Convergent series with the integral test}
Given a series $ \sum_{ k=1 } ^{ \infty } a_k $ has a remainder of $ R_n $ that is called or error and can be found using
\[
R_n = \sum_{ k=1 } ^{ \infty } a_k - \sum_{ k=1 } ^{ n } a_k = a_{ n+1 }+a_{ n+2 }+a_{ n+3 } \ldots
.\] 
This is just saying that when we have a sum to the number n that we are ignoring all values that are above n, so our error from the actual can be found with the above. \\

The important part is that we can use integral to find and estimate our area. Think about this using right and left endpoint rectangles. This can be written using the inequality
\[
\int_{ n+1 }^{ \infty } f\left( x \right) dx<R_n < \int_{ n }^{ \infty } f\left( x \right) dx
.\] 
And 
\[
S=\sum_{ k=1 } ^{ \infty } a_k = \sum_{ k=1 } ^{ n } a_k + R_n = S_n + R_n 
.\] 
Now combining our inequalities we find that 
\[
S_n + \int_{ n+1 }^{ \infty } f\left( x \right) dx<S < \int_{ n }^{ \infty } f\left( x \right) dx+S_n
.\] 
Now calling our lower bound $ L_n $ and upper bound $ U_n $ we can just call it $ L_n < S< U_n $. This inequality is important because it allows us to estimate the sum of a series by finding the integral of the function.


\subsubsection*{Ex. }
\paragraph{Consider the series $ \sum_{ i=1 } ^{ \infty } \frac{ 1 }{ n^2 }  $}
\paragraph{Part a) Find an upper bound for the remainder in terms of n}
Start by figuring our $ S_n $. Write out some terms from our sum first,
\[
\sum_{ n=1 } ^{ \infty } \frac{ 1 }{ n^2 } =1+\frac{ 1 }{ 2^2 } +\frac{ 1 }{ 3^2 } +\frac{ 1 }{ 4^2 } 
.\] 
Let's start by finding the upper bound $ R_n < \int_{ n }^{ \infty } f\left( x \right) dx$ as $ \frac{ 1 }{ n }  $ from our p theorem. 

\paragraph{Part b) find how many terms are needed to ensure that the remainder ($ error $) is less than $ 10^{ -3 } $. \\}
Start by writing our inequality to be $ \frac{ 1 }{ n } < 10^{ -3 } $ or $ 10^{ 3 }<n $. Now we need to add one to our number to find that n would have to be at least $ 1001 $ terms. 

\paragraph{Find the lower and upper bounds ($ L_n $ and $ U_n $ respectively) on the exact value of the series\\}
Start by letting $ n=1001 $ and plug into our inequality $ L_n \le \sum_{ n=1 } ^{ \infty } a_n \le U_n$ 
\[
S_{ 1001 }\int_{ 1002 }^{ \infty } \frac{ 1 }{ x^2 } dx\le S \le \int_{ 1001 }^{ \infty } \frac{ 1 }{ x^2 } dx + S_{ 1001 }
.\] 
Computing this out we find,
\[
	S_{ 1001} + \frac{ 1002^{ 1-2 } }{ 2-1 } \le S \le \frac{ 1001^{ 1-2 } }{ 2-1 } + S_{ 1001 }
.\] 

\paragraph{Part D) find an interval in which the value of the series must lie if you approximate it using ten terms of the series. \\}
Start with our value of n being 10. Now write our inequality to be $ L_{ 10 } \le S\le U_{ 10 } $. So,
\[
S_{ 10 }+\int_{ 11 }^{ \infty } \frac{ 1 }{ x^2 } dx\le S\le S_{ 10 }+\int_{ 10 }^{ \infty } \frac{ 1 }{ x^2 } dx
.\] 
\[
S_{ 10 }+\frac{ 11^{ 1-2 } }{ 2-1 }\le S\le S_{ 10 }+ \frac{ 10^{ 1-2 } }{ 2-1 }
.\] 
Plugging this into a calculator gives us $ S_n = 1.549767731 $, so our inequality is $ 1.640676822< S < 1.649767731$. Which is a good estimate. Given our actual answer to be $ \frac{ \pi^2 }{ 6 }  $, we find that the estimated value of $ 1.644934067 $ is in our range. 
\section*{Tests cont.}%
\label{sec:Tests}
\paragraph{Note that quiz 3 is open by tomorrow and due in a week I assume.}

\paragraph{Ex.}
Apply the ratio test to 
\[
\sum_{ n=0 } ^{ \infty } \left( -1 \right) ^{ n }\frac{ \sqrt{ n} }{ n+1 }
.\] 
Note that our sequence can be called $ a_n $. Now if we sub n for n+1 we can now take the limit,
\[
\lim_{ n \to \infty} \left\| \frac{ a_{ n+1 } }{ a_n } \right\| = \lim_{ n \to \infty} \frac{ \sqrt{ n+1} }{ n+2 }\cdot \frac{ n+1 }{ \sqrt{ n} }= \sqrt{ \frac{ n+1 }{ n }} \cdot \frac{ n+1 }{ n+2 } = 1\cdot 1 = 1
.\] 
Which makes our test inconclusive. This just means that we have to use another test to determine convergence or divergence. \\
Instead looking at our sequence, we can use AST to find that it's monotonically decreasing and that the limit is 0. This means that the series converges by AST. So because $ \frac{ \sqrt{ n} }{ n+1 } $ is monotonically decreasing and $ \lim_{ n \to \infty} \frac{ \sqrt{ n} }{ n+1 } = 0 $, we can say that the series converges by AST.

\paragraph{What if we also wanted to find if the absolute also converges?}
\[
\sum_{ } ^{  } |a_n| = \sum_{  } ^{  } \frac{ \sqrt{ n} }{ n+1 }
.\] 
Comparing to a smaller function $ \sum_{  } ^{  } \frac{ 1 }{ n+1 } $ shows that the series converges by p-test. This means that the series conditionally converges.
\subsection*{11.5.4}%
\label{sub:11.5.4}
\paragraph{The root test }
(i) If $ \lim_{ n \to \infty} \sqrt[ n ]{ \left\| a_n \right\| } <1 $, then $ \sum_{  } ^{  } \left\| a_n \right\| $ converges \\
(ii) If $ \lim_{ n \to \infty} \sqrt[ n ]{ \left\| a_n \right\| } >1 $, then $ \sum_{  } ^{  } a_n $ diverges \\
(iii) If $ \lim_{ n \to \infty} \sqrt[ n ]{ \left\| a_n \right\| } =1 $, then inconclusive. \\ \\ 
\paragraph{Ex.}
\[
\sum_{ n=1 } ^{ \infty } \left( \frac{ n+1 }{ 2n } \right) ^{ n }
.\] 
We commonly use this test when we have something complex to the power of n, so,
\[
\sqrt[ n ]{ \left\| a_n \right\| } =\sqrt[ n ]{ \frac{ n+1 }{ 2n }^{ n } } = \lim_{ n \to \infty} \frac{ n+1 }{ 2n }=\frac{ 1 }{ 2 } < 1
.\] 
Which proves that our sum $ \sum_{  } ^{  } a_n $ converges because its less than 1.

\paragraph{Ex.}
\[
\sum_{ k=1 } ^{ \infty } \left( 1+\frac{ 3 }{ k }  \right) ^{ k^2 }
.\] 
Start by applying the root,
\[
\sqrt[ k ]{ \left\| a_n \right\| } = \sqrt[ k ]{ \left( 1+\frac{ 3 }{ k }  \right) ^{ k^2 } } = \lim_{ n \to \infty}  \left( 1+\frac{ 3 }{ k }  \right) ^{ k }
.\] 
Solving this at the above point makes it indecisive. So we can use the ratio test to find that the series converges. Let our limit equal L, then,
\[
k \ln^{  } \left( 1+\frac{ 3 }{ k }  \right) = \ln^{  } \left( L \right) 
.\] 
Now as $ k \to \infty $ we get an indeterminate form so we can use L'Hopital's rule to get,
\[
k\ln^{  } \left( 1+\frac{ 3 }{ k }  \right) = \frac{ \frac{ 1 }{ 1+\frac{ 3 }{ k }  }\cdot \left( -\frac{ 3 }{ k^2 }  \right)  }{ -\frac{ 1 }{ k^2 }  } = \frac{ 3 }{ 1+\frac{ 3 }{ k }  } \to 3 = \ln^{  } \left( L \right) 
.\] 
So $ L = e^{ 3 } $, which means that the series diverges because $ e^{ 3  }>1 $.
\paragraph{Everything above is done up to section 11.6 and 11.7 is just review on how to actually use these tests}

\paragraph{Example list to do if wanted (compare first to second to prove divergence or convergence)}
\begin{align*}
	1.& \sum_{  } ^{  } \frac{ 1 }{ 5^{ n } } , \sum_{  } ^{  } \frac{ 1 }{ 5^{ n }+n } \\
2.&\sum_{  } ^{  } \frac{ \left( -1 \right) ^{ n } }{ n^{ \frac{ 3 }{ 2 }  } }, \sum_{  } ^{  } \frac{ 1 }{ n^{ \frac{ 3 }{ 2 }  } } \\
3.&\sum_{  } ^{  } \frac{ n }{ r^{ n } } , \sum_{  } ^{  } \frac{ 3^{ n } }{ n } \\
4.&\sum_{  } ^{  } \frac{ n+1 }{ n }, \sum_{  } ^{  } \left( -1 \right) ^{ n } \frac{ n+1 }{ n } \\
5.&\sum_{ n=1 } ^{  } \frac{ n }{ n^2+1 } , \sum_{  } ^{  } \left( \frac{ n }{ n^2+1 }  \right) ^{ n } \\
6.&\sum_{  } ^{  } \frac{ \ln^{  } \left( n \right)  }{ n }, \sum_{ n=10 } ^{  } \frac{ 1 }{ n\ln^{  } \left( n \right)  } \\
7.&\sum_{  } ^{  } \frac{ 1 }{ n+n! } , \sum_{  } ^{  } \left( \frac{ 1 }{ n } +\frac{ 1 }{ n! }  \right) \\
8.&\sum_{  } ^{  } \frac{ 1 }{ \sqrt{ n^2+1} } , \sum_{ } ^{  } \frac{ 1 }{ n\sqrt{ n^2+1} } \\
\end{align*}

\documentclass[a4paper]{article}

\usepackage[utf8]{inputenc}
\usepackage[T1]{fontenc}
\usepackage{textcomp}
\usepackage[english]{babel}
\usepackage{amsmath, amssymb}
\title{Review for exam 3}
\author{yawnbo}
\date{\today}

\maketitle

\pdfsuppresswarningpagegroup=1

\begin{document}
\paragraph{Question 5}
\[
\int_{}^{} e^{\sqrt[3]{x}}dx \text{ using u sub  }
.\] 
\[
u=x^{\frac{1}{3}} \text{ an } du=\frac{1}{3}x^{-\frac{2}{3}}dx
.\] 
Solving for dx gets 
\[
dx= \frac{3du}{x^{-\frac{2}{3}}}=3x^{\frac{2}{3}}du=3\left( x^{\frac{1}{3}} \right) ^2du=3u^2du
.\] 
Which can then be tabulated to
\[
3u^2e^{4}-6ue^{u}+6u+C
.\] 
and subbing back in x
\[
3x^{\frac{2}{3}}e^{\sqrt[3]{x}}-6^{\sqrt[3]{x}}e^{\sqrt[3]{x}}+6\sqrt[3]{x}+C
.\] 
\section{Comparison examples from yesterday}%
\label{sec:Comparison examples from yesterday}

\begin{align}
 &\int_{1}^{\infty} \frac{dx}{\sqrt{x}+e^{3x}} \\
 &\int_{0}^{0.5} \frac{dx}{x^{8}+x^2} \\
 &\int_{1}^{\infty} \frac{dx}{\sqrt{x^{5}+2}} \\
 &\int_{0}^{5} \frac{dx}{x^{\frac{1}{3}}+x^3} \\
 &\int_{0}^{\infty} \frac{dx}{\sqrt{x^{\frac{1}{3}+x^3}}} \\
 &\int_{1}^{\infty} \frac{dx}{x^{4}+e^{x}}\\
\end{align}
Generally guess which one we would assume we have, for one assume convergence, so we need somthing larger.
\newpage
\subsection{For number 5}%
\label{sub:For number 5}
\paragraph{This is a type 1 and two improper.}
\[
\int_{0}^{\infty} \frac{dx}{\sqrt{x^{\frac{1}{3}}+x^3}}
.\] 
becomes
\[
\int_{0}^{1} \frac{dx}{\sqrt{x^{\frac{1}{3}}+x^3}} + \int_{1}^{\infty} \frac{dx}{\sqrt{x^{\frac{1}{3}}+x^3}}
.\] 
By dominance theory we can look at the second integral of $x^3$ which converges by p theorem. 
Now we look at the first one to establish its divergence so we need a function thats bigger
\[
x^{\frac{1}{3}}+x^3 \le 2x^{\frac{1}{3}}
.\] 
We choose that number because when testing $x^3+x^3$ we find that just one of them is smaller than $x^{\frac{1}{3}}$ which means that if we are finding a larger function we should take our x to the power of a fraction.
Doing normal comparison we find that this converges instead and our analysis is wrong because we found a value that doesn't help us

Now we want to try something like $x^{-1}$ to get a value that is 

THiS ONE IS ON THE EXAM THIS NEEDS TO BE FIGURED OUT 
\[
\int_{0}^{\infty} \frac{dx}{\sqrt{x^{\frac{1}{3}}+x^3}}
.\] 
\end{document}


\end{document}
