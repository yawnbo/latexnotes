\documentclass[a4paper]{article}

\usepackage[utf8]{inputenc}
\usepackage[T1]{fontenc}
\usepackage{textcomp}
\usepackage[english]{babel}
\usepackage{amsmath, amssymb}

\title{Second half of integral test}
\author{yawnbo}
\date{\today}

\pdfsuppresswarningpagegroup=1

\begin{document}
\maketitle
\paragraph{Note that quiz 2 is due next Tuesday. }
\section*{11.3 Convergent series with the integral test}%
\label{sec:11.3 Convergent series with the integral test}
Given a series $ \sum_{ k=1 } ^{ \infty } a_k $ has a remainder of $ R_n $ that is called or error and can be found using
\[
R_n = \sum_{ k=1 } ^{ \infty } a_k - \sum_{ k=1 } ^{ n } a_k = a_{ n+1 }+a_{ n+2 }+a_{ n+3 } \ldots
.\] 
This is just saying that when we have a sum to the number n that we are ignoring all values that are above n, so our error from the actual can be found with the above. \\

The important part is that we can use integral to find and estimate our area. Think about this using right and left endpoint rectangles. This can be written using the inequality
\[
\int_{ n+1 }^{ \infty } f\left( x \right) dx<R_n < \int_{ n }^{ \infty } f\left( x \right) dx
.\] 
And 
\[
S=\sum_{ k=1 } ^{ \infty } a_k = \sum_{ k=1 } ^{ n } a_k + R_n = S_n + R_n 
.\] 
Now combining our inequalities we find that 
\[
S_n + \int_{ n+1 }^{ \infty } f\left( x \right) dx<S < \int_{ n }^{ \infty } f\left( x \right) dx+S_n
.\] 
Now calling our lower bound $ L_n $ and upper bound $ U_n $ we can just call it $ L_n < S< U_n $. This inequality is important because it allows us to estimate the sum of a series by finding the integral of the function.


\subsubsection*{Ex. }
\paragraph{Consider the series $ \sum_{ i=1 } ^{ \infty } \frac{ 1 }{ n^2 }  $}
\paragraph{Part a) Find an upper bound for the remainder in terms of n}
Start by figuring our $ S_n $. Write out some terms from our sum first,
\[
\sum_{ n=1 } ^{ \infty } \frac{ 1 }{ n^2 } =1+\frac{ 1 }{ 2^2 } +\frac{ 1 }{ 3^2 } +\frac{ 1 }{ 4^2 } 
.\] 
Let's start by finding the upper bound $ R_n < \int_{ n }^{ \infty } f\left( x \right) dx$ as $ \frac{ 1 }{ n }  $ from our p theorem. 

\paragraph{Part b) find how many terms are needed to ensure that the remainder ($ error $) is less than $ 10^{ -3 } $. \\}
Start by writing our inequality to be $ \frac{ 1 }{ n } < 10^{ -3 } $ or $ 10^{ 3 }<n $. Now we need to add one to our number to find that n would have to be at least $ 1001 $ terms. 

\paragraph{Find the lower and upper bounds ($ L_n $ and $ U_n $ respectively) on the exact value of the series\\}
Start by letting $ n=1001 $ and plug into our inequality $ L_n \le \sum_{ n=1 } ^{ \infty } a_n \le U_n$ 
\[
S_{ 1001 }\int_{ 1002 }^{ \infty } \frac{ 1 }{ x^2 } dx\le S \le \int_{ 1001 }^{ \infty } \frac{ 1 }{ x^2 } dx + S_{ 1001 }
.\] 
Computing this out we find,
\[
	S_{ 1001} + \frac{ 1002^{ 1-2 } }{ 2-1 } \le S \le \frac{ 1001^{ 1-2 } }{ 2-1 } + S_{ 1001 }
.\] 

\paragraph{Part D) find an interval in which the value of the series must lie if you approximate it using ten terms of the series. \\}
Start with our value of n being 10. Now write our inequality to be $ L_{ 10 } \le S\le U_{ 10 } $. So,
\[
S_{ 10 }+\int_{ 11 }^{ \infty } \frac{ 1 }{ x^2 } dx\le S\le S_{ 10 }+\int_{ 10 }^{ \infty } \frac{ 1 }{ x^2 } dx
.\] 
\[
S_{ 10 }+\frac{ 11^{ 1-2 } }{ 2-1 }\le S\le S_{ 10 }+ \frac{ 10^{ 1-2 } }{ 2-1 }
.\] 
Plugging this into a calculator gives us $ S_n = 1.549767731 $, so our inequality is $ 1.640676822< S < 1.649767731$. Which is a good estimate. Given our actual answer to be $ \frac{ \pi^2 }{ 6 }  $, we find that the estimated value of $ 1.644934067 $ is in our range. 
\end{document}
