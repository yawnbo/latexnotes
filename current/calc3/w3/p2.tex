\section{01/23/25 Day 2 w3}%
\label{sec:01/23/25 Final notes on 11.3}

\subsection*{Theorem}%
\label{sub:Theorem}
\paragraph{The p-series of $ \sum_{ i=1 } ^{ \infty } \frac{ 1 }{ i^{ p } }  $ converges for p>1. This is basically exactly the same as p integrals and is proved using them either way.}

\subsection{Section 11.4 The comparison test}%
\label{sub:Section 11.4 The comparison test}
\paragraph{Preamble:}
If we let $ a_n,b_n >0 $, then,
\paragraph{(i)} If $ \sum_{  } ^{  } b_n $ converges and $ a_n \le b_n $, then $ \sum_{  } ^{  } a_n $ also converges. 
\paragraph{(ii)} If $ \sum_{  } ^{ } b_n $ diverges and $ b_n \le a_n $, then $ \sum_{  } ^{  } a_n $ also diverges. \\
This can be used in the same way that comparisons can be proved in integrals. This is done by just finding a simpler series to compare with because we will only need this for very complex series that cannot be proven using our other known methods. 

\paragraph{Ex.}
\[
\sum_{  } ^{  } \frac{ 1 }{ n^2+6n+13 } 
.\] 
For this we can use integration using the integral test, but integration would be technical and not the easiest thing ever. (completing the square and trig sub), so instead we can use our comparison test. Start with an assumption. This can be done by looking at our function as a function with dominance applied to it. Using dominance we get $ \frac{ 1 }{ n^2+6n+13 } \approx \frac{ 1 }{ n^2 }  $ and because $ p>1 $ this should converge. Knowing this we should find a series that is LARGER to compare with. (because if we have a larger function then we know everything less will converge) So we start by removing some terms from our original summation. \\

Removing terms from the denominator we get
\[
\sum_{  } ^{  } \frac{ 1 }{ n^2+6n+13 } < \sum_{  } ^{  } \frac{ 1 }{ n^2+6n } <\sum_{  } ^{  } \frac{ 1 }{ n^2 } 
.\] 

This shows us the method of proving that our function is smaller and now that it's at our formula we can stop because it's proven using our integrals from calc2. Again as a last note, if we are trying to prove convergence, we need to find a larger and simpler function, but if we are trying to prove divergence we find a smaller function that is also simpler.

\paragraph{Ex. Given}
\[
\sum_{ n=1 } ^{ \infty } \frac{ n }{ n^2-\cos^{ 2 } \left( n \right)  } 
.\] 
\paragraph{Prove divergence or convergence. \\}

Again, we first make an assumption, let's look at our function and notice that cos will just bounce between 1 and -1, so we can ignore it. Now using dominance we find that our function will approximate to $ \sum_{  } ^{  } \frac{ 1 }{ n }  $ which is a harmonic series and divergent. This is also a p value of 1 so, we know it won't work. Now that we guessed divergence, we know we want to find something that is smaller and simpler. We start by reconstructing our function with cos and create an inequality.

\begin{gather*}
0<\cos^{ 2 } \left( n \right) <1\\
0\ge -1\cos^{ 2 } \left( n \right) \ge -1 \\
n^2\ge n^2-\cos^{ 2 } \left( n \right) \ge n^2-1 \\
\frac{ n }{ n^2 } \le \frac{ n }{ n^2-\cos^{ 2 } \left( n \right)  } \le \frac{ n }{ n^2-1 } \\
\sum_{ n=1 } ^{ \infty } \frac{ n }{ n^2 } \le \sum_{ n=1 } ^{ \infty }  \frac{ n }{ n^2-\cos^{ 2 } \left( n \right)  } \le \sum_{ n=1 } ^{ \infty } \frac{ n }{ n^2-1 } \\
\end{gather*}
This technique is called construction and can be done for proving all our comparisons but may be more complex than other ways. 

\paragraph{EX.}
\[
\sum_{ n=1 } ^{ \infty } \frac{ n^3+3 }{ n^{ 5 }+6 }
.\] 
\paragraph{Prove convergence or divergence. \\}
This will behave similar to $ \sum_{ n=1 } ^{ \infty } \frac{ n^3 }{ n^{ 5 } } $ which will converge because $ p>1 $. Now we know we want to find something bigger, so lets start by throwing away some terms. 

\[
<\sum_{ n=1 } ^{ \infty } \frac{ n^3+3 }{ n^{ 5 } }
.\] 
Given the above sum we can just split the sum into two fractions and prove it like that
\[
\sum_{ n=1 } ^{ \infty } \frac{ n^3 }{ n^{ 5 } } +\sum_{ n=1 } ^{ \infty } \frac{ 3 }{ n^{ 5 } } \implies \sum_{ n=1 } ^{ \infty } \frac{ 1 }{ n^2 } +3\sum_{ n=1 } ^{ \infty } \frac{ 1 }{ n^{ 5 } } 
.\] 
Because both of our sums converge, we know that our actual sum will also converge. (the above converges based on p values and can be cut down to the theorem from before that a convergent sum + a convergent sum will also be convergent) 
\subsection*{Limit comparison test}%
\label{sub:Limit comparison test}
Let $ a_n, b_n > 0 $,
\paragraph{(i)}
If $ 0<\lim_{ n \to \infty} \frac{ a_n }{ b_n }<\infty  $ then either, both $ \sum_{  } ^{  } a_n, \sum_{  } ^{  } b_n $ converge, or both diverge. It cannot be interchanged. This means that you cannot have $ a_n $ be convergent and $ b_n $ be divergent. 

\paragraph{Ex.}
\[
\sum_{ k=1 } ^{ \infty } \frac{ 4k^2+k+2 }{ 3k^3+9 }
.\] 
\paragraph{Prove convergence or divergence.\\}
In a case like this normal comparison would be difficult so let's use our limit comparison test. First make an assumption using dominance, so, $ a_n \approx \frac{ 4k^2 }{ 3k^3 } \implies \frac{ 4 }{ 3 } \sum_{  } ^{  } \frac{ 1 }{ k } $. Now that we see that we have a harmonic, we can use $ \frac{ 4 }{ 3k }  $ as our $ b_n $. (This would also work with its reciprocol so $ \lim_{ n \to \infty} \frac{ b_n }{ a_n } )$. 

So let's take our limit,
\[
\lim_{ n \to \infty} \frac{ a_n }{ b_n }=\lim_{ n \to \infty} \frac{ 4k^2+k+2 }{ 3k^3+9 }\cdot \frac{ 3 }{ 4 } \cdot k
.\] 
Distributing our terms we will get a $ 12k^3 $ function on the top and bottom so our resulting limit will be finite by dominance. Because $ 0<\lim_{ n \to \infty} \frac{ a_n }{ b_n }<\infty $ is proven true, we know that we have either convergence or divergence for both. And because we already showed that $ a_n $ diverges by dominance to the harmonic, we know that $ b_n $ will also be divergent and our whole function will be divergent. 
