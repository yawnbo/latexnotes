\section{01/17}%
\label{sec:01/17}

\subsection{Common mistakes on the quiz}%
\label{sub:Common mistakes on the quiz}
\paragraph{Question 1}
Some people are showing that the absolute value of the limit is less than 1, but we need to show that the limit itself is less than 1. One thing that you can do is use subsequences such as doing an even or odd n to prove it.

\paragraph{NOTE}
Theorems will be given on the exams and do not need to be remembered. 

\paragraph{Question 5 part 1}
Show that $ \ln^{  } \left( b_n \right) = \frac{ 1 }{ n } \sum_{ k=1 } ^{ n } \ln^{  } \left( \frac{ k }{ n }  \right) $ \\

Specifically, during the process of simplifying $ \frac{ 1 }{ n } \ln^{  } \left( n! \right) -\ln^{  } \left( n \right) $ gets messed up when trying to combine the terms into a single sum. This needs to be proven using the sum of $ \ln^{  } \left( n \right)  $.


\subsection{P integrals}%
\label{sub:P integrals}
\[
\int_{ a }^{ \infty } \frac{ dx }{ x^{ p } } \text{ or } \int_{ 0 }^{ 1 } \frac{ dx }{ x^{ p } }
.\] 
where
\begin{align*}
\int_{ a }^{ \infty } \frac{ dx }{ x^{ p } } \text{ converges if p > 1 to } \frac{ a^{ 1-p } }{ p_1 }\\
\int_{ 0 }^{ 1 } \frac{ dx }{ x^{ p } } \text{ converges if p < 1 to } \frac{ 1 }{ 1-p }
.\end{align*}

\paragraph{Ex.}
\[
\int_{ 9 }^{ \infty } \frac{ dx }{ x^{ 10 } } =\frac{ n_1-10 }{ 10-1 }=\frac{ n^{ -9 } }{ 9 } = \frac{ 1 }{ 9n^{ 9 } } 
.\] 

\paragraph{Ex.}
\[
\int_{ n+1 }^{ \infty } \frac{ dx }{ x^{ 10 } } = \frac{ \left( n+1 \right) ^{ 1-10 } }{ 10-1 }=\frac{ \left( n+1 \right) ^{ -9 } }{ 9 }=\frac{ 1 }{ 9\left( n+1 \right) ^{ 9 } } 
.\] 
This is the only type that will be used in this class and we will never(?) see the 0 to 1

\section{11.3 The integral Test}%
\label{sec:11.3 The integral Test}

Prof gave a paper worksheet that showed
\[
\sum_{ k=2 } ^{ n } a_k<\int_{ 1 }^{ n } f\left( x \right) dx<\sum_{ k=1 } ^{ n-1 } a_k
.\] 
When we assume $ f\left( x \right)  $ is monatonic (how do you spell it) decreasing over $ \left( 0,\infty \right)  $ or $ f''\left( x \right) >0 $ we look at this because it will approach 0 and we may have a convergent sum, or $ \lim_{ x \to \infty} f\left( x \right) =0 $. This is shown in some sketches that the integral will be the absolute area and our sum on the left will underestimate while the right one will overestimate. \\

With this we show that (1) if our integral is finite then the sum on the left will also be finite. This is just basic comparison, because if we have a bigger function, then anything under it should be finite. This also goes if we show the right sum to be finite, then our integral and other sum will also be finite. \\

We also have (2) where the sum $ \sum_{ k=1 } ^{ \infty } a_k $ converges to a value S, then given $ \int_{ 1 }^{ n } f\left( x \right) dx <\sum_{ k=1 } ^{ n-1 } a_k<\sum_{ k=1 } ^{ \infty } a_k=S<\infty$. This is when the sequence $ \left\{ \int_{ 1 }^{ n } f\left( x \right) dx \right\}  $ is monotonic increasing (b/c f(x) > 0) and bounded above by S. Thus $ \int_{ 1 }^{ \infty } f\left( x \right) dx\le S $.

\paragraph{Ex. The harmonic series}
\[
\sum_{ n=1 } ^{ \infty } \frac{ 1 }{ n } =\infty
.\] 
This can be compared to our improper integral
\[
\int_{ 1 }^{ \infty } \frac{ 1 }{ x } dx= \text{ finite because p=1 }
.\] 

\paragraph{Ex. Determine conv. or div.}
\[
\sum_{ k=1 } ^{ \infty } \frac{ k }{ k^2+1 } 
.\] 
Remembering our theorem of that if $ \sum_{  } ^{  } a_k $ converges then $ a_k\to 0 $, we can use the converse and see what $ \int_{ 1 }^{ \infty } \frac{ x }{ x^2+1 } dx $ is. Before using this we should make sure that we match our requirements, (1) we need our function to be monotonic decreasing and (2) that $ \lim_{ x \to \infty} f\left( x \right) =0 $. \\

Our second req. is clearly passed using dominance theory and isn't a problem so now we just check if our function is monotonic decreasing. The easiest way would just be to derive our top,
\[
\left( x^2+1 \right) \left( 1 \right) -\left( x \right) \left( 2x \right) \to -x^2+1<0
.\] 

To find that we have a decreasing function as it's less than 0 and, now we can just use u sub on the integral to find what our p value is,
\begin{align*}
u&= x^2+1 \\
du&= 2xdx \\
\to \frac{ du }{ 2 } &= xdx \\
\int_{ 2 }^{ \infty } \frac{ du }{ 2u } &= \infty
.\end{align*}

\paragraph{Ex.}
\[
\sum_{ k=3 } ^{ \infty } \frac{ 1 }{ \sqrt{ 2k-5} } 
.\] 
Rewrite using integral,
\[
\int_{ 3 }^{ \infty } \frac{ 1 }{ \sqrt{ 2x-5} } dx
.\] 
and we can see that this clearly goes to 0 with dominance so our second condition is met aand we check our first,
\[
f'=d\left( 2x-5 \right) ^{ -\frac{ 1 }{ 2 }  }=-\frac{ 1 }{ 2 } \left( 2x-5 \right) ^{ -\frac{ 3 }{ 2 }  }\left( 2 \right) <0
.\] 
Which we find to be true and that our function is decreasing. This lets us directly take the integral using u sub now,
\begin{gather*}
u=2x-5 \\
du = 2dx
\end{gather*}
\[
\int_{ 3 }^{ \infty } \frac{ dx }{ \sqrt{ 2x-5} }=\int_{ 1 }^{ \infty } \frac{ du }{ 2\sqrt{ u} } =\infty
.\] 
This also goes to infinity b/c p=1/2.

\paragraph{Ex.}
\[
\sum_{ k=0 } ^{ \infty } \frac{ 1 }{ k^2+4 } 
.\] 
Rewrite using integral,
\[
\int_{ - }^{ \infty } \frac{ dx }{ x^2+4 } 
.\] 
Again, it's obvious this goes to 0, so we take the d/dx
\[
f'=-\frac{ 2x }{ \left( x^2+4 \right) ^2 } <0
.\] 
Which matches our other condition and we can integrate.

\[
\int_{ 0 }^{ \infty } \frac{ dx }{ x^2+4 } 
.\] 
This requires us to use trig sub so let
\begin{gather*}
x=2\tan^{  } \left( \theta \right) \\
dx=2\sec^{ 2 } \left( \theta \right) d\theta
\end{gather*}
so,
\[
\int_{  }^{ } \frac{ 2\sec^{ 2 } \left( \theta \right) d\theta }{ 4\sec^{ 2 } \left( \theta \right)  }
.\] 

Where we can use $ \theta = \arctan^{  }\left( \frac{ x }{ 2 }  \right) $ to change our limits to $ \int_{ 0 }^{ \frac{ \pi }{ 2 }  }  $, so
\[
\frac{ 1 }{ 2 } \theta \bigg|_{ 0 }^{ \frac{ \pi }{ 2 }  } =\frac{ \pi }{ 4 }
.\] 
Now because this integral is a finite value we find that our sum will converge. However this won't be the limit of the sum and this will be talked about on monday i think.
