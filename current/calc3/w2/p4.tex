\section{01/16}%
\label{sec:01/16}
\subsection{Examples from the worksheet}%
\label{sub:Examples from the worksheet}

\subsubsection*{Ex 9b}
\[
\sum_{ n=0 } ^{ \infty } \frac{ 3\left( -2 \right) ^{ n }-5^{ n } }{ 8^{ n } }
.\] 
Because we see that our sum starts at 0 we should probably break it into $ \sum_{ m=0 } ^{ \infty } ar^{ m } $.

\[
\sum_{ n=0 } ^{ \infty } \left( \frac{ 3\left( -2 \right) ^{ n } }{ 8^{ n } }-\frac{ 5^{ n } }{ 8^{ n } } \right) 
.\] 
\[
=\sum_{ i=0 } ^{ \infty } 3\left( -\frac{ 1 }{ 4 }  \right) ^{ n }-\sum_{ i=0 } ^{ \infty } \left( \frac{ 5 }{ 8 }  \right) ^{ n } 
.\] 
Now that we have our 2 geometrics we can just find it to be
\[
\frac{ 3 }{ 1+\frac{ 1 }{ 4 }  } -\frac{ 1 }{ 1+\frac{ 5 }{ 8 }  } 
.\] 

\section{Harmonic series 11.2}%
\label{sec:Harmonic series 11.2}

These can be found as something like 
\[
1+\frac{ 1 }{ 2 } +\frac{ 1 }{ 3 } +\frac{ 1 }{ 4 } +\frac{ 1 }{ 5 } 
.\] 
These are generally infinite but we should still prove it. So,
\[
\text{ Prove } \sum_{ n=1 } ^{ \infty } \frac{ 1 }{ n } \text{ diverges }
.\] 
We can start by looking at our partial sums so lets start with
\begin{align*}
	S_2 &= 1+ \frac{ 1 }{ 2 } \\
	s_4 &= 1+ \frac{ 1 }{ 2 } +\frac{ 1 }{ 3 } +\frac{ 1 }{ 4 } \\
	S_8 &= 1+\frac{ 1 }{ 2 } +\frac{ 1 }{ 3 } +\frac{ 1 }{ 4 } +\frac{ 1 }{ 5 } +\frac{ 1 }{ 6 } +\frac{ 1 }{ 7 } +\frac{ 1 }{ 8 } 
\end{align*}
So we can rewrite some of our terms to find what value they are larger than. 
\begin{align*}
S_2 &= \frac{ 1 }{ 4 } +\frac{ 1 }{ 4 }  \\
S_4 &= 1+\frac{ 2 }{ 2 }  \\
S_8 &= 1+\frac{ 3 }{ 2 }  \\
.\end{align*}
Because we have a pattern of $ S_{ 2^{ n } }>1+\frac{ n }{ 2 }  $ we can start to find our limit. So,
\[
\lim_{ n \to \infty} S_{ 2^{ n } }=\sum_{ i=1 } ^{ n } \frac{ 1 }{ n } 
.\] 
Becuase our $ \lim_{ n \to \infty} 1+\frac{ n }{ 2 } = \infty $ we say that our sum is infinite because we found a smaller function that diverges which proves that everything above it will also diverge. 

\subsection{Theorem}%
\label{sub:Theorem}
If $ \sum_{  } ^{  } a_n $ converges, then $ a_n \to 0 $. However, this can be misunderstood with the converse which is if $ a_n \to 0$, then the $ \sum_{  } ^{  } a_n $ converges. This is wrong because this wouldn't be true for the general case. One such case is the harmonic series $ \sum_{ n=1 } ^{ \infty } \frac{ 1 }{ n }  $. Our $ a_n $ goes to 0, but the sum of this will go to infinity and diverge.

Instead we can write the contrapositive of the theorem as, if $ \lim_{ n \to \infty} a_n \neq  0 $ then $ \sum_{  } ^{  } a_n $ diverges. 

\paragraph{Ex.}%
\label{par:Ex.}
\[
\text{ Prove }\sum_{  } ^{  } 2^{ n }\text{ diverges }
.\] 
For this it's simply enough to show that $ \lim_{ n \to \infty} 2^{ n }=\infty $ which counters our orignal theorem. The part where people go wrong is thinking that if we prove $ \lim_{ n \to \infty} a_n = 0 $ then our sum must converge. However, this is not enough and must have other theorems along with it to properly prove what we are trying to show. 
\paragraph{Ex.}
\[
\text{ Determine divergence or convergence on }\frac{ 1 }{ 3 } +\frac{ 1 }{ 6 } +\frac{ 1 }{ 9 } +\frac{ 1 }{ 12 } +\frac{ 1 }{ 15 } +\ldots
.\]
This can be done by writing our sum as 
\[
\sum_{ n=1 } ^{ \infty } \frac{ 1 }{ 3n } = \frac{ 1 }{ 3 } \sum_{ n=1 } ^{ \infty } \frac{ 1 }{ n } 
.\] 
Which is just our harmonic series so this would diverge.

\subsection{Theorem}%
\label{sub:Theorem}
If $ \sum_{  } ^{  } a_n $ and $ \sum_{  } ^{  } b_n $ each converge then so does $ k\sum_{  } ^{  } a_n $ and $ \sum_{  } ^{  } \left(   a_n \pm b_n \right)$. 
\paragraph{Ex.}
This can be used for the following
\[
\text{ Show }\sum_{ n=1 } ^{ \infty } \left( \frac{ 1 }{ e^{ n } } +\frac{ 1 }{ n\left( n+1 \right)  }  \right) \text{ converges/diverges }
.\] 

So,
\[
\sum_{ n=1 } ^{ \infty } \frac{ 1 }{ e^{ n } } +\sum_{ n=1 } ^{ \infty } \frac{ 1 }{ n\left( n+1 \right)  } 
.\] 
We can look at our first sum and find that it will be a geometric series that can be written as
\[
\sum_{ n=1 } ^{ \infty } \left( \frac{ 1 }{ e }  \right) ^{ n }
.\] 
Now because our base of $ \frac{ 1 }{ e }  $ is less than 1, we find that this is convergent. For our second sum we can find this as a telescoping series and can find that this is convergent. Because both functions are proven convergent we know that the original sum is also convergent based on our new theorem.

\paragraph{Tomorrow will be Improper integrals again from calc 2. Pull out the notes from calc 2}

