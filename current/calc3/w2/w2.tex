\documentclass[a4paper]{article}

\usepackage[utf8]{inputenc}
\usepackage[T1]{fontenc}
\usepackage{textcomp}
\usepackage[english]{babel}
\usepackage{amsmath, amssymb}

\title{Series}
\author{yawnbo}
\date{\today}

\pdfsuppresswarningpagegroup=1

\begin{document}
\maketitle
\section{Limits of recursive sequences}%
\label{sec:Limits of recursive sequences}

\paragraph{Ex.}%
\label{par:Ex.}

\[
a_1 = \sqrt{ 3}, a_2 = \sqrt{ 3 \cdot \sqrt{ 3}}, a_3 = \sqrt{ 3 \cdot \sqrt{ 3 \cdot \sqrt{ 3}}}, \ldots
.\] 

\paragraph{Definition\\}

If we let $ a_1 = \sqrt{ 3} $ Then for $ n>1 $: $ a_n = \sqrt{ 3\cdot a_{ n-1 }} $.
This creates our nesting and we can begin to solve for our limit. 

\paragraph{First \\}
We want to prove that $ a_n $ has a limit. For this, we need to prove that the sequence is monotonic (increase/decrease) and that we are bounded somewhere along the sequence. When these two parts are proven together, we know that we have convergence at some point. \\

For this we can just write out a few terms, so,
\[
a_1 = \sqrt{ 3} \approx 1.77, a_2 = \sqrt{ 3\cdot \sqrt{ 3}} \approx 2.28, a_3 = \sqrt{ 3\cdot \sqrt{ 3\cdot \sqrt{ 3}}} \approx 2.62, a_{ 4 } \approx 2.8 , a_{ 5 } \approx 2.90
.\] 
This differs from how we would find the limit with a function because we can't take a solid $ \frac{ d }{ dx }  $ so we can just do this for now to find an approximation.\\

Due to the above, we can prove that we have an increasing sequence so we can define some upper limit/bound to prove. If we write out a few more terms, $ a_6 \approx 2.95 a_7 \approx 2.97 $ which seems to approach 3 which we can label as our upper bound. \\

This is all that is needed to prove that a limit EXISTS but we still need to find the limit. Let's assume that our limit is $ L $. Now we should know that,
\[
\lim_{ n \to \infty} a_n = L
.\] 
\[
\lim_{ n \to \infty} a_{ n-1 }= L 
.\] 
\newpage
Knowing that these should be equal, we can write out and solve for our limits
\[
a_n = \sqrt{ 3\cdot a_{ n-1 }}
.\] 
\[
\lim_{ n \to \infty} a_n = \sqrt{ 3\cdot \lim_{ n \to \infty} \cdot a_{ n-1 }}
.\] 
\[
L = \sqrt{ 3\cdot L}
.\] 
\[
L^2=3L \to l^2-3l = 0 \to L(L-3) = 0
.\] 
Which shows that $ L = 0 \text{ or }3 $, but we ignore the 0 because our sequence is non-zero. 

\section{Geometric sequences}%
\label{sec:Geometric sequences}
\paragraph{Recall}
that an arithmetic sequence is one that increases in the pattern of $ a_1 + m = a_2 $ where we add by a factor of m while a geometric sequence is one that increases in the pattern of $ a_1 \cdot r, a_1 \cdot r^2, a_1 \cdot r^{ 3 }$ where we instead multiply by a factor of r. Basically just common difference vs common ration. 

\paragraph{Convergence of $ \left\{ r^{ n } \right\}_{ n=1 }^{ \infty }  $.}
If we were to take the limit ($ \lim_{ n \to \infty} r^{ n }$), then we have a few cases to look at, first, if $ r=1 $ then our limit will just approach 1 because our sequence will look like $ 1^{ 1 }, 1^{ 2 },1^{ 3 },1^{ 4 }\ldots $ which will forever be one. At the case of $ r=-1 $, we will have a sequence that looks like $ -1^{ 1 }, -1^{ 2 }, -1^{ 3 }, -1^{ 4 }\ldots $ which will oscillate between 1 and -1 giving us a divergent sequence. If we instead have $ -1<r<1 $ then we will have a convergent sequence that will approach 0. If we have $ r>1 $ then we will have a divergent sequence that will approach positive $ \infty $ while for $ r<1 $ we would have $ \pm \infty $, but for each of these cases, the limits don't exist. 

\paragraph{Summary}
To end, $ r^{ n } \text{ converges for } r\epsilon\left( -1,1 \right] $.

\section{Sec 11.2 Series}%
\label{sec:Sec 11.2 Series}

\paragraph{Definition\\}
For $ \left\{ a_n \right\} ^{ \infty }_{ n=1 } $, the sum of its terms $ a_1 + a_2 + a_3 \ldots = \sum_{ n=1 } ^{ \infty } a_n$ is referred to as an infinite series or just a series. \\

Our goal with this section is to find out when our sum will equal a finite, infinite, or indeterminate value. Basically just convergence and divergence with fancy terms now. \\

An important thing to note is that,
\[
\sum_{ i=1 } ^{ n } a_{ i } \text{ is always finite}
.\] 

\paragraph{Ex.}
\[
\sum_{ i=1 } ^{ n } i= 1+2+3+4+5\ldots + \left( n-1 \right) + n = \frac{ n\left( n+1 \right)  }{ 2 }
.\] 

\newpage
\paragraph{Definition again\\}
For $ \sum_{ i=1 } ^{ \infty } a_i $, 
\[
\sum_{ i=1 } ^{ n } a_i = S_n \text{ which is the nth partial sum of }\sum_{ n=1 } ^{ \infty } a_n
.\] 

\paragraph{Ex. we can have a series like,}
\begin{align*}
&\sum_{ i=1 } ^{ n } a_i \\
S_1 &= \sum_{ i=1 } ^{ 1 } a_i = a_1 \\
S_2 &= \sum_{ i=1 } ^{ 2 } a_i = a_1 + a_2 \\
S_3 &= \sum_{ i=1 } ^{ 3 } a_i = a_1 + a_2 + a_3 \\
\ldots
S_n &= \sum_{ i=1 } ^{ n } a_i = a_1 + a_2 + a_3 + \ldots + a_n
\end{align*}
which is really just a sequence of partial sums. Going back to our definition we know that $ \lim_{ n \to \infty} \left\{ S_n \right\} =\sum_{ i=1 } ^{ \infty } a_i $. But if our limit DNE then our sum will diverge instead. 

\subsection{Going back to Geometric Series}%
\label{sub:Going back to Geometric Series}

This is a case where we would use our sum.
\[
\sum_{ n=1 } ^{ \infty } ar^{ n-1 }=a + ar^{ 1 } + ar^{ 2 } + ar^{ 3 } + \ldots = a\left( 1+r+r^{ 2 }+r^{ 3 }+\ldots \right)
.\] 
We can find our common ratio to be $ \frac{ ar^3 }{ ar^2 }=r $.\\

Now lets say we want to find when does $ \sum_{  } ^{  } ar^{ n-1 } $ converge? Lets try with some cases,
\begin{gather*}
r=1 \\
\sum_{ i=1 } ^{ \infty } ar^{ i-1 } = a + a + a + a + \ldots \\
\sum_{ i=1 } ^{ n } ar^{ i-1 }= a + a + a \ldots + a = na = \left\{ S_{ i } \right\}^{ \infty }_{ i=1 } 
\end{gather*}

Now taking the limit, 
\[
\lim_{ i \to \infty} \left\{ S_{ i } \right\} =\lim_{ n \to \infty} \sum_{ i=1 } ^{ \infty } a\left( 1 \right) = \lim_{ n \to \infty} na = \pm \infty
.\] 
We find that this diverges, unlike $ \left\{ a\left( 1 \right) ^{ n } \right\}  $ which converges. So adding our sum makes our series diverge. \\

Now lets try,
\begin{gather*}
r\neq 1: -1<r<1\\
\sum_{ i=1 } ^{ n } ar^{ i-1 }=a+ar+ar^2+ \ldots + ar^{ n-1 }= S_{ n }
\end{gather*}
But this doesn't get us anywhere so lets multiply the whole thing by r,
\[
\sum_{ i=1 } ^{ n } ar^{ i-1 }= ar + ar^2+ ar^3+ \ldots + ar^{ n }= S_n \cdot r
.\] 
Lets call the equation before introducing the r $ eq_1 $ and the second one $ eq_2 $. Subtracting $ eq_1 $ from $ eq_2 $ we get,
\[
a-ar^{ n } = S_n - S_n r = S_n\left( 1-r \right) \to
.\] 
\[
S_n=\frac{ a-ar^{ n } }{ 1-r } = \frac{ a\left( 1-r^{ n } \right)  }{ 1-r }
.\] 
Which proves that 
\[
\sum_{ i=1 } ^{ n } ar^{ n-1 }=S_n = \frac{ a\left( 1-r^{ n } \right)  }{ 1-r }
.\] 
\paragraph{Example where this is useful}
\[
\sum_{ i=1 } ^{ 100 } 2\left( \frac{ 3 }{ 7 }  \right) ^{ i-1 }
.\] 
With our formula we can plug in all our values to find that
\[
	\frac{ 2\left( 1-\frac{ 3 }{ 7 }  \right)^{ 100 } }{ 1-\frac{ 3 }{ 7 }  }
.\] 	
\section*{Tests cont.}%
\label{sec:Tests}
\paragraph{Note that quiz 3 is open by tomorrow and due in a week I assume.}

\paragraph{Ex.}
Apply the ratio test to 
\[
\sum_{ n=0 } ^{ \infty } \left( -1 \right) ^{ n }\frac{ \sqrt{ n} }{ n+1 }
.\] 
Note that our sequence can be called $ a_n $. Now if we sub n for n+1 we can now take the limit,
\[
\lim_{ n \to \infty} \left\| \frac{ a_{ n+1 } }{ a_n } \right\| = \lim_{ n \to \infty} \frac{ \sqrt{ n+1} }{ n+2 }\cdot \frac{ n+1 }{ \sqrt{ n} }= \sqrt{ \frac{ n+1 }{ n }} \cdot \frac{ n+1 }{ n+2 } = 1\cdot 1 = 1
.\] 
Which makes our test inconclusive. This just means that we have to use another test to determine convergence or divergence. \\
Instead looking at our sequence, we can use AST to find that it's monotonically decreasing and that the limit is 0. This means that the series converges by AST. So because $ \frac{ \sqrt{ n} }{ n+1 } $ is monotonically decreasing and $ \lim_{ n \to \infty} \frac{ \sqrt{ n} }{ n+1 } = 0 $, we can say that the series converges by AST.

\paragraph{What if we also wanted to find if the absolute also converges?}
\[
\sum_{ } ^{  } |a_n| = \sum_{  } ^{  } \frac{ \sqrt{ n} }{ n+1 }
.\] 
Comparing to a smaller function $ \sum_{  } ^{  } \frac{ 1 }{ n+1 } $ shows that the series converges by p-test. This means that the series conditionally converges.
\subsection*{11.5.4}%
\label{sub:11.5.4}
\paragraph{The root test }
(i) If $ \lim_{ n \to \infty} \sqrt[ n ]{ \left\| a_n \right\| } <1 $, then $ \sum_{  } ^{  } \left\| a_n \right\| $ converges \\
(ii) If $ \lim_{ n \to \infty} \sqrt[ n ]{ \left\| a_n \right\| } >1 $, then $ \sum_{  } ^{  } a_n $ diverges \\
(iii) If $ \lim_{ n \to \infty} \sqrt[ n ]{ \left\| a_n \right\| } =1 $, then inconclusive. \\ \\ 
\paragraph{Ex.}
\[
\sum_{ n=1 } ^{ \infty } \left( \frac{ n+1 }{ 2n } \right) ^{ n }
.\] 
We commonly use this test when we have something complex to the power of n, so,
\[
\sqrt[ n ]{ \left\| a_n \right\| } =\sqrt[ n ]{ \frac{ n+1 }{ 2n }^{ n } } = \lim_{ n \to \infty} \frac{ n+1 }{ 2n }=\frac{ 1 }{ 2 } < 1
.\] 
Which proves that our sum $ \sum_{  } ^{  } a_n $ converges because its less than 1.

\paragraph{Ex.}
\[
\sum_{ k=1 } ^{ \infty } \left( 1+\frac{ 3 }{ k }  \right) ^{ k^2 }
.\] 
Start by applying the root,
\[
\sqrt[ k ]{ \left\| a_n \right\| } = \sqrt[ k ]{ \left( 1+\frac{ 3 }{ k }  \right) ^{ k^2 } } = \lim_{ n \to \infty}  \left( 1+\frac{ 3 }{ k }  \right) ^{ k }
.\] 
Solving this at the above point makes it indecisive. So we can use the ratio test to find that the series converges. Let our limit equal L, then,
\[
k \ln^{  } \left( 1+\frac{ 3 }{ k }  \right) = \ln^{  } \left( L \right) 
.\] 
Now as $ k \to \infty $ we get an indeterminate form so we can use L'Hopital's rule to get,
\[
k\ln^{  } \left( 1+\frac{ 3 }{ k }  \right) = \frac{ \frac{ 1 }{ 1+\frac{ 3 }{ k }  }\cdot \left( -\frac{ 3 }{ k^2 }  \right)  }{ -\frac{ 1 }{ k^2 }  } = \frac{ 3 }{ 1+\frac{ 3 }{ k }  } \to 3 = \ln^{  } \left( L \right) 
.\] 
So $ L = e^{ 3 } $, which means that the series diverges because $ e^{ 3  }>1 $.
\paragraph{Everything above is done up to section 11.6 and 11.7 is just review on how to actually use these tests}

\paragraph{Example list to do if wanted (compare first to second to prove divergence or convergence)}
\begin{align*}
	1.& \sum_{  } ^{  } \frac{ 1 }{ 5^{ n } } , \sum_{  } ^{  } \frac{ 1 }{ 5^{ n }+n } \\
2.&\sum_{  } ^{  } \frac{ \left( -1 \right) ^{ n } }{ n^{ \frac{ 3 }{ 2 }  } }, \sum_{  } ^{  } \frac{ 1 }{ n^{ \frac{ 3 }{ 2 }  } } \\
3.&\sum_{  } ^{  } \frac{ n }{ r^{ n } } , \sum_{  } ^{  } \frac{ 3^{ n } }{ n } \\
4.&\sum_{  } ^{  } \frac{ n+1 }{ n }, \sum_{  } ^{  } \left( -1 \right) ^{ n } \frac{ n+1 }{ n } \\
5.&\sum_{ n=1 } ^{  } \frac{ n }{ n^2+1 } , \sum_{  } ^{  } \left( \frac{ n }{ n^2+1 }  \right) ^{ n } \\
6.&\sum_{  } ^{  } \frac{ \ln^{  } \left( n \right)  }{ n }, \sum_{ n=10 } ^{  } \frac{ 1 }{ n\ln^{  } \left( n \right)  } \\
7.&\sum_{  } ^{  } \frac{ 1 }{ n+n! } , \sum_{  } ^{  } \left( \frac{ 1 }{ n } +\frac{ 1 }{ n! }  \right) \\
8.&\sum_{  } ^{  } \frac{ 1 }{ \sqrt{ n^2+1} } , \sum_{ } ^{  } \frac{ 1 }{ n\sqrt{ n^2+1} } \\
\end{align*}

\documentclass[a4paper]{article}

\usepackage[utf8]{inputenc}
\usepackage[T1]{fontenc}
\usepackage{textcomp}
\usepackage[english]{babel}
\usepackage{amsmath, amssymb}
\title{Review for exam 3}
\author{yawnbo}
\date{\today}

\maketitle

\pdfsuppresswarningpagegroup=1

\begin{document}
\paragraph{Question 5}
\[
\int_{}^{} e^{\sqrt[3]{x}}dx \text{ using u sub  }
.\] 
\[
u=x^{\frac{1}{3}} \text{ an } du=\frac{1}{3}x^{-\frac{2}{3}}dx
.\] 
Solving for dx gets 
\[
dx= \frac{3du}{x^{-\frac{2}{3}}}=3x^{\frac{2}{3}}du=3\left( x^{\frac{1}{3}} \right) ^2du=3u^2du
.\] 
Which can then be tabulated to
\[
3u^2e^{4}-6ue^{u}+6u+C
.\] 
and subbing back in x
\[
3x^{\frac{2}{3}}e^{\sqrt[3]{x}}-6^{\sqrt[3]{x}}e^{\sqrt[3]{x}}+6\sqrt[3]{x}+C
.\] 
\section{Comparison examples from yesterday}%
\label{sec:Comparison examples from yesterday}

\begin{align}
 &\int_{1}^{\infty} \frac{dx}{\sqrt{x}+e^{3x}} \\
 &\int_{0}^{0.5} \frac{dx}{x^{8}+x^2} \\
 &\int_{1}^{\infty} \frac{dx}{\sqrt{x^{5}+2}} \\
 &\int_{0}^{5} \frac{dx}{x^{\frac{1}{3}}+x^3} \\
 &\int_{0}^{\infty} \frac{dx}{\sqrt{x^{\frac{1}{3}+x^3}}} \\
 &\int_{1}^{\infty} \frac{dx}{x^{4}+e^{x}}\\
\end{align}
Generally guess which one we would assume we have, for one assume convergence, so we need somthing larger.
\newpage
\subsection{For number 5}%
\label{sub:For number 5}
\paragraph{This is a type 1 and two improper.}
\[
\int_{0}^{\infty} \frac{dx}{\sqrt{x^{\frac{1}{3}}+x^3}}
.\] 
becomes
\[
\int_{0}^{1} \frac{dx}{\sqrt{x^{\frac{1}{3}}+x^3}} + \int_{1}^{\infty} \frac{dx}{\sqrt{x^{\frac{1}{3}}+x^3}}
.\] 
By dominance theory we can look at the second integral of $x^3$ which converges by p theorem. 
Now we look at the first one to establish its divergence so we need a function thats bigger
\[
x^{\frac{1}{3}}+x^3 \le 2x^{\frac{1}{3}}
.\] 
We choose that number because when testing $x^3+x^3$ we find that just one of them is smaller than $x^{\frac{1}{3}}$ which means that if we are finding a larger function we should take our x to the power of a fraction.
Doing normal comparison we find that this converges instead and our analysis is wrong because we found a value that doesn't help us

Now we want to try something like $x^{-1}$ to get a value that is 

THiS ONE IS ON THE EXAM THIS NEEDS TO BE FIGURED OUT 
\[
\int_{0}^{\infty} \frac{dx}{\sqrt{x^{\frac{1}{3}}+x^3}}
.\] 
\end{document}

\section{01/16}%
\label{sec:01/16}
\subsection{Examples from the worksheet}%
\label{sub:Examples from the worksheet}

\subsubsection*{Ex 9b}
\[
\sum_{ n=0 } ^{ \infty } \frac{ 3\left( -2 \right) ^{ n }-5^{ n } }{ 8^{ n } }
.\] 
Because we see that our sum starts at 0 we should probably break it into $ \sum_{ m=0 } ^{ \infty } ar^{ m } $.

\[
\sum_{ n=0 } ^{ \infty } \left( \frac{ 3\left( -2 \right) ^{ n } }{ 8^{ n } }-\frac{ 5^{ n } }{ 8^{ n } } \right) 
.\] 
\[
=\sum_{ i=0 } ^{ \infty } 3\left( -\frac{ 1 }{ 4 }  \right) ^{ n }-\sum_{ i=0 } ^{ \infty } \left( \frac{ 5 }{ 8 }  \right) ^{ n } 
.\] 
Now that we have our 2 geometrics we can just find it to be
\[
\frac{ 3 }{ 1+\frac{ 1 }{ 4 }  } -\frac{ 1 }{ 1+\frac{ 5 }{ 8 }  } 
.\] 

\section{Harmonic series 11.2}%
\label{sec:Harmonic series 11.2}

These can be found as something like 
\[
1+\frac{ 1 }{ 2 } +\frac{ 1 }{ 3 } +\frac{ 1 }{ 4 } +\frac{ 1 }{ 5 } 
.\] 
These are generally infinite but we should still prove it. So,
\[
\text{ Prove } \sum_{ n=1 } ^{ \infty } \frac{ 1 }{ n } \text{ diverges }
.\] 
We can start by looking at our partial sums so lets start with
\begin{align*}
	S_2 &= 1+ \frac{ 1 }{ 2 } \\
	s_4 &= 1+ \frac{ 1 }{ 2 } +\frac{ 1 }{ 3 } +\frac{ 1 }{ 4 } \\
	S_8 &= 1+\frac{ 1 }{ 2 } +\frac{ 1 }{ 3 } +\frac{ 1 }{ 4 } +\frac{ 1 }{ 5 } +\frac{ 1 }{ 6 } +\frac{ 1 }{ 7 } +\frac{ 1 }{ 8 } 
\end{align*}
So we can rewrite some of our terms to find what value they are larger than. 
\begin{align*}
S_2 &= \frac{ 1 }{ 4 } +\frac{ 1 }{ 4 }  \\
S_4 &= 1+\frac{ 2 }{ 2 }  \\
S_8 &= 1+\frac{ 3 }{ 2 }  \\
.\end{align*}
Because we have a pattern of $ S_{ 2^{ n } }>1+\frac{ n }{ 2 }  $ we can start to find our limit. So,
\[
\lim_{ n \to \infty} S_{ 2^{ n } }=\sum_{ i=1 } ^{ n } \frac{ 1 }{ n } 
.\] 
Becuase our $ \lim_{ n \to \infty} 1+\frac{ n }{ 2 } = \infty $ we say that our sum is infinite because we found a smaller function that diverges which proves that everything above it will also diverge. 

\subsection{Theorem}%
\label{sub:Theorem}
If $ \sum_{  } ^{  } a_n $ converges, then $ a_n \to 0 $. However, this can be misunderstood with the converse which is if $ a_n \to 0$, then the $ \sum_{  } ^{  } a_n $ converges. This is wrong because this wouldn't be true for the general case. One such case is the harmonic series $ \sum_{ n=1 } ^{ \infty } \frac{ 1 }{ n }  $. Our $ a_n $ goes to 0, but the sum of this will go to infinity and diverge.

Instead we can write the contrapositive of the theorem as, if $ \lim_{ n \to \infty} a_n \neq  0 $ then $ \sum_{  } ^{  } a_n $ diverges. 

\paragraph{Ex.}%
\label{par:Ex.}
\[
\text{ Prove }\sum_{  } ^{  } 2^{ n }\text{ diverges }
.\] 
For this it's simply enough to show that $ \lim_{ n \to \infty} 2^{ n }=\infty $ which counters our orignal theorem. The part where people go wrong is thinking that if we prove $ \lim_{ n \to \infty} a_n = 0 $ then our sum must converge. However, this is not enough and must have other theorems along with it to properly prove what we are trying to show. 
\paragraph{Ex.}
\[
\text{ Determine divergence or convergence on }\frac{ 1 }{ 3 } +\frac{ 1 }{ 6 } +\frac{ 1 }{ 9 } +\frac{ 1 }{ 12 } +\frac{ 1 }{ 15 } +\ldots
.\]
This can be done by writing our sum as 
\[
\sum_{ n=1 } ^{ \infty } \frac{ 1 }{ 3n } = \frac{ 1 }{ 3 } \sum_{ n=1 } ^{ \infty } \frac{ 1 }{ n } 
.\] 
Which is just our harmonic series so this would diverge.

\subsection{Theorem}%
\label{sub:Theorem}
If $ \sum_{  } ^{  } a_n $ and $ \sum_{  } ^{  } b_n $ each converge then so does $ k\sum_{  } ^{  } a_n $ and $ \sum_{  } ^{  } \left(   a_n \pm b_n \right)$. 
\paragraph{Ex.}
This can be used for the following
\[
\text{ Show }\sum_{ n=1 } ^{ \infty } \left( \frac{ 1 }{ e^{ n } } +\frac{ 1 }{ n\left( n+1 \right)  }  \right) \text{ converges/diverges }
.\] 

So,
\[
\sum_{ n=1 } ^{ \infty } \frac{ 1 }{ e^{ n } } +\sum_{ n=1 } ^{ \infty } \frac{ 1 }{ n\left( n+1 \right)  } 
.\] 
We can look at our first sum and find that it will be a geometric series that can be written as
\[
\sum_{ n=1 } ^{ \infty } \left( \frac{ 1 }{ e }  \right) ^{ n }
.\] 
Now because our base of $ \frac{ 1 }{ e }  $ is less than 1, we find that this is convergent. For our second sum we can find this as a telescoping series and can find that this is convergent. Because both functions are proven convergent we know that the original sum is also convergent based on our new theorem.

\paragraph{Tomorrow will be Improper integrals again from calc 2. Pull out the notes from calc 2}


\section{01/17}%
\label{sec:01/17}

\subsection{Common mistakes on the quiz}%
\label{sub:Common mistakes on the quiz}
\paragraph{Question 1}
Some people are showing that the absolute value of the limit is less than 1, but we need to show that the limit itself is less than 1. One thing that you can do is use subsequences such as doing an even or odd n to prove it.

\paragraph{NOTE}
Theorems will be given on the exams and do not need to be remembered. 

\paragraph{Question 5 part 1}
Show that $ \ln^{  } \left( b_n \right) = \frac{ 1 }{ n } \sum_{ k=1 } ^{ n } \ln^{  } \left( \frac{ k }{ n }  \right) $ \\

Specifically, during the process of simplifying $ \frac{ 1 }{ n } \ln^{  } \left( n! \right) -\ln^{  } \left( n \right) $ gets messed up when trying to combine the terms into a single sum. This needs to be proven using the sum of $ \ln^{  } \left( n \right)  $.


\subsection{P integrals}%
\label{sub:P integrals}
\[
\int_{ a }^{ \infty } \frac{ dx }{ x^{ p } } \text{ or } \int_{ 0 }^{ 1 } \frac{ dx }{ x^{ p } }
.\] 
where
\begin{align*}
\int_{ a }^{ \infty } \frac{ dx }{ x^{ p } } \text{ converges if p > 1 to } \frac{ a^{ 1-p } }{ p_1 }\\
\int_{ 0 }^{ 1 } \frac{ dx }{ x^{ p } } \text{ converges if p < 1 to } \frac{ 1 }{ 1-p }
.\end{align*}

\paragraph{Ex.}
\[
\int_{ 9 }^{ \infty } \frac{ dx }{ x^{ 10 } } =\frac{ n_1-10 }{ 10-1 }=\frac{ n^{ -9 } }{ 9 } = \frac{ 1 }{ 9n^{ 9 } } 
.\] 

\paragraph{Ex.}
\[
\int_{ n+1 }^{ \infty } \frac{ dx }{ x^{ 10 } } = \frac{ \left( n+1 \right) ^{ 1-10 } }{ 10-1 }=\frac{ \left( n+1 \right) ^{ -9 } }{ 9 }=\frac{ 1 }{ 9\left( n+1 \right) ^{ 9 } } 
.\] 
This is the only type that will be used in this class and we will never(?) see the 0 to 1

\section{11.3 The integral Test}%
\label{sec:11.3 The integral Test}

Prof gave a paper worksheet that showed
\[
\sum_{ k=2 } ^{ n } a_k<\int_{ 1 }^{ n } f\left( x \right) dx<\sum_{ k=1 } ^{ n-1 } a_k
.\] 
When we assume $ f\left( x \right)  $ is monatonic (how do you spell it) decreasing over $ \left( 0,\infty \right)  $ or $ f''\left( x \right) >0 $ we look at this because it will approach 0 and we may have a convergent sum, or $ \lim_{ x \to \infty} f\left( x \right) =0 $. This is shown in some sketches that the integral will be the absolute area and our sum on the left will underestimate while the right one will overestimate. \\

With this we show that (1) if our integral is finite then the sum on the left will also be finite. This is just basic comparison, because if we have a bigger function, then anything under it should be finite. This also goes if we show the right sum to be finite, then our integral and other sum will also be finite. \\

We also have (2) where the sum $ \sum_{ k=1 } ^{ \infty } a_k $ converges to a value S, then given $ \int_{ 1 }^{ n } f\left( x \right) dx <\sum_{ k=1 } ^{ n-1 } a_k<\sum_{ k=1 } ^{ \infty } a_k=S<\infty$. This is when the sequence $ \left\{ \int_{ 1 }^{ n } f\left( x \right) dx \right\}  $ is monotonic increasing (b/c f(x) > 0) and bounded above by S. Thus $ \int_{ 1 }^{ \infty } f\left( x \right) dx\le S $.

\paragraph{Ex. The harmonic series}
\[
\sum_{ n=1 } ^{ \infty } \frac{ 1 }{ n } =\infty
.\] 
This can be compared to our improper integral
\[
\int_{ 1 }^{ \infty } \frac{ 1 }{ x } dx= \text{ finite because p=1 }
.\] 

\paragraph{Ex. Determine conv. or div.}
\[
\sum_{ k=1 } ^{ \infty } \frac{ k }{ k^2+1 } 
.\] 
Remembering our theorem of that if $ \sum_{  } ^{  } a_k $ converges then $ a_k\to 0 $, we can use the converse and see what $ \int_{ 1 }^{ \infty } \frac{ x }{ x^2+1 } dx $ is. Before using this we should make sure that we match our requirements, (1) we need our function to be monotonic decreasing and (2) that $ \lim_{ x \to \infty} f\left( x \right) =0 $. \\

Our second req. is clearly passed using dominance theory and isn't a problem so now we just check if our function is monotonic decreasing. The easiest way would just be to derive our top,
\[
\left( x^2+1 \right) \left( 1 \right) -\left( x \right) \left( 2x \right) \to -x^2+1<0
.\] 

To find that we have a decreasing function as it's less than 0 and, now we can just use u sub on the integral to find what our p value is,
\begin{align*}
u&= x^2+1 \\
du&= 2xdx \\
\to \frac{ du }{ 2 } &= xdx \\
\int_{ 2 }^{ \infty } \frac{ du }{ 2u } &= \infty
.\end{align*}

\paragraph{Ex.}
\[
\sum_{ k=3 } ^{ \infty } \frac{ 1 }{ \sqrt{ 2k-5} } 
.\] 
Rewrite using integral,
\[
\int_{ 3 }^{ \infty } \frac{ 1 }{ \sqrt{ 2x-5} } dx
.\] 
and we can see that this clearly goes to 0 with dominance so our second condition is met aand we check our first,
\[
f'=d\left( 2x-5 \right) ^{ -\frac{ 1 }{ 2 }  }=-\frac{ 1 }{ 2 } \left( 2x-5 \right) ^{ -\frac{ 3 }{ 2 }  }\left( 2 \right) <0
.\] 
Which we find to be true and that our function is decreasing. This lets us directly take the integral using u sub now,
\begin{gather*}
u=2x-5 \\
du = 2dx
\end{gather*}
\[
\int_{ 3 }^{ \infty } \frac{ dx }{ \sqrt{ 2x-5} }=\int_{ 1 }^{ \infty } \frac{ du }{ 2\sqrt{ u} } =\infty
.\] 
This also goes to infinity b/c p=1/2.

\paragraph{Ex.}
\[
\sum_{ k=0 } ^{ \infty } \frac{ 1 }{ k^2+4 } 
.\] 
Rewrite using integral,
\[
\int_{ - }^{ \infty } \frac{ dx }{ x^2+4 } 
.\] 
Again, it's obvious this goes to 0, so we take the d/dx
\[
f'=-\frac{ 2x }{ \left( x^2+4 \right) ^2 } <0
.\] 
Which matches our other condition and we can integrate.

\[
\int_{ 0 }^{ \infty } \frac{ dx }{ x^2+4 } 
.\] 
This requires us to use trig sub so let
\begin{gather*}
x=2\tan^{  } \left( \theta \right) \\
dx=2\sec^{ 2 } \left( \theta \right) d\theta
\end{gather*}
so,
\[
\int_{  }^{ } \frac{ 2\sec^{ 2 } \left( \theta \right) d\theta }{ 4\sec^{ 2 } \left( \theta \right)  }
.\] 

Where we can use $ \theta = \arctan^{  }\left( \frac{ x }{ 2 }  \right) $ to change our limits to $ \int_{ 0 }^{ \frac{ \pi }{ 2 }  }  $, so
\[
\frac{ 1 }{ 2 } \theta \bigg|_{ 0 }^{ \frac{ \pi }{ 2 }  } =\frac{ \pi }{ 4 }
.\] 
Now because this integral is a finite value we find that our sum will converge. However this won't be the limit of the sum and this will be talked about on monday i think.

\end{document}
