\documentclass[a4paper]{article}

\usepackage[utf8]{inputenc}
\usepackage[T1]{fontenc}
\usepackage{textcomp}
\usepackage[english]{babel}
\usepackage{amsmath, amssymb}

\title{Series}
\author{yawnbo}
\date{\today}
\maketitle

\pdfsuppresswarningpagegroup=1

\begin{document}
\section{Limits of recursive sequences}%
\label{sec:Limits of recursive sequences}

\paragraph{Ex.}%
\label{par:Ex.}

\[
a_1 = \sqrt{ 3}, a_2 = \sqrt{ 3 \cdot \sqrt{ 3}}, a_3 = \sqrt{ 3 \cdot \sqrt{ 3 \cdot \sqrt{ 3}}}, \ldots
.\] 

\paragraph{Definition\\}

If we let $ a_1 = \sqrt{ 3} $ Then for $ n>1 $: $ a_n = \sqrt{ 3\cdot a_{ n-1 }} $.
This creates our nesting and we can begin to solve for our limit. 

\paragraph{First \\}
We want to prove that $ a_n $ has a limit. For this, we need to prove that the sequence is monotonic (increase/decrease) and that we are bounded somewhere along the sequence. When these two parts are proven together, we know that we have convergence at some point. \\

For this we can just write out a few terms, so,
\[
a_1 = \sqrt{ 3} \approx 1.77, a_2 = \sqrt{ 3\cdot \sqrt{ 3}} \approx 2.28, a_3 = \sqrt{ 3\cdot \sqrt{ 3\cdot \sqrt{ 3}}} \approx 2.62, a_{ 4 } \approx 2.8 , a_{ 5 } \approx 2.90
.\] 
This differs from how we would find the limit with a function because we can't take a solid $ \frac{ d }{ dx }  $ so we can just do this for now to find an approximation.\\

Due to the above, we can prove that we have an increasing sequence so we can define some upper limit/bound to prove. If we write out a few more terms, $ a_6 \approx 2.95 a_7 \approx 2.97 $ which seems to approach 3 which we can label as our upper bound. \\

This is all that is needed to prove that a limit EXISTS but we still need to find the limit. Let's assume that our limit is $ L $. Now we should know that,
\[
\lim_{ n \to \infty} a_n = L
.\] 
\[
\lim_{ n \to \infty} a_{ n-1 }= L 
.\] 
\newpage
Knowing that these should be equal, we can write out and solve for our limits
\[
a_n = \sqrt{ 3\cdot a_{ n-1 }}
.\] 
\[
\lim_{ n \to \infty} a_n = \sqrt{ 3\cdot \lim_{ n \to \infty} \cdot a_{ n-1 }}
.\] 
\[
L = \sqrt{ 3\cdot L}
.\] 
\[
L^2=3L \to l^2-3l = 0 \to L(L-3) = 0
.\] 
Which shows that $ L = 0 \text{ or }3 $, but we ignore the 0 because our sequence is non-zero. 

\section{Geometric sequences}%
\label{sec:Geometric sequences}
\paragraph{Recall}
that an arithmetic sequence is one that increases in the pattern of $ a_1 + m = a_2 $ where we add by a factor of m while a geometric sequence is one that increases in the pattern of $ a_1 \cdot r, a_1 \cdot r^2, a_1 \cdot r^{ 3 }$ where we instead multiply by a factor of r. Basically just common difference vs common ration. 

\paragraph{Convergence of $ \left\{ r^{ n } \right\}_{ n=1 }^{ \infty }  $.}
If we were to take the limit ($ \lim_{ n \to \infty} r^{ n }$), then we have a few cases to look at, first, if $ r=1 $ then our limit will just approach 1 because our sequence will look like $ 1^{ 1 }, 1^{ 2 },1^{ 3 },1^{ 4 }\ldots $ which will forever be one. At the case of $ r=-1 $, we will have a sequence that looks like $ -1^{ 1 }, -1^{ 2 }, -1^{ 3 }, -1^{ 4 }\ldots $ which will oscillate between 1 and -1 giving us a divergent sequence. If we instead have $ -1<r<1 $ then we will have a convergent sequence that will approach 0. If we have $ r>1 $ then we will have a divergent sequence that will approach positive $ \infty $ while for $ r<1 $ we would have $ \pm \infty $, but for each of these cases, the limits don't exist. 

\paragraph{Summary}
To end, $ r^{ n } \text{ converges for } r\epsilon\left( -1,1 \right] $.

\section{Sec 11.2 Series}%
\label{sec:Sec 11.2 Series}

\paragraph{Definition\\}
For $ \left\{ a_n \right\} ^{ \infty }_{ n=1 } $, the sum of its terms $ a_1 + a_2 + a_3 \ldots = \sum_{ n=1 } ^{ \infty } a_n$ is referred to as an infinite series or just a series. \\

Our goal with this section is to find out when our sum will equal a finite, infinite, or indeterminate value. Basically just convergence and divergence with fancy terms now. \\

An important thing to note is that,
\[
\sum_{ i=1 } ^{ n } a_{ i } \text{ is always finite}
.\] 

\paragraph{Ex.}
\[
\sum_{ i=1 } ^{ n } i= 1+2+3+4+5\ldots + \left( n-1 \right) + n = \frac{ n\left( n+1 \right)  }{ 2 }
.\] 

\newpage
\paragraph{Definition again\\}
For $ \sum_{ i=1 } ^{ \infty } a_i $, 
\[
\sum_{ i=1 } ^{ n } a_i = S_n \text{ which is the nth partial sum of }\sum_{ n=1 } ^{ \infty } a_n
.\] 

\paragraph{Ex. we can have a series like,}
\begin{align*}
&\sum_{ i=1 } ^{ n } a_i \\
S_1 &= \sum_{ i=1 } ^{ 1 } a_i = a_1 \\
S_2 &= \sum_{ i=1 } ^{ 2 } a_i = a_1 + a_2 \\
S_3 &= \sum_{ i=1 } ^{ 3 } a_i = a_1 + a_2 + a_3 \\
\ldots
S_n &= \sum_{ i=1 } ^{ n } a_i = a_1 + a_2 + a_3 + \ldots + a_n
\end{align*}
which is really just a sequence of partial sums. Going back to our definition we know that $ \lim_{ n \to \infty} \left\{ S_n \right\} =\sum_{ i=1 } ^{ \infty } a_i $. But if our limit DNE then our sum will diverge instead. 

\subsection{Going back to Geometric Series}%
\label{sub:Going back to Geometric Series}

This is a case where we would use our sum.
\[
\sum_{ n=1 } ^{ \infty } ar^{ n-1 }=a + ar^{ 1 } + ar^{ 2 } + ar^{ 3 } + \ldots = a\left( 1+r+r^{ 2 }+r^{ 3 }+\ldots \right)
.\] 
We can find our common ratio to be $ \frac{ ar^3 }{ ar^2 }=r $.\\

Now lets say we want to find when does $ \sum_{  } ^{  } ar^{ n-1 } $ converge? Lets try with some cases,
\begin{gather*}
r=1 \\
\sum_{ i=1 } ^{ \infty } ar^{ i-1 } = a + a + a + a + \ldots \\
\sum_{ i=1 } ^{ n } ar^{ i-1 }= a + a + a \ldots + a = na = \left\{ S_{ i } \right\}^{ \infty }_{ i=1 } 
\end{gather*}

Now taking the limit, 
\[
\lim_{ i \to \infty} \left\{ S_{ i } \right\} =\lim_{ n \to \infty} \sum_{ i=1 } ^{ \infty } a\left( 1 \right) = \lim_{ n \to \infty} na = \pm \infty
.\] 
We find that this diverges, unlike $ \left\{ a\left( 1 \right) ^{ n } \right\}  $ which converges. So adding our sum makes our series diverge. \\

Now lets try,
\begin{gather*}
r\neq 1: -1<r<1\\
\sum_{ i=1 } ^{ n } ar^{ i-1 }=a+ar+ar^2+ \ldots + ar^{ n-1 }= S_{ n }
\end{gather*}
But this doesn't get us anywhere so lets multiply the whole thing by r,
\[
\sum_{ i=1 } ^{ n } ar^{ i-1 }= ar + ar^2+ ar^3+ \ldots + ar^{ n }= S_n \cdot r
.\] 
Lets call the equation before introducing the r $ eq_1 $ and the second one $ eq_2 $. Subtracting $ eq_1 $ from $ eq_2 $ we get,
\[
a-ar^{ n } = S_n - S_n r = S_n\left( 1-r \right) \to
.\] 
\[
S_n=\frac{ a-ar^{ n } }{ 1-r } = \frac{ a\left( 1-r^{ n } \right)  }{ 1-r }
.\] 
Which proves that 
\[
\sum_{ i=1 } ^{ n } ar^{ n-1 }=S_n = \frac{ a\left( 1-r^{ n } \right)  }{ 1-r }
.\] 
\paragraph{Example where this is useful}
\[
\sum_{ i=1 } ^{ 100 } 2\left( \frac{ 3 }{ 7 }  \right) ^{ i-1 }
.\] 
With our formula we can plug in all our values to find that
\[
	\frac{ 2\left( 1-\frac{ 3 }{ 7 }  \right)^{ 100 } }{ 1-\frac{ 3 }{ 7 }  }
.\] 	
\end{document}
