\section{Short class 01/15}%
\label{sec:Short class 01/15}

\subsection{Geometric and power series review}%
\label{sub:Geometric and power series review}

Geometric happens when we are using 
\[
\sum_{ n=1 } ^{ \infty } ar^{ n-1 }=\frac{ a }{ 1-r } 
.\] 
Power happens when we have 
\[
1+x+x^2+x^3+\ldots=\sum_{ n=0 } ^{ \infty } x^{ n }=\frac{ 1 }{ 1-x } ,\left\| x \right\|<1 \text{ or }-1<x<1
.\] 

\paragraph{Ex}%
\label{par:Ex}
\[
\text{ For }\sum_{ n=0 } ^{ \infty } \left( -2x \right) ^{ n }\text{ Find the result, and the x-values it converges for }
.\] 
Just plug in values into our formula to get
\[
\frac{ 1 }{ 1-\left( -2x \right) ^{ n } }, \left\| -2x \right\|<1 \text{ or }\frac{ 1 }{ 2 } >x>-\frac{ 1 }{ 2 } \text{ which is our interval of convergence}
.\] 
This is basically the same thing as the geometric sequence but, we are solving for our r value instead of knowing it off the bat.

\paragraph{Ex.}
\[
\text{ For }\sum_{ n=0 } ^{ \infty } \left( \frac{ 2 }{ x }  \right) ^{ n }
.\]
With our identity we get
\[
\frac{ 1 }{ 1-\frac{ 2 }{ x }  } =\frac{ x }{ x-2 } 
.\] 
For this we get a technical detail that $ x\neq 2 $. Now to find our interval of convergence we can find that we have $ \left\| \frac{ 2 }{ x }  \right\|<1 $ which gives us $ \left| \frac{ 1 }{ x }  \right|<\frac{ 1 }{ 2 }   $ which is basically saying that we can have an x value anywhere that isn't between -2 and 2 inclusive. 

\section{Telescoping series}%
\label{sec:Telescoping series}
\paragraph{Ex.}%
\label{par:Ex.}
\[
\sum_{ n=1 } ^{ \infty } \frac{ 1 }{ n\left( n+1 \right)  } 
.\] 
If we were to just start getting some terms we would have
\[
\frac{ 1 }{ 2 } +\frac{ 1 }{ 6 } +\frac{ 1 }{ 12 } +\ldots
.\] 
But what happens if we instead do a partial decomposition of our function?
\begin{gather*}
\frac{ 1 }{ n\left( n+1 \right)  } =\frac{ A }{ n } +\frac{ B }{ n+1 } \\
1=A\left( n+1 \right) +B\left( n \right) \implies B = -1 \text{ and }A=1\\
\end{gather*}

Going back to the sum we can now rewrite it as
\begin{gather*}
\sum_{ n=1 } ^{ \infty } \left( \frac{ 1 }{ n } -\frac{ 1 }{ n+1 }  \right) \\
=\left( \frac{ 1 }{ 1 } -\frac{ 1 }{ 2 }  \right) +\left( \frac{ 1 }{ 2 } -\frac{ 1 }{ 3 }  \right) +\left( \frac{ 1 }{ 3 } -\frac{ 1 }{ 4 }  \right) +\ldots+\left( \frac{ 1 }{ n-1 } -\frac{ 1 }{ n }  \right)  +\left( \frac{ 1 }{ n } -\frac{ 1 }{ n+1 }  \right) 
\end{gather*}
Looking at our terms, we can see that all of them cancel other than 
\[
1-\frac{ 1 }{ n+1 } 
.\] 
Which holds for all telescope series. Now we can easily take the limit to find 
\[
\lim_{ n \to \infty} S_n = \lim_{ n \to \infty} \left( 1-\frac{ 1 }{ n+1 }  \right) =1
.\] 
For all telescopic series' we can easily find the answer if we break the function into partials. While it's possible to do with the original function it's a lot harder and should usually be done with partials. 

Harmonic series will be done tomorrow. 
