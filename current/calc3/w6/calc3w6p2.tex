\section*{02/12/25}%
\label{sec:02/12/25}
\paragraph{Practice review is out and the review has an extra point problem about fractals worth 4 points. }

\begin{exampleblock}{}
	\paragraph{Ex.}
	Find the maclaurin series for $ f\left( x \right) = \tan^{ -1 } \left( x \right)  $. Start by getting our derivatives,
	\begin{align*}
		f\left( x \right) &= \tan^{ -1 } \left( x \right) \\
		f^{ 1 }\left( x \right) &= \frac{ 1 }{ 1+x^2 } \\
		f^{ 2 }\left( x \right) &= \frac{ -1\left( 1+x^2 \right) ^{ -2 }\cdot 2x }{ \left( 1+x^2 \right) ^2 } \\
		f^{ 3 }\left( x \right) &= \frac{ -2\left( 1+x^2 \right)^{ -3 } -2x\left( 1+x^2 \right)^{ -2 }  }{ \left( 1+x^2 \right) ^2 } = 2\left( 1+x^2 \right) ^{ -3 }\left( 2x-\left( 1+x^2 \right) \right) \\
		f^{ 4 }\left( x \right) &=  \frac{ 24x\left( 1+x^2 \right)  }{ \left( 1+x^2 \right) ^{ 4 } }  \\
		f^{ 5 }\left( x \right) &= \frac{ 24\left( 1-10x^2+5x^{ 4 } \right)  }{ \left( 1+x^2 \right) ^{ 5 } } \\
	\end{align*}
	Because we want a mclaurin, we want to evaluate each function at 0,
	\begin{align*}
		f\left( 0 \right) &= \tan^{ -1 } \left( 0 \right) = 0 \\
		f^{ 1 }\left( 0 \right) &= \frac{ 1 }{ 1+0 } = 1 \\
		f^{ 2 }\left( 0 \right) &= \frac{ 0 }{ 1 } = 0 \\
		f^{ 3 }\left( 0 \right) &= = -2 \\
		f^{ 4 }\left( 0 \right) &= 0 \\
		f^{ 5 }\left( 0 \right) &= 24 \\
	\end{align*}
	We see that this gives us the pattern $ \underbrace{ 0 }_{ k=0 } ,1,0,-2,0,24,0,-720 $. Looking at these we see that we have a sequence of even factorials. This can now be built into a sum,
	\[
	\sum_{ k=0 } ^{ \infty } \frac{ \left( -1 \right) ^{ k }\left( 2k \right) ! }{ k! }
	.\] 
	Now because we only want the odds we can instead write it as,
	\[
	\sum_{ k=0 } ^{ \infty } \frac{ \left( -1 \right) ^{ k }\left( 2k \right) ! }{ \left(2k+1\right)! }x^{ 2k+1 }
	.\] 
	Now cancelling,
	\[
	\sum_{ k=0 } ^{ \infty } \frac{ \left( -1 \right) ^{ k } }{ 2k+1 }x^{ 2k+1 }
	.\] 
	\paragraph{The easier way with integration}
	Start by taking the $ \frac{ d }{ dx }  $,
	\[
	\frac{ d }{ dx } \tan^{ -1 } \left( x \right) = \frac{ 1 }{ 1+x^2 } \to \frac{ 1 }{ 1-\left( x^2 \right)  } = \sum_{ k=0 } ^{ \infty } \left( -x^2 \right) ^{ n }
	.\] 
	But this is just the sum of the derivative, so we can integrate both sides to find
	\[
	\int_{  }^{  } \frac{ d }{ dx } \tan^{ -1 } \left( x \right) =\int_{  }^{  } \sum_{ k=0 } ^{ \infty } \left( -1 \right) ^{ n }x^{ 2n } \to \sum_{ k=0 } ^{ \infty } \left( -1 \right) ^{ n }\frac{ x^{ 2n+1 } }{ 2n+1 }+C
	.\] 
	Now to find C we just plug in 0 for x,
	\[
	\underbrace{ \sum_{ k=0 } ^{ \infty } \frac{ \left( -1 \right) ^{ n }0^{ 2n+1 } }{ 2n+1 } }_{ 0 } \implies C = 0
	.\] 
	This again goes back to the same series we found earlier and shows that power series are unique up to their IOC, or there is only ONE power series for each IOC. 
	\paragraph{Ex.Making a new series from old series.}
	Knowing that \[
	e^{ x }=\sum_{ n=0 } ^{ \infty } \frac{ x^{ n } }{ n! }
	.\] 
	How can we make a power series for $ e^{ 5x+1 } $?
	We can just do this with subsitution where our new power series will look like 
	\[
=	\sum_{ n=0 } ^{ \infty } \frac{ \left( 5x+1 \right) ^{ n } }{ n! }
	.\] 
	Same thing for $ e^{ x^{ 3 } } $,
	\[
	=\sum_{ n=0 } ^{ \infty } \frac{ \left( x^3 \right) ^{ n } }{ n! }
	.\] 
	For something like $ \ln^{  } \left( x^2 \right)  $, we would have to set the inside of our known  maclaurin, $ \ln^{  } \left( 1+x \right) =\sum_{ k=1 } ^{ \infty } \left( -1 \right) ^{ k-1 }\frac{ x^{ k } }{ k } $ to $ x^2 $ so,
	\[
	\ln^{  } \left( x^2 \right) = \ln^{  } \left( 1+\left( x^2-1 \right)  \right) = \sum_{ k=1 } ^{ \infty } \left( -1 \right) ^{ k-1 }\frac{ \left( x^2-1 \right) ^{ k } }{ k }
	.\] 
	Also works for $ \frac{ 1 }{ x }  $ like
	\[
	\frac{ 1 }{ 1-\left( 1-x \right)  } = \sum_{ k=0 } ^{ \infty } \left( 1-x \right) ^{ n }
	.\] 
\paragraph{Find the taylor series for $ f\left( x \right) =x^{ -3 } $ centered at $ c=1 $}
Taking $ \frac{ d }{ dx }  $ 's we get,
\begin{align*}
	f\left( x \right) &= x^{ -3 } \to 1\\
	f^{ 1 }\left( x \right) &= -3x^{ -4 } \to -3 \\
	f^{ 2 }\left( x \right) &= 3\cdot 4 x^{ -5 } \to 3\cdot 4\\
	f^{ 3 }\left( x \right) &= -3\cdot 4\cdot 5 x^{ -6 } \to -3\cdot 4\cdot 5\\
	\ldots \\
\end{align*}
Giving us the sequence,
\[
f^{ n }\left( 1 \right) =\left( -1 \right) ^{ n }\frac{ \left( n+2 \right) ! }{ 2 }
.\] 
Taylor's theorem tells us that this will be
\[
x^{ -3 }= \sum_{k=0}^{\infty} \frac{ \left( -1 \right) ^{ k }\left( k+2 \right) ! }{ 2\cdot k! }\left( x-1 \right) ^{ k } = \sum_{ k=0 } ^{ \infty } \left( -1 \right) ^{ k } \frac{ \left( k+2 \right) \left( k+1 \right) \left( x-1 \right) ^{ k } }{ 2 }
.\] 
With this IOC can be done using ratio test, and should become $ \left| x -1 \right| < 1 $ or $ R=1 $. 
\end{exampleblock}
Friday's class will probably be the last lecture before the test and will be on sec11.11 or the $ T_n\left( x \right) : $ n-th degree Taylor polynomial. 
\paragraph{Quick Ex.}
\[
e^{ x }=\sum_{ k=0 } ^{ \infty } \frac{ x^{ k } }{ k! }= 1 + \frac{ x }{ 1 } +\frac{ x^2 }{ 2 } + \frac{ x^3 }{ 3! }+ \frac{ x^{ 4 } }{ 4! }\ldots
.\] 
If we take a partial sum $ S_0 $ then this would be called the $ T_0\left( x \right) 1 $ or the $ 0^{ \text{ th } } $-deg Taylor Polynomial. Basically just the estimation of the radical function in terms of a taylor polynomial. 
