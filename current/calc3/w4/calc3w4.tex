\documentclass[a4paper]{article}

\usepackage{cancel}
\usepackage[utf8]{inputenc}
\usepackage[T1]{fontenc}
\usepackage{textcomp}
\usepackage[english]{babel}
\usepackage{amsmath, amssymb}

\title{W4 Notes}
\author{yawnbo}
\date{\today}

\pdfsuppresswarningpagegroup=1

\begin{document}
\maketitle
\section*{11.5 Ratio/root test \& absolute convergence}%
\label{sec:11.5 Ratio/root test}
\paragraph{Def. Absolute convergence}
A series $ \sum_{  } ^{  } a_n $ is absolutely convergent if $ \sum_{  } ^{  } \left\| a_n \right\| $ converges.
\paragraph{Ex.}
\[
\sum_{ n=1 } ^{ \infty } \frac{ -1^{ n } }{ n^{ 4 } } \to \sum_{  } ^{  } \left\| \frac{ -1^{ n } }{ n^{ 4 } } \right\| = \sum_{ n=1 } ^{ \infty } \frac{ 1 }{ n^{ 4 } } 
.\] 
The above will converge because of p-test so this series will be absolutely convergent. All the definition is that the series will converge if the absolute value of the series converges. \\

\paragraph{Def. Conditional convergence}
$ \sum_{  } ^{  } a_n $ is conditionally convergent if $ \sum_{  } ^{  } \left\| a_n \right\| $ converges but $ \sum_{ } ^{  } a_n $ diverges. 

\paragraph{Ex.}
\[
	\sum_{ n=1 } ^{ \infty } \frac{ \left( -1 \right) ^{ n+1 } }{ n } \text{ is conditionally convergent.  }
.\] 
This is because if we put an absolute to make it $ \sum_{ n=1 } ^{ \infty } \left\| \frac{ \left( -1 \right) ^{ n+1 } }{ n } \right\| $ will equal $ \sum_{ n=1 } ^{ \infty } \frac{ 1 }{ n }  $ which is the harmonic series.

\subsection*{11.5.1 Theorem}%
\label{sub:11.5.1 Theorem}
\paragraph{Def.}
If $ \sum_{  } ^{  } \left\| a_n \right\| $ converges, then $ \sum_{  } ^{  } a_n $ converges. With the contrapositive being, if $ \sum_{  } ^{  } a_n $ diverges, then $ \sum_{  } ^{  } \left\| a_n \right\| $ also diverges. 
\paragraph{Ex. Show that $ \sum_{  } ^{  } \frac{ \sin^{  } \left( n^2 \right)  }{ n^2 } $ converges or diverges.}
Let's use the previously established theorem, consider 
\[
\sum_{ n=1 } ^{ \infty } \left\| \frac{ \sin^{  } \left( n^2 \right)  }{ n^2 } \right\|
.\] As our new series and use squeeze theorem to show that it converges.

\[
\sum_{ n=1 } ^{ \infty } \left\| \frac{ \sin^{  } \left( n^2 \right)  }{ n^2 } \right\| \le \sum_{ n=1 } ^{ \infty } \frac{ 1 }{ n^2 } 
.\] 
Which can now be found using direct comparison because our new series is larger than our sin. Know that we have a convergent series that is larger, we know that the smaller series must also converge by absolute test and comparison. 
\subsection*{11.5.2 Examples}%
\label{sub:11.5.2 Examples}

\paragraph{For a) $ \sum_{ k=1 } ^{ \infty } \frac{ \left( -1 \right) ^{ k+1 } }{ \sqrt{ k} } \text{ b)  }\sum_{ n=1 } ^{ \infty } \frac{ \left( -1 \right) ^{ n+1 } }{ \sqrt{ n^3} } \text{ c)  } \sum_{ n=1 } ^{ \infty } \frac{ \left( -1 \right) ^{ k }k }{ k+1 } $ }
\paragraph{Which series converge absolutely, converge conditionally, or diverge?}
Each of these become
\begin{align*}
a = \sum_{  } ^{  } \frac{ 1 }{ \sqrt{ k} } \text{ which diverges by p-test } \\
b = \sum_{  } ^{  } \frac{ 1 }{ \sqrt{ n^3} } \text{ which converges by p-test } \\
c = \sum_{  } ^{  } \frac{ k }{ k+1 } \text{ which diverges } \\
.\end{align*}
\paragraph{a will be conditionally convergent by AST because $ \frac{ 1 }{ \sqrt{ k} } $ is decreasing and $ \lim_{ k \to \infty } \frac{ 1 }{ \sqrt{ k} } = 0 $. b will be absolutely convergent because $ \frac{ 1 }{ \sqrt{ n^3} } $ is decreasing and $ \lim_{ n \to \infty } \frac{ 1 }{ \sqrt{ n^3} } = 0 $. c will diverge because $ \frac{ k }{ k+1 } $ is increasing and $ \lim_{ k \to \infty } \frac{ k }{ k+1 } = 1 $.}

\subsection*{11.5.3 The ratio test}%
\label{sub:11.5.3 The ratio test}
(i) If $ \lim_{ n \to \infty} \left\| \frac{ a_{ n+1 } }{ a_n } \right\|<1 $ but not 0, then $ \sum_{  } ^{  } \left\| a_n \right\| $ converges. \\ \\
(ii) If $ \lim_{ n \to \infty} \left\| \frac{ a_{ n+1 } }{ a_n } \right\| >1 $ then $ \sum_{  } ^{  } a_n $ is divergent. \\
(iii) If $ \lim_{ n \to \infty} \left\| \frac{ a_{ n+1 } }{ a_n } \right\|=1 $ then the test is inconclusive
\paragraph{Ex.}
\[
\text{ Test }\sum_{ n=1 } ^{ \infty } \frac{ \left( -1 \right) ^{ n-1 }n }{ e^{ n } } \text{ for convergence or divergence }
.\]
Start by plugging in $ a_{ n+1 } $ as our value so,
\[
a_{ n+1 }= \frac{ \left( -1 \right) ^{ \left( n+1 \right) -1 } n+1}{ e^{ n+1 } } 
.\] 
And write it over the absolute to get rid of the $ -1^{ n } $,
\[
\left\| \frac{ a_{ n+1 } }{ a_n } \right\| = \frac{ \frac{ n+1 }{ e^{ n+1 } } }{ \frac{ n }{ e^{ n } }  } = \lim_{ n \to \infty} \frac{ n+1 }{ n }\cdot \frac{ e^{ n } }{ e^{ n } } = \frac{ e^{ n } }{ e^{ n }\cdot e } = \frac{ 1 }{ e } = 1\cdot \frac{ 1 }{ e } = \frac{ 1 }{ e } <1
.\] 
Which solves our (i) case and we know that the series converges. This can also be done with the integral test using
\[
\int_{ 1 }^{ \infty } \frac{ x }{ e^{ x } } dx
.\] 
and showing that it is finite. End of monday class.

\section*{Tests cont.}%
\label{sec:Tests}
\paragraph{Note that quiz 3 is open by tomorrow and due in a week I assume.}

\paragraph{Ex.}
Apply the ratio test to 
\[
\sum_{ n=0 } ^{ \infty } \left( -1 \right) ^{ n }\frac{ \sqrt{ n} }{ n+1 }
.\] 
Note that our sequence can be called $ a_n $. Now if we sub n for n+1 we can now take the limit,
\[
\lim_{ n \to \infty} \left\| \frac{ a_{ n+1 } }{ a_n } \right\| = \lim_{ n \to \infty} \frac{ \sqrt{ n+1} }{ n+2 }\cdot \frac{ n+1 }{ \sqrt{ n} }= \sqrt{ \frac{ n+1 }{ n }} \cdot \frac{ n+1 }{ n+2 } = 1\cdot 1 = 1
.\] 
Which makes our test inconclusive. This just means that we have to use another test to determine convergence or divergence. \\
Instead looking at our sequence, we can use AST to find that it's monotonically decreasing and that the limit is 0. This means that the series converges by AST. So because $ \frac{ \sqrt{ n} }{ n+1 } $ is monotonically decreasing and $ \lim_{ n \to \infty} \frac{ \sqrt{ n} }{ n+1 } = 0 $, we can say that the series converges by AST.

\paragraph{What if we also wanted to find if the absolute also converges?}
\[
\sum_{ } ^{  } |a_n| = \sum_{  } ^{  } \frac{ \sqrt{ n} }{ n+1 }
.\] 
Comparing to a smaller function $ \sum_{  } ^{  } \frac{ 1 }{ n+1 } $ shows that the series converges by p-test. This means that the series conditionally converges.
\subsection*{11.5.4}%
\label{sub:11.5.4}
\paragraph{The root test }
(i) If $ \lim_{ n \to \infty} \sqrt[ n ]{ \left\| a_n \right\| } <1 $, then $ \sum_{  } ^{  } \left\| a_n \right\| $ converges \\
(ii) If $ \lim_{ n \to \infty} \sqrt[ n ]{ \left\| a_n \right\| } >1 $, then $ \sum_{  } ^{  } a_n $ diverges \\
(iii) If $ \lim_{ n \to \infty} \sqrt[ n ]{ \left\| a_n \right\| } =1 $, then inconclusive. \\ \\ 
\paragraph{Ex.}
\[
\sum_{ n=1 } ^{ \infty } \left( \frac{ n+1 }{ 2n } \right) ^{ n }
.\] 
We commonly use this test when we have something complex to the power of n, so,
\[
\sqrt[ n ]{ \left\| a_n \right\| } =\sqrt[ n ]{ \frac{ n+1 }{ 2n }^{ n } } = \lim_{ n \to \infty} \frac{ n+1 }{ 2n }=\frac{ 1 }{ 2 } < 1
.\] 
Which proves that our sum $ \sum_{  } ^{  } a_n $ converges because its less than 1.

\paragraph{Ex.}
\[
\sum_{ k=1 } ^{ \infty } \left( 1+\frac{ 3 }{ k }  \right) ^{ k^2 }
.\] 
Start by applying the root,
\[
\sqrt[ k ]{ \left\| a_n \right\| } = \sqrt[ k ]{ \left( 1+\frac{ 3 }{ k }  \right) ^{ k^2 } } = \lim_{ n \to \infty}  \left( 1+\frac{ 3 }{ k }  \right) ^{ k }
.\] 
Solving this at the above point makes it indecisive. So we can use the ratio test to find that the series converges. Let our limit equal L, then,
\[
k \ln^{  } \left( 1+\frac{ 3 }{ k }  \right) = \ln^{  } \left( L \right) 
.\] 
Now as $ k \to \infty $ we get an indeterminate form so we can use L'Hopital's rule to get,
\[
k\ln^{  } \left( 1+\frac{ 3 }{ k }  \right) = \frac{ \frac{ 1 }{ 1+\frac{ 3 }{ k }  }\cdot \left( -\frac{ 3 }{ k^2 }  \right)  }{ -\frac{ 1 }{ k^2 }  } = \frac{ 3 }{ 1+\frac{ 3 }{ k }  } \to 3 = \ln^{  } \left( L \right) 
.\] 
So $ L = e^{ 3 } $, which means that the series diverges because $ e^{ 3  }>1 $.
\paragraph{Everything above is done up to section 11.6 and 11.7 is just review on how to actually use these tests}

\paragraph{Example list to do if wanted (compare first to second to prove divergence or convergence)}
\begin{align*}
	1.& \sum_{  } ^{  } \frac{ 1 }{ 5^{ n } } , \sum_{  } ^{  } \frac{ 1 }{ 5^{ n }+n } \\
2.&\sum_{  } ^{  } \frac{ \left( -1 \right) ^{ n } }{ n^{ \frac{ 3 }{ 2 }  } }, \sum_{  } ^{  } \frac{ 1 }{ n^{ \frac{ 3 }{ 2 }  } } \\
3.&\sum_{  } ^{  } \frac{ n }{ r^{ n } } , \sum_{  } ^{  } \frac{ 3^{ n } }{ n } \\
4.&\sum_{  } ^{  } \frac{ n+1 }{ n }, \sum_{  } ^{  } \left( -1 \right) ^{ n } \frac{ n+1 }{ n } \\
5.&\sum_{ n=1 } ^{  } \frac{ n }{ n^2+1 } , \sum_{  } ^{  } \left( \frac{ n }{ n^2+1 }  \right) ^{ n } \\
6.&\sum_{  } ^{  } \frac{ \ln^{  } \left( n \right)  }{ n }, \sum_{ n=10 } ^{  } \frac{ 1 }{ n\ln^{  } \left( n \right)  } \\
7.&\sum_{  } ^{  } \frac{ 1 }{ n+n! } , \sum_{  } ^{  } \left( \frac{ 1 }{ n } +\frac{ 1 }{ n! }  \right) \\
8.&\sum_{  } ^{  } \frac{ 1 }{ \sqrt{ n^2+1} } , \sum_{ } ^{  } \frac{ 1 }{ n\sqrt{ n^2+1} } \\
\end{align*}

\documentclass[a4paper]{article}

\usepackage[utf8]{inputenc}
\usepackage[T1]{fontenc}
\usepackage{textcomp}
\usepackage[english]{babel}
\usepackage{amsmath, amssymb}
\title{Review for exam 3}
\author{yawnbo}
\date{\today}

\maketitle

\pdfsuppresswarningpagegroup=1

\begin{document}
\paragraph{Question 5}
\[
\int_{}^{} e^{\sqrt[3]{x}}dx \text{ using u sub  }
.\] 
\[
u=x^{\frac{1}{3}} \text{ an } du=\frac{1}{3}x^{-\frac{2}{3}}dx
.\] 
Solving for dx gets 
\[
dx= \frac{3du}{x^{-\frac{2}{3}}}=3x^{\frac{2}{3}}du=3\left( x^{\frac{1}{3}} \right) ^2du=3u^2du
.\] 
Which can then be tabulated to
\[
3u^2e^{4}-6ue^{u}+6u+C
.\] 
and subbing back in x
\[
3x^{\frac{2}{3}}e^{\sqrt[3]{x}}-6^{\sqrt[3]{x}}e^{\sqrt[3]{x}}+6\sqrt[3]{x}+C
.\] 
\section{Comparison examples from yesterday}%
\label{sec:Comparison examples from yesterday}

\begin{align}
 &\int_{1}^{\infty} \frac{dx}{\sqrt{x}+e^{3x}} \\
 &\int_{0}^{0.5} \frac{dx}{x^{8}+x^2} \\
 &\int_{1}^{\infty} \frac{dx}{\sqrt{x^{5}+2}} \\
 &\int_{0}^{5} \frac{dx}{x^{\frac{1}{3}}+x^3} \\
 &\int_{0}^{\infty} \frac{dx}{\sqrt{x^{\frac{1}{3}+x^3}}} \\
 &\int_{1}^{\infty} \frac{dx}{x^{4}+e^{x}}\\
\end{align}
Generally guess which one we would assume we have, for one assume convergence, so we need somthing larger.
\newpage
\subsection{For number 5}%
\label{sub:For number 5}
\paragraph{This is a type 1 and two improper.}
\[
\int_{0}^{\infty} \frac{dx}{\sqrt{x^{\frac{1}{3}}+x^3}}
.\] 
becomes
\[
\int_{0}^{1} \frac{dx}{\sqrt{x^{\frac{1}{3}}+x^3}} + \int_{1}^{\infty} \frac{dx}{\sqrt{x^{\frac{1}{3}}+x^3}}
.\] 
By dominance theory we can look at the second integral of $x^3$ which converges by p theorem. 
Now we look at the first one to establish its divergence so we need a function thats bigger
\[
x^{\frac{1}{3}}+x^3 \le 2x^{\frac{1}{3}}
.\] 
We choose that number because when testing $x^3+x^3$ we find that just one of them is smaller than $x^{\frac{1}{3}}$ which means that if we are finding a larger function we should take our x to the power of a fraction.
Doing normal comparison we find that this converges instead and our analysis is wrong because we found a value that doesn't help us

Now we want to try something like $x^{-1}$ to get a value that is 

THiS ONE IS ON THE EXAM THIS NEEDS TO BE FIGURED OUT 
\[
\int_{0}^{\infty} \frac{dx}{\sqrt{x^{\frac{1}{3}}+x^3}}
.\] 
\end{document}

\end{document}
