\documentclass[a4paper]{article}

\usepackage{cancel}
\usepackage[utf8]{inputenc}
\usepackage[T1]{fontenc}
\usepackage{textcomp}
\usepackage[english]{babel}
\usepackage{amsmath, amssymb}

\title{W4 Notes}
\author{yawnbo}
\date{\today}

\pdfsuppresswarningpagegroup=1

\begin{document}
\maketitle
\section*{11.5 Ratio/root test \& absolute convergence}%
\label{sec:11.5 Ratio/root test}
\paragraph{Def. Absolute convergence}
A series $ \sum_{  } ^{  } a_n $ is absolutely convergent if $ \sum_{  } ^{  } \left\| a_n \right\| $ converges.
\paragraph{Ex.}
\[
\sum_{ n=1 } ^{ \infty } \frac{ -1^{ n } }{ n^{ 4 } } \to \sum_{  } ^{  } \left\| \frac{ -1^{ n } }{ n^{ 4 } } \right\| = \sum_{ n=1 } ^{ \infty } \frac{ 1 }{ n^{ 4 } } 
.\] 
The above will converge because of p-test so this series will be absolutely convergent. All the definition is that the series will converge if the absolute value of the series converges. \\

\paragraph{Def. Conditional convergence}
$ \sum_{  } ^{  } a_n $ is conditionally convergent if $ \sum_{  } ^{  } \left\| a_n \right\| $ converges but $ \sum_{ } ^{  } a_n $ diverges. 

\paragraph{Ex.}
\[
	\sum_{ n=1 } ^{ \infty } \frac{ \left( -1 \right) ^{ n+1 } }{ n } \text{ is conditionally convergent.  }
.\] 
This is because if we put an absolute to make it $ \sum_{ n=1 } ^{ \infty } \left\| \frac{ \left( -1 \right) ^{ n+1 } }{ n } \right\| $ will equal $ \sum_{ n=1 } ^{ \infty } \frac{ 1 }{ n }  $ which is the harmonic series.

\subsection*{11.5.1 Theorem}%
\label{sub:11.5.1 Theorem}
\paragraph{Def.}
If $ \sum_{  } ^{  } \left\| a_n \right\| $ converges, then $ \sum_{  } ^{  } a_n $ converges. With the contrapositive being, if $ \sum_{  } ^{  } a_n $ diverges, then $ \sum_{  } ^{  } \left\| a_n \right\| $ also diverges. 
\paragraph{Ex. Show that $ \sum_{  } ^{  } \frac{ \sin^{  } \left( n^2 \right)  }{ n^2 } $ converges or diverges.}
Let's use the previously established theorem, consider 
\[
\sum_{ n=1 } ^{ \infty } \left\| \frac{ \sin^{  } \left( n^2 \right)  }{ n^2 } \right\|
.\] As our new series and use squeeze theorem to show that it converges.

\[
\sum_{ n=1 } ^{ \infty } \left\| \frac{ \sin^{  } \left( n^2 \right)  }{ n^2 } \right\| \le \sum_{ n=1 } ^{ \infty } \frac{ 1 }{ n^2 } 
.\] 
Which can now be found using direct comparison because our new series is larger than our sin. Know that we have a convergent series that is larger, we know that the smaller series must also converge by absolute test and comparison. 
\subsection*{11.5.2 Examples}%
\label{sub:11.5.2 Examples}

\paragraph{For a) $ \sum_{ k=1 } ^{ \infty } \frac{ \left( -1 \right) ^{ k+1 } }{ \sqrt{ k} } \text{ b)  }\sum_{ n=1 } ^{ \infty } \frac{ \left( -1 \right) ^{ n+1 } }{ \sqrt{ n^3} } \text{ c)  } \sum_{ n=1 } ^{ \infty } \frac{ \left( -1 \right) ^{ k }k }{ k+1 } $ }
\paragraph{Which series converge absolutely, converge conditionally, or diverge?}
Each of these become
\begin{align*}
a = \sum_{  } ^{  } \frac{ 1 }{ \sqrt{ k} } \text{ which diverges by p-test } \\
b = \sum_{  } ^{  } \frac{ 1 }{ \sqrt{ n^3} } \text{ which converges by p-test } \\
c = \sum_{  } ^{  } \frac{ k }{ k+1 } \text{ which diverges } \\
.\end{align*}
\paragraph{a will be conditionally convergent by AST because $ \frac{ 1 }{ \sqrt{ k} } $ is decreasing and $ \lim_{ k \to \infty } \frac{ 1 }{ \sqrt{ k} } = 0 $. b will be absolutely convergent because $ \frac{ 1 }{ \sqrt{ n^3} } $ is decreasing and $ \lim_{ n \to \infty } \frac{ 1 }{ \sqrt{ n^3} } = 0 $. c will diverge because $ \frac{ k }{ k+1 } $ is increasing and $ \lim_{ k \to \infty } \frac{ k }{ k+1 } = 1 $.}

\subsection*{11.5.3 The ratio test}%
\label{sub:11.5.3 The ratio test}
(i) If $ \lim_{ n \to \infty} \left\| \frac{ a_{ n+1 } }{ a_n } \right\|<1 $ but not 0, then $ \sum_{  } ^{  } \left\| a_n \right\| $ converges. \\ \\
(ii) If $ \lim_{ n \to \infty} \left\| \frac{ a_{ n+1 } }{ a_n } \right\| >1 $ then $ \sum_{  } ^{  } a_n $ is divergent. \\
(iii) If $ \lim_{ n \to \infty} \left\| \frac{ a_{ n+1 } }{ a_n } \right\|=1 $ then the test is inconclusive
\paragraph{Ex.}
\[
\text{ Test }\sum_{ n=1 } ^{ \infty } \frac{ \left( -1 \right) ^{ n-1 }n }{ e^{ n } } \text{ for convergence or divergence }
.\]
Start by plugging in $ a_{ n+1 } $ as our value so,
\[
a_{ n+1 }= \frac{ \left( -1 \right) ^{ \left( n+1 \right) -1 } n+1}{ e^{ n+1 } } 
.\] 
And write it over the absolute to get rid of the $ -1^{ n } $,
\[
\left\| \frac{ a_{ n+1 } }{ a_n } \right\| = \frac{ \frac{ n+1 }{ e^{ n+1 } } }{ \frac{ n }{ e^{ n } }  } = \lim_{ n \to \infty} \frac{ n+1 }{ n }\cdot \frac{ e^{ n } }{ e^{ n } } = \frac{ e^{ n } }{ e^{ n }\cdot e } = \frac{ 1 }{ e } = 1\cdot \frac{ 1 }{ e } = \frac{ 1 }{ e } <1
.\] 
Which solves our (i) case and we know that the series converges. This can also be done with the integral test using
\[
\int_{ 1 }^{ \infty } \frac{ x }{ e^{ x } } dx
.\] 
and showing that it is finite. End of monday class.

\section{01/23/25 Day 2 w3}%
\label{sec:01/23/25 Final notes on 11.3}

\subsection*{Theorem}%
\label{sub:Theorem}
\paragraph{The p-series of $ \sum_{ i=1 } ^{ \infty } \frac{ 1 }{ i^{ p } }  $ converges for p>1. This is basically exactly the same as p integrals and is proved using them either way.}

\subsection{Section 11.4 The comparison test}%
\label{sub:Section 11.4 The comparison test}
\paragraph{Preamble:}
If we let $ a_n,b_n >0 $, then,
\paragraph{(i)} If $ \sum_{  } ^{  } b_n $ converges and $ a_n \le b_n $, then $ \sum_{  } ^{  } a_n $ also converges. 
\paragraph{(ii)} If $ \sum_{  } ^{ } b_n $ diverges and $ b_n \le a_n $, then $ \sum_{  } ^{  } a_n $ also diverges. \\
This can be used in the same way that comparisons can be proved in integrals. This is done by just finding a simpler series to compare with because we will only need this for very complex series that cannot be proven using our other known methods. 

\paragraph{Ex.}
\[
\sum_{  } ^{  } \frac{ 1 }{ n^2+6n+13 } 
.\] 
For this we can use integration using the integral test, but integration would be technical and not the easiest thing ever. (completing the square and trig sub), so instead we can use our comparison test. Start with an assumption. This can be done by looking at our function as a function with dominance applied to it. Using dominance we get $ \frac{ 1 }{ n^2+6n+13 } \approx \frac{ 1 }{ n^2 }  $ and because $ p>1 $ this should converge. Knowing this we should find a series that is LARGER to compare with. (because if we have a larger function then we know everything less will converge) So we start by removing some terms from our original summation. \\

Removing terms from the denominator we get
\[
\sum_{  } ^{  } \frac{ 1 }{ n^2+6n+13 } < \sum_{  } ^{  } \frac{ 1 }{ n^2+6n } <\sum_{  } ^{  } \frac{ 1 }{ n^2 } 
.\] 

This shows us the method of proving that our function is smaller and now that it's at our formula we can stop because it's proven using our integrals from calc2. Again as a last note, if we are trying to prove convergence, we need to find a larger and simpler function, but if we are trying to prove divergence we find a smaller function that is also simpler.

\paragraph{Ex. Given}
\[
\sum_{ n=1 } ^{ \infty } \frac{ n }{ n^2-\cos^{ 2 } \left( n \right)  } 
.\] 
\paragraph{Prove divergence or convergence. \\}

Again, we first make an assumption, let's look at our function and notice that cos will just bounce between 1 and -1, so we can ignore it. Now using dominance we find that our function will approximate to $ \sum_{  } ^{  } \frac{ 1 }{ n }  $ which is a harmonic series and divergent. This is also a p value of 1 so, we know it won't work. Now that we guessed divergence, we know we want to find something that is smaller and simpler. We start by reconstructing our function with cos and create an inequality.

\begin{gather*}
0<\cos^{ 2 } \left( n \right) <1\\
0\ge -1\cos^{ 2 } \left( n \right) \ge -1 \\
n^2\ge n^2-\cos^{ 2 } \left( n \right) \ge n^2-1 \\
\frac{ n }{ n^2 } \le \frac{ n }{ n^2-\cos^{ 2 } \left( n \right)  } \le \frac{ n }{ n^2-1 } \\
\sum_{ n=1 } ^{ \infty } \frac{ n }{ n^2 } \le \sum_{ n=1 } ^{ \infty }  \frac{ n }{ n^2-\cos^{ 2 } \left( n \right)  } \le \sum_{ n=1 } ^{ \infty } \frac{ n }{ n^2-1 } \\
\end{gather*}
This technique is called construction and can be done for proving all our comparisons but may be more complex than other ways. 

\paragraph{EX.}
\[
\sum_{ n=1 } ^{ \infty } \frac{ n^3+3 }{ n^{ 5 }+6 }
.\] 
\paragraph{Prove convergence or divergence. \\}
This will behave similar to $ \sum_{ n=1 } ^{ \infty } \frac{ n^3 }{ n^{ 5 } } $ which will converge because $ p>1 $. Now we know we want to find something bigger, so lets start by throwing away some terms. 

\[
<\sum_{ n=1 } ^{ \infty } \frac{ n^3+3 }{ n^{ 5 } }
.\] 
Given the above sum we can just split the sum into two fractions and prove it like that
\[
\sum_{ n=1 } ^{ \infty } \frac{ n^3 }{ n^{ 5 } } +\sum_{ n=1 } ^{ \infty } \frac{ 3 }{ n^{ 5 } } \implies \sum_{ n=1 } ^{ \infty } \frac{ 1 }{ n^2 } +3\sum_{ n=1 } ^{ \infty } \frac{ 1 }{ n^{ 5 } } 
.\] 
Because both of our sums converge, we know that our actual sum will also converge. (the above converges based on p values and can be cut down to the theorem from before that a convergent sum + a convergent sum will also be convergent) 
\subsection*{Limit comparison test}%
\label{sub:Limit comparison test}
Let $ a_n, b_n > 0 $,
\paragraph{(i)}
If $ 0<\lim_{ n \to \infty} \frac{ a_n }{ b_n }<\infty  $ then either, both $ \sum_{  } ^{  } a_n, \sum_{  } ^{  } b_n $ converge, or both diverge. It cannot be interchanged. This means that you cannot have $ a_n $ be convergent and $ b_n $ be divergent. 

\paragraph{Ex.}
\[
\sum_{ k=1 } ^{ \infty } \frac{ 4k^2+k+2 }{ 3k^3+9 }
.\] 
\paragraph{Prove convergence or divergence.\\}
In a case like this normal comparison would be difficult so let's use our limit comparison test. First make an assumption using dominance, so, $ a_n \approx \frac{ 4k^2 }{ 3k^3 } \implies \frac{ 4 }{ 3 } \sum_{  } ^{  } \frac{ 1 }{ k } $. Now that we see that we have a harmonic, we can use $ \frac{ 4 }{ 3k }  $ as our $ b_n $. (This would also work with its reciprocol so $ \lim_{ n \to \infty} \frac{ b_n }{ a_n } )$. 

So let's take our limit,
\[
\lim_{ n \to \infty} \frac{ a_n }{ b_n }=\lim_{ n \to \infty} \frac{ 4k^2+k+2 }{ 3k^3+9 }\cdot \frac{ 3 }{ 4 } \cdot k
.\] 
Distributing our terms we will get a $ 12k^3 $ function on the top and bottom so our resulting limit will be finite by dominance. Because $ 0<\lim_{ n \to \infty} \frac{ a_n }{ b_n }<\infty $ is proven true, we know that we have either convergence or divergence for both. And because we already showed that $ a_n $ diverges by dominance to the harmonic, we know that $ b_n $ will also be divergent and our whole function will be divergent. 

\documentclass[a4paper]{article}

\usepackage[utf8]{inputenc}
\usepackage[T1]{fontenc}
\usepackage{textcomp}
\usepackage[english]{babel}
\usepackage{amsmath, amssymb}
\title{Review for exam 3}
\author{yawnbo}
\date{\today}

\maketitle

\pdfsuppresswarningpagegroup=1

\begin{document}
\paragraph{Question 5}
\[
\int_{}^{} e^{\sqrt[3]{x}}dx \text{ using u sub  }
.\] 
\[
u=x^{\frac{1}{3}} \text{ an } du=\frac{1}{3}x^{-\frac{2}{3}}dx
.\] 
Solving for dx gets 
\[
dx= \frac{3du}{x^{-\frac{2}{3}}}=3x^{\frac{2}{3}}du=3\left( x^{\frac{1}{3}} \right) ^2du=3u^2du
.\] 
Which can then be tabulated to
\[
3u^2e^{4}-6ue^{u}+6u+C
.\] 
and subbing back in x
\[
3x^{\frac{2}{3}}e^{\sqrt[3]{x}}-6^{\sqrt[3]{x}}e^{\sqrt[3]{x}}+6\sqrt[3]{x}+C
.\] 
\section{Comparison examples from yesterday}%
\label{sec:Comparison examples from yesterday}

\begin{align}
 &\int_{1}^{\infty} \frac{dx}{\sqrt{x}+e^{3x}} \\
 &\int_{0}^{0.5} \frac{dx}{x^{8}+x^2} \\
 &\int_{1}^{\infty} \frac{dx}{\sqrt{x^{5}+2}} \\
 &\int_{0}^{5} \frac{dx}{x^{\frac{1}{3}}+x^3} \\
 &\int_{0}^{\infty} \frac{dx}{\sqrt{x^{\frac{1}{3}+x^3}}} \\
 &\int_{1}^{\infty} \frac{dx}{x^{4}+e^{x}}\\
\end{align}
Generally guess which one we would assume we have, for one assume convergence, so we need somthing larger.
\newpage
\subsection{For number 5}%
\label{sub:For number 5}
\paragraph{This is a type 1 and two improper.}
\[
\int_{0}^{\infty} \frac{dx}{\sqrt{x^{\frac{1}{3}}+x^3}}
.\] 
becomes
\[
\int_{0}^{1} \frac{dx}{\sqrt{x^{\frac{1}{3}}+x^3}} + \int_{1}^{\infty} \frac{dx}{\sqrt{x^{\frac{1}{3}}+x^3}}
.\] 
By dominance theory we can look at the second integral of $x^3$ which converges by p theorem. 
Now we look at the first one to establish its divergence so we need a function thats bigger
\[
x^{\frac{1}{3}}+x^3 \le 2x^{\frac{1}{3}}
.\] 
We choose that number because when testing $x^3+x^3$ we find that just one of them is smaller than $x^{\frac{1}{3}}$ which means that if we are finding a larger function we should take our x to the power of a fraction.
Doing normal comparison we find that this converges instead and our analysis is wrong because we found a value that doesn't help us

Now we want to try something like $x^{-1}$ to get a value that is 

THiS ONE IS ON THE EXAM THIS NEEDS TO BE FIGURED OUT 
\[
\int_{0}^{\infty} \frac{dx}{\sqrt{x^{\frac{1}{3}}+x^3}}
.\] 

\section{Practices}%
\label{sec:Practices}
\[
\int_{}^{} \frac{xe^{x}}{\sqrt{1+e^{x}}}dx
.\] 
Rewrite 
\[
\int \frac{xe^{x}}{\sqrt{1+e^{x}}} dx = \int \frac{x}{\sqrt{u}} du
.\] 
go to parts
\[
\int x \cdot u^{-1/2} du
.\] 
\end{document}

\section*{11.8 Power series cont}%
\label{sec:11.8 Power series cont}
\subsection*{11.8.1 Derivatives in calc 1}%
\label{sub:11.8.1 Derivatives in calc 1}
\paragraph{When we had} some thing like \[
\frac{ d }{ dx } \left( f\left( x \right) \pm g\left( x \right)  \right) 
.\] 
we could distribute the $ \frac{ d }{ dx }  $ which shows linearity of the distribution function. So what happens when we take a $ \frac{ d }{ dx }  $ of the power series $ \sum_{ n=0 } ^{ \infty } C_n \left( x \right) ^{ n } $? This will become something like
\[
=\frac{ d }{ dx } \left( C_1 \cdot x^{ 0 }+c_2x^{ 1 }+c_2x^2+c_3x^3 \ldots c_nx^{ n }\right) 
.\] 
Where we can simplify it to 
\[
= c_1+2c_2x+3c_3x^2+4c_4x^3 \ldots nc_nx^{ n-1 } \implies \sum_{ n=1 } ^{ \infty } c_nx^{ n-1 }
.\] 
Or,
\[
\sum_{ n=1 } ^{ \infty } c_nn\left( x \right) ^{ n-1 }
.\] 
Note that this sholdn't start at $ n=0 $ because the first term will just be 0 thanks to our n multiplyer. 
\paragraph{Power Series for the function $ e^{ x } $ \\}
Start with the function $ y' = y\text{, } y\left( 0 \right) =1 $ whose solution will be $ y=e^{ x } $. Now we can call $ f'\left( x \right) = f\left( x \right)  $ or $ \sum_{ n=0 } ^{ \infty } a_nx^{ n } = a_{ 0 }+a_1x + a_2x^2 + a_3x^3 \ldots$ This sum can now be derived to get 
\[
f'\left( x \right) = \sum_{ n=1 } ^{ \infty } a_n \cdot nx^{ n-1 }= a_1\left( 1 \right) +a_2\left( 2x \right) +a_3\left( 3x^2 \right) \ldots
.\] 
Where we can now set terms equal to each other,
\begin{align*}
a_0 &= a_1 \\
a_1x &=  a_2\left( 2x \right)  \\
a_2x^2 &= a_3 3x^2 \\
\ldots
.\end{align*}
Now we can use algebra on each term to find
\[
a_2 = \frac{ a_1 }{ 2 }, a_3 = \frac{ a_2 }{ 3 } = \frac{ a_1 }{ 2\cdot 3 }, a_4 = \frac{ a_3 }{ 4 } = \frac{ a_1 }{ 2\cdot 3\cdot 4 } \ldots a_n = \frac{ a_1 }{ n\cdot \left( n-1 \right) \ldots 3 \cdot 2\cdot 1 }
.\] 
which gives us the power series but we still want to find $ a_1 $, so we can find the sum to cancel all terms other than $ a_0 $ to find that our $ a_n = \frac{ 1 }{ n! }  $. Which gives us the final equation,
\[
e^{ x }= \sum_{ n=0 } ^{ \infty } \frac{ 1 }{ n! } x^{ n }
.\] 
Now that we have the summation we can look at our expansion, 
\[
\sum_{ n=0 } ^{ \infty } \frac{ x^{ n } }{ n! }= 1 + \frac{ x^{ 1 } }{ 1! }+ \frac{ x^{ 2 } }{ 2! }+ \frac{ x^{ 3 } }{ 3! }+ \ldots
\] 
Which can be graphed to approximate the function $ e^{ x } $. Questions can appear using this on the next quiz where we have to find an interval of x values that will give a certain range of error like $ 10^{ -3 } $. Note that these are called transedental functions ($ e^{ x }$) and we are converting them to simple ones with our summation. A common one in physics is using x to approximate $ \sin^{  } \left( x \right)  $ for values close to 0. 

\paragraph{Ex.}
\paragraph{Prove that}
\[
\ln^{  } \left( 1+x \right) = \sum_{ n=1 } ^{ \infty } \left( -1 \right) ^{ n-1 }\frac{ x^{ n } }{ n } \text{ when } \left| x \right|<1
.\] 
Start with the function $ \sum_{ n=0 } ^{ \infty } x^{ n } $ which converges to $ \frac{ 1 }{ 1-x }  $ for the values of $ \left| x \right|< 1 $. We essentially want to build the function in order to get from $ \frac{ 1 }{ 1-x }  $ to the $ \ln^{  } \left( 1-x \right)  $. As our first step let's make the x negative,  $ \sum_{ n=0 } ^{ \infty } \left( -x \right) ^{ n } =\frac{ 1 }{ 1+x }  $. This doesn't really get us anywhere so what if we instead try something else. \\
Let's instead make it $ \frac{ 1-x^2 }{ 1-x }= 1+x $. To get us here we need to simplify $ \left( 1-x^2 \right) \left( \sum_{ n=0 } ^{ \infty } x^{ n } \right)  $. So,
\[
=\left( 1-x^2 \right) \left( 1+x+x^2+x^3\ldots \right) 
.\] 
Now looking at our sequence we can see that each multiplication by one will give us a term in the sequence. So we can simplify this to 
\[
\sum_{ n=0 } ^{ \infty } x^{ n }+\left( -x^2-x^3-x^{ 4 }-x^{ 5 }\ldots \right) = 1+x
.\] 
This gives us the inside of our $ \ln^{  } \left( 1+x \right)  $, but we still need the rest which we will come back to later. 

\subsection*{11.9}%
\label{sub:11.9}
\paragraph{The geometric series identity}
\[
\sum_{ n=0 } ^{ \infty } x^{ n }= \frac{ 1 }{ 1-x } 
.\] 
This is one of the most useful identities for making a power series because we can use it to build them easily. For example,
\[
\frac{ 1 }{ 1+2x } =\frac{ 1 }{ 1-\left( -2x \right)  } = \sum_{ n=0 } ^{ \infty } \left( -2x \right) ^{ n } = \sum_{ n=0 } ^{ \infty } \left( -1 \right) ^{ n }\left( 2x \right) ^{ n }
.\] 
\paragraph{Ex.}
\paragraph{Make the below into a power series.}
\[
\frac{ 1 }{ 2+x^2 } 
.\] 
So let's start by factoring out our two,
\[
\frac{ 1 }{ 2\left( 1+\frac{ x^2 }{ 2 } \right)  } = \frac{ 1 }{ 2\left( 1-\left( -\frac{ x }{ 2 }  \right) ^2 \right)  } = \frac{ 1 }{ 2 } \left[ \frac{ 1 }{ 1-\left( -\frac{ x^2 }{ 2 }  \right)  }  \right] = \frac{ 1 }{ 2 } \sum_{ n=0 } ^{ \infty } \left( -\frac{ x^2 }{ 2 }  \right)^{ n } 
.\] 
\[
= \frac{ 1 }{ 2 } \sum_{ n=0 } ^{ \infty } \left( -1 \right) ^{ n }\left( \frac{ x^2 }{ 2}  \right) ^{ n }
.\] 
What if we wanted to find our interval of convergence? We can start with the ratio test,
\begin{gather*}
\lim_{ n \to \infty} \left| \frac{ a_{ n+1 } }{ a_n } \right| < 1\\
\left| \frac{ \left( \frac{ x^2 }{ 2 }  \right) ^{ n+1 } }{ \left( \frac{ x^2 }{ 2 }  \right) ^{ n } } \right|<1 \to \left| \left( \frac{ x^2 }{ 2 }  \right)  \right|<1 \to \left| x^2 \right|<2\\
\end{gather*}
Which gives the interval $ -\sqrt{ 2}<x<\sqrt{ 2} $ where we can now test our intervals and easily see that they will diverge because the function doesn't go to 0. 
\paragraph{Ex.}
\[
\frac{ x }{ 16+2x^3 } 
.\] 
Lets start by removing an x and a 16,
\[
x\left( \frac{ 1 }{ 16+2x^3 }  \right) =\frac{ x }{ 16 } \left( \frac{ 1 }{ 1+\frac{ x^3 }{ 8 }  }  \right) = \frac{ x }{ 16 } \left( \frac{ 1 }{ 1-\left( -\frac{ x^3 }{ 8 }  \right)  }  \right) = \frac{ x }{ 16 } \sum_{ n=0 } ^{ \infty } \left( -\frac{ x^3 }{ 8 }  \right) ^{ n }
.\] 
Which can now be written as the alternating series,
\[
\frac{ x }{ \underbrace{ 16 }_{ 2^{ 4 } }  } \sum_{ n=0 } ^{ \infty } \frac{ \left( -1 \right) ^{ n }x^{ 3n } }{ \underbrace{ 8^{ n } }_{ 2^{ 3n } }  } = \sum_{ n=0 } ^{ \infty } \frac{ \left( -1 \right) ^{ n } x^{ 3n+1 }}{ 2^{ 3n+3 } }
.\] 
The IOC (interval of convergence) here is (-2,2) but wasn't proven in class because it would take a while. Just use ratio test if you want to prove it on our snow days. His quote not mine. Actually one more,\\
\paragraph{Ex. $ \tan^{ -1 } \left( x \right)  $. }
Using $ \int_{  }^{  } \frac{ 1 }{ 1+x^2 } dx= \tan^{ -1 } \left( x \right)  $, how can we make a power series of,
\[
\frac{ 1 }{ 1+x^2 } = \frac{ 1 }{ 1 - \left( -x^2 \right)  } = \sum_{ n=0 } ^{ \infty } \left( -1 \right) ^{ n }x^{ 2n }
.\] 
Now we want to integrate both sides, remembering that integrals are linear as well,
\[
\int_{  }^{  } \frac{ 1 }{ 1+x^2 } = \sum_{ n=0 } ^{ \infty } \left( -1 \right) ^{ n }\int_{  }^{  } x^{ 2n }dx
.\] 
Note that this happens because n is constant with respect to x so there is no point in keeping it inside the integral in this case. Now,
\[
\int_{  }^{  } \left( 1-x^2 + x^{ 4 }- x^{ 6 } + x^{ 8 }\ldots\right) 
.\] 
Which becomes,
\[
C + x-x^3 + \frac{ x^5 }{ 5 } - \frac{ x^7 }{ 7 } + \frac{ x^9 }{ 9 } \ldots
.\] 
Which can be simplified to,
\[
C+ \sum_{ n=1 } ^{ \infty } \left( -1 \right) ^{ n-1 }\frac{ x^{ 2n-1 } }{ 2n-1 } = \tan^{ -1 } \left( x \right) 
.\] 
And to find our initial condition we can plug in 0 to find that $ C=0 $ because $ \tan^{ -1 } \left( 0 \right) =0 $. Which proves that 
\[
\tan^{ -1 } \left( x \right) = \sum_{ n=1 } ^{ \infty } \left( -1 \right) ^{ n }\frac{ x^{ 2n-1 } }{ 2n-1 }
.\] 


\end{document}
