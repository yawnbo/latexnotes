\documentclass[a4paper]{article}

\usepackage{cancel}
\usepackage[utf8]{inputenc}
\usepackage[T1]{fontenc}
\usepackage{textcomp}
\usepackage[english]{babel}
\usepackage{amsmath, amssymb}

\title{Quiz 2 Calculus 3}
\author{yawnbo}
\date{\today}

\pdfsuppresswarningpagegroup=1

\begin{document}
\maketitle
\section*{Question 1 }%
\paragraph{Compute the total area of the (infinitely many) triangles.\\}
Start by noticing that the area of the triangles is given by $ \frac{ 1 }{ 2 } bh $. Knowing that our height is constant at $ \frac{ 1 }{ 2 }  $ we just need to find b. Start by writing the sum that matches the sequence $ \frac{ 1 }{ 2 } ,\frac{ 1 }{ 4 },\frac{ 1 }{ 8 } \ldots $ and simplifying it to use our geometric formula. 
\[
	\sum_{ n=1 } ^{ \infty } \left( 1 \right) \left( \frac{ 1 }{ 2 }  \right) ^{ n }\to \frac{ 1 }{ 2 } \sum_{ n=1 } ^{ \infty } \left( 1 \right) \left( \frac{ 1 }{ 2 }  \right) ^{ n-1 } \implies b = \lim_{ n \to \infty} \frac{ \frac{ 1 }{ 2 }\left( 1- \cancel{ \left(  \frac{ 1 }{ 2 } \right) ^{ n }}\right)  }{ 1-\frac{ 1 }{ 2 }  }=\frac{ \frac{ 1 }{ 2 }  }{ \frac{ 1 }{ 2 }  }=1
.\] 
Which leaves us with $ \frac{ 1 }{ 2 } \left( \frac{ 1 }{ 2 }   \cdot 1 \right)  $ or just $ \frac{ 1 }{ 4 }  $ as the sum of the areas of the triangles.
\section*{Question 2 }%
\paragraph{If $ \sum_{  } ^{  } a_n $ is a convergent series, prove, using any theorem presented in class so far, that $ \sum_{  } ^{  } \frac{ 1 }{ a_n }  $ diverges. \\}

To start, we can use the theorem that states that if a sum $ \sum_{ n=1 } ^{ \infty } a_n $ converges, then $ \lim_{ n \to \infty} a_n=0 $. Knowing this, we can take the limit of $ \frac{ 1 }{ a_n }  $ to find that as $ n \to 0 $ our new series ($ b_n $) will have a limit of $ \lim_{ n \to \infty} b_n=\infty $ which is divergent and conflicts with our theorem that an infinite sum must have a series whose limit is 0.
\newpage
\section*{Question 3 }%
\paragraph{Prove whether $ \sum_{ n=1 } ^{ \infty } \ln^{  } \left( 1+\frac{ 1 }{ n }  \right)  $ is convergent or divergent. Hint: Try to use a similar argument that was used in class for the harmonic series \\ }
We can start by rewriting the sum to simplify the terms as $ \sum_{ n=1 } ^{ \infty } \ln^{  } \left( n+1 \right) -\sum_{ n=1 } ^{ \infty } \ln^{  } \left( n \right)  $. Now taking a partial sum of the $ n^{ th } $ term and cancelling,
\begin{align*}
	S_n = \cancel{\ln^{  } \left( 2 \right)} + \cancel{ \ln^{  } \left( 3 \right)  } - \cancel{\ln^{  } \left( 2 \right)} \ldots \cancel{ \ln^{  } \left( n \right)  } - \cancel{ \ln^{  } \left( n-1 \right)  } + \ln^{  } \left( n+1 \right) - \cancel{ \ln^{  } \left( n \right)  }
.\end{align*}
We find that we are left with $ \ln^{  } \left( n+1 \right)  $ as our series goes to $ n $, and taking the limit of this gets us $ \lim_{ n \to \infty} \ln^{  } \left( n+1 \right) =\infty $ because $ \ln^{  } \left( n \right)  $ will grow without bound when going to infinity, proving that our series is divergent because it does not approach a finite number, nevertheless does it go to 0.
\section*{Question 4}%
\paragraph{For the series $ \sum_{ n=1 } ^{ \infty } a_n $, its $ n^{ th } $ partial sum is given by $ S_n = \frac{ n-1 }{ n+1 } $. \\ \\ Provide an expression for $ a_n $ and compute $ \sum_{ n=1 } ^{ \infty } a_n $\\}
Start by finding an expression for $ a_n $ by taking the difference of the partial sums,
\[
	a_n = S_n - S_{ n-1 } = \left( \frac{ n-1 }{ n+1 } - \frac{ n-2 }{ n }\right) = \frac{ \cancel{n^2} - \cancel{n} - \cancel{n^2}+\cancel{ n }+2 }{ n\left( n+1 \right)  } = \frac{ 2 }{ n\left( n+1 \right)  }
.\] 
Now using our given formula as $ n\to \infty $ to find the infinite sum,
\[
\lim_{ n \to \infty} S_n = \lim_{ n \to \infty} \frac{ n-1 }{ n+1 } = \lim_{ n \to \infty} \frac{ n }{ n } = 1
.\] 
\section*{Question 5}%
\paragraph{Solve the equation $ \sum_{ n=0 } ^{ \infty } e^{ nc } =10 $ for c. \\}
We can start by rewriting the sum as a geometric series,
\[
\sum_{ n=0 } ^{ \infty } e^{ nc } = \sum_{ n=1 } ^{ \infty } \left( e^{ c } \right)^{ n-1 } = \frac{ 1 }{ 1-e^{ c } } = 10 
.\] 
Now we can solve for $ c $ with algebra to find, $ c = \ln^{  } \left( \frac{ 9 }{ 10 }  \right)  $.
\newpage
\section*{Question 6}%
\label{sec:Question 6}
\paragraph{Use the identity: $ \sum_{ n=1 } ^{ \infty } \frac{ 1 }{ n^2 } =\frac{ \pi^2 }{ 6 } $ to find the exact value of:}
\[
\sum_{ n=3 } ^{ \infty } \frac{ 1 }{ \left( n+1 \right) ^2 } 
.\] 
Start by reindexing the series to match the identity and finding another sum for the first few terms, (let $ m=n+1 $)
\[
\sum_{ m=4 } ^{ \infty } \frac{ 1 }{ m^2 } = \frac{ \pi^2 }{ 6 }-\sum_{ m=1 } ^{ 3 } \frac{ 1 }{ m^2 } 
= \frac{ \pi^2 }{ 6 } - \left( 1+\frac{ 1 }{ 4 } +\frac{ 1 }{ 9 } \right) = \frac{ \pi^2 }{ 6 } - \frac{ 49 }{ 36 } = \frac{ 6 \pi^2-49 }{ 36 }
.\] 
Leaving $ \frac{ 6 \pi^2-49 }{ 36 } $ as the result of the given sum.
\section*{Question 7}%
\label{sec:Question 7}
\paragraph{Determine the interval of convergence for }
\[
\sum_{ n=0 } ^{ \infty } 3^{ n }\left( x-2 \right) ^{ -n }
.\] 
Start by writing our sum as a geometric series,
\[
\sum_{ n=0 } ^{ \infty } \frac{ 3^{ n } }{ \left( x-2 \right) ^{ n } }=\sum_{ n=0 } ^{ \infty } \left( \frac{ 3 }{ x-2 } \right) ^{ n }
.\] 
Now that this matches our geometric series formula, we can find the interval of convergence by finding the absolute value of the r value,
\[
\left| \frac{ 3 }{ x-2 }  \right| < 1 \implies \left| 3 \right| < \left| x-2 \right| \implies 3 < \left| x-2 \right| \implies x < -1 \text{ or }x > 5
.\] 
So our interval of convergence would be
\[
	\left( -\infty,-1 \right) \cup \left( 5,\infty \right) 
.\] 

\end{document}
