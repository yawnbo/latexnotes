\documentclass{report}

\input{../../../../pkgs/preamble}
\input{../../../../pkgs/macros}
\input{../../../../pkgs/letterfonts}

\title{\Huge{Calc3}\\ Quiz 4 }
\author{\huge{yawnbo}}
\date{\today}

\begin{document}

\maketitle
\newpage% or \cleardoublepage
% \pdfbookmark[<level>]{<title>}{<dest>}
\pdfbookmark[section]{\contentsname}{toc}

\pagebreak

\section*{Question 1}%
\label{sec:Question 1}
\paragraph{Determine the interval of convergence for $ \sum_{ n=1 } ^{ \infty } \frac{ \left( 2x-1 \right) ^{ n } }{ 5^{ n }\sqrt{ n} } $. Also state its radius of convergence. \\ \\}
We can start by applying root test to find our interval,
\[
\lim_{ n \to \infty} \sqrt[ n ]{ \left| \frac{ \left( 2x-1 \right) ^{ n } }{ 5^{ n }\sqrt{ n} } \right| } = \underbrace{ \lim_{ n \to \infty} \frac{ 1 }{ \sqrt[ 2n ]{ n }  } }_{ 1 }  \left| \frac{ 2x-1 }{ 5 } \right|<1
.\] 
\begin{gather*}
	 \left| \frac{ 2x-1 }{ 5 } \right|<1 \\
	-5 < 2x-1 < 5 \\
	-2 < x < 3
.\end{gather*}
Now testing endpoints,
\[
\sum_{ n=1 } ^{ \infty } \frac{ \left( 2\left( -2 \right) -1 \right) ^{ n } }{ 5^{ n }\sqrt{ n} } \to \sum_{ n=1 } ^{ \infty } \left( -1 \right) ^{ n }\frac{ 1 }{ \sqrt{ n } }
.\] 
By AST, because $ \lim_{ n \to \infty} \frac{ 1 }{ \sqrt{ n }  } \to 0 $ our sum at $ -2 $ will converge. Now 3,
\[
\sum_{ n=1 } ^{ \infty } \frac{ \left( 5 \right) ^{ n } }{ 5^{ n }\sqrt{ n }  } = \sum_{ n=1 } ^{ \infty } \frac{ 1 }{ \sqrt{ n }  } 
.\] 
By p-test this series will diverge because $ P<1 $. Getting our IOC to be $ -2\le x < 3 $ and our radius of convergence to be $ \frac{ 3-\left( -2 \right)  }{ 2 }= \frac{ 5 }{ 2 } =R $. 

\section*{Question 2}%
\label{sec:Question 2}
\paragraph{Demonstrate using differentiation and integration to construct the power series for $ f\left( x \right) = x\ln^{  } \left( 1+x^2 \right)  $. Afterwards state its interval of convergence. \\ \\ }
Start by rewriting our function,
\[
x\ln^{  } \left( 1+x^2 \right) \to x\left( \frac{ d }{ dx } \ln^{  } \left( 1+x^2 \right)  \right) \to x\left( \frac{ 1 }{ 1- \left( -x^2  \right) } \cdot 2 \right) \to x\left( 2x\sum_{ n=1 } ^{ \infty } \left( -1 \right) ^{ n }x^{ 2n+1 } \right) \to 2x\sum_{ n=1 } ^{ \infty } \left( -1 \right) ^{ n }\frac{ x^{ 2n+2 } }{ 2n+2 } 
.\] 
Leaving us with,
\[
\to \sum_{ n=1 } ^{ \infty } \left( -1 \right) ^{ n }\frac{ x^{ 2n+3 } }{ n+1 }
.\] 
Finding the IOC we can use ratio test,
\[
\lim_{ n \to \infty} \left| \frac{ a_{ n+1 } }{ a_n } \right| = \lim_{ n \to \infty} \left| \frac{ x^{ 2n+5 } }{ n+2 } \cdot \frac{ n+1 }{ x^{ 2n+3 } } \right| =
\underbrace{ \lim_{ n \to \infty} \frac{ n+1 }{ n+2 } }_{ 1 } \cdot \left| x^2 \right|< 1\text{ or }-1<x<1
.\] 
Testing endpoints we see our sums become $ \sum_{ n=1 } ^{ \infty } \left( -1 \right) ^{ n }\frac{ -1 }{ n+1 } $ and $ \sum_{ n=1 } ^{ \infty } \left( -1 \right) ^{ n } \frac{ 1 }{ n+1 }$ which can both be treated as harmonics by AST and will converge because the harmonic has a $ \lim_{ n \to \infty}  $ of 0 and is monononic decreasing. Thus our IOC is $ -1\le x\le 1 $.
\newpage
\section*{Question 3}%
\paragraph{Construct the Taylor Series for $ f\left( x \right) =\sin^{  } \left( x \right)  $ centered at $ c=\pi $. Afterwards, state the domain for the series.\\ \\ }
Writing out our derivatives,
\begin{align*}
	f\left( x \right) &=\sin^{  } \left( x \right) \to 0\\
	f^{ 1 }\left( x \right) &= \cos^{  } \left( x \right) \to -1 \\
	f^{ 2 }\left( x \right) &= -\sin^{  } \left( x \right) \to 0 \\
	f^{ 3 }\left( x \right) &= -\cos^{  } \left( x \right) \to 1 \\
	f^{ 4 }\left( x \right) &= \sin^{  } \left( x \right) \to 0 
.\end{align*}
By Taylor's theorem, if we ignore 0's our sequence will be,
\[
-1, \frac{ 1 }{ 3! }, -\frac{ 1 }{ 5! } , \frac{ 1 }{ 7! } \ldots
.\] 
Which can be written as $ \sum_{ n=0 } ^{ \infty } \left( -1 \right) ^{ n }\frac{ 1 }{ \left( 2n+1 \right) ! } \left( x-\pi \right) ^{ 2n+1 } $. The domain of this series will be all real numbers because the sine function is periodic and will repeat itself every $ 2\pi $.
\section*{Bonus question}%
\paragraph{a) Let $ L_n $ be the perimeter of $ I_n $. Show that $ \lim_{ n \to \infty} L_n = \infty$}
\paragraph{b) Let $ A_n $ be the area of $ I_n $. Find $ \lim_{ n \to \infty} A_n  $; i.e., find the area of the fractal. \\ \\}
We can start by getting some ratios of our area and only focusing on how it increases after $ I_0 $. The area of each new triangle when $ a_0 = \frac{ \sqrt{ 3 }  }{ 4 } $ should be $ a_0 \sum_{ n=1 } ^{ \infty } \frac{1}{ 9^{ n } }  $ and we should have 4 times as many triangles with each itteration after the first one (which was instead 3), so, $ 3 \cdot 4^{ n-1 } = \frac{ 3 }{ 4 } \cdot \sum_{ n=1 } ^{ \infty } 4^{ n } $ is the amount of triangles we need to calculate area for, which is given by,
 \[
a_0\sum_{ n=1 } ^{ \infty }\frac{ 1 }{ 9^{ n } }\cdot \frac{ 3 }{ 4 } \sum_{ n=1 } ^{ \infty } \cdot 4^{ n } \to \frac{ 3 }{ 4 } a_0 \sum_{ n=1 } ^{ \infty } \left( \frac{ 4 }{ 9 }  \right)^{ n }
.\] 
Because this sum is only accounting for the increase after the first triangle, our sum should be adding on $ a_0 $ to get the actual area, and we can write this as the sum,
\[
	a_0 + \frac{ 3 }{ 4 } a_0 \sum_{ n=1 } ^{ \infty } \left( \frac{ 4 }{ 9 }  \right)^{ n } \to a_0\left(   1+\frac{ 1 }{ 3 } \sum_{ n=0 } ^{ \infty } \left( \frac{ 4 }{ 9 }  \right) ^{ n }\right)
.\] 
Which can now be easily evaluated as $ a_0\left( 1+\frac{ 1 }{ 3 } \cdot \frac{ 1 }{ 1-\frac{ 4 }{ 9 } } \right) = a_0\left( 1+\frac{ 1 }{ 3 } \cdot \frac{ 9 }{ 5 } \right) = a_0\left( 1+\frac{ 3 }{ 5 } \right) = \frac{ 8 }{ 5 } \cdot \frac{ \sqrt{ 3 }  }{ 4 } = \frac{ 2\sqrt{ 3 }  }{ 3 }$. 
\end{document}
