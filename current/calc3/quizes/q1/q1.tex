\documentclass[a4paper]{article}

\usepackage[utf8]{inputenc}
\usepackage[T1]{fontenc}
\usepackage{textcomp}
\usepackage[english]{babel}
\usepackage{amsmath, amssymb}

\title{Quiz 1 Calculus 3}
\author{yawnbo}
\date{\today}

\pdfsuppresswarningpagegroup=1

\begin{document}
\maketitle
\section*{Question 1 }%
\label{sec:Question 1 }
\begin{gather*}
\text{ For a sequence, }a_n = \frac{ \left( -1 \right) ^{ n+1 }n }{ n+\sqrt{ n} } \\ 
\text{ form an argument on whether the sequence converges or diverges. }
\end{gather*}
For this we can start by writing out the first few terms of the sequence to find a pattern, $ a_1 = \frac{1}{2}, a_2 = -\frac{2}{2+\sqrt{ 2}}, a_3 = \frac{3}{3+\sqrt{ 3}}, a_4 = -\frac{4}{6}  $. Which shows that we have an ossilation between evens and odds because of $ \left( -1 \right) ^{ n+1 } $. Because this happens on even intervals we can use subsequences to show that they will diverge. 
To start, let our subsequences be
\[
\lim_{ n \to \infty} a_{ 2n } = L \text{ and } \lim_{ n \to \infty} a_{ 2n+1 }=L
.\] 
or
\begin{align*}
	a_{ 2n }=\frac{ \left( -1 \right) ^{ 2n+1 }2n }{ 2n+\sqrt{ 2n} } \text{ and } a_{ 2n+1 }=\frac{ \left( -1 \right) ^{ 2n+2 }2n+1 }{ 2n+1+\sqrt{ 2n+1} }
.\end{align*}
Because we set our terms to always be even or odd, we can predict how $ \left( -1 \right) ^{ f+1 } $ will behave. So when,
\[
f=2n \implies \left( -1 \right) ^{ 2n+1 }=-1 \text{ and } f=2n+1 \implies \left( -1 \right) ^{ 2n+2 }=1
.\]
Rewriting our equations we get,
\[
a_{ 2n }=\frac{ -2n }{ 2n+\sqrt{ 2n} } \text{ and } a_{ 2n+1 }=\frac{ 2n+1 }{ 2n+1+\sqrt{ 2n+1} }
.\] 
Where we can use Dominance Theory to find our limits.
\[
\lim_{ n \to \infty} a_{ 2n }\approx \lim_{ n \to \infty} \frac{ -2n }{ 2n }=-1 \text{ and } \lim_{ n \to \infty} a_{ 2n+1 }\approx \lim_{ n \to \infty} \frac{ 2n }{ 2n }=1
.\] 
Because these two limits do not converge to the same value, we can say that the sequence will diverge.

\newpage
\section*{Question 2}%
\label{sec:Question 2}

\begin{gather*}
\text{ For a sequence, }a_n = \frac{ \left( 2n-1 \right) ! }{ \left( 2n+1 \right) ! }\\ 
\text{ form an argument on whether the sequence converges or diverges. }
\end{gather*}
We can start to rewrite the sequence using factorial laws,
\[
	\left( 2n-1 \right) ! \to \frac{ 2n! }{ 2n } \text{ and } \left( 2n+1 \right) ! \to 2n!\left( 2n+1 \right) 
.\] 
Simplifying our sequence we get,
\[
\frac{ \frac{ 2n! }{ 2n } }{ 2n!\left( 2n+1 \right)  }\to \frac{ 2n! }{ 2n!\left( 2n+1 \right) \left( 2n \right)  }\to \frac{1}{\left( 2n+1 \right) \left( 2n \right) }
.\] 
Which we can now directly take the limit of our sequence to find that it will converge to 0.
\[
\lim_{ n \to \infty} \frac{1}{\left( 2n+1 \right) \left( 2n \right) }= 0
.\] 
\section*{Question 3}%
\label{sec:Question 3}

\paragraph{Produce an explicit formula for the sequence: $ \frac{1}{2},\frac{2}{4},\frac{6}{8},\frac{24}{16},\ldots $}

\[
\left\{ a_n \right\} _{ n=1 }^{ \infty }=\frac{ n! }{ 2^{ n } }
.\] 

\section*{Question 4}%
\paragraph{If $ \left\{ a_n \right\}  $ converges, then $ \left\{ a_n \right\}  $ is bounded. Determine if the following statements are true or false. If the statement is false, give an example of a sequence that shows it is false (known as a counter example)}


\subsection*{(i) If $ a_n $ is bounded, then it converges.}%
False, we can have sequences such as $ \left\{ a_n \right\} _{ n=1 }^{ \infty }=\left( -1 \right) ^{ n } $ that are bounded but occilate between values and do not converge. 
\subsection*{(ii) If $ a_n $ is not bounded, then it diverges.}%
True, if a sequence is unbounded then a limit does not exist and $ a_n $ will grow or fall without constraint. 
\subsection*{(iii) If $ a_n $ diverges, then it is not bounded.}%
True, if a sequence diverges then it will grow or fall without constraint and will not be bounded.

\section*{Question 5}%
\[
\text{ Let }b_n = \frac{ \sqrt[ n ]{ n! }  }{ n }. 
.\] 
\subsection*{(i) Show that $ \ln^{  } \left( b_n \right) =\frac{ 1 }{ n } \sum_{ i=1 } ^{ n } \ln^{  } \left( \frac{ i }{ n }  \right)$}%
Start by taking the ln of $ b_n $,
\[
\ln^{  } \left( b_n \right) = \ln^{  } \left( \frac{ \sqrt[ n ]{ n! }  }{ n }  \right) = \ln^{  } \left( \sqrt[ n ]{ n! }  \right) - \ln^{  } \left( n \right) = \frac{ 1 }{ n } \ln^{  } \left( n! \right) -\ln^{  } \left( n \right) 
.\] 
Simplifying both of our terms gets us
\begin{gather*}
\ln^{  } \left( n! \right) = \ln^{  } \left( 1\cdot 2\cdot 3\cdot 4\cdot \ldots n \right) \to \ln^{  } \left( n! \right) = \ln^{  } \left( 1 \right) + \ln^{  } \left( 2 \right) + \ln^{  } \left( 3 \right) + \ldots + \ln^{  } \left( n \right) \\ 
\text{ or } \sum_{ i=1 } ^{ n } \ln^{  } \left( i \right)
\end{gather*}
and because we want to simplify in the sum at some point we can take the sum of the ln of $ i $ and divide by $ n $ to get
\[
\sum_{ i=1 } ^{ n }\ln^{  } \left( n \right) =n\ln^{  } \left( n \right) \implies \frac{ 1 }{ n } \sum_{ i=1 } ^{ n } \ln^{  } \left( n \right) =\ln^{  } \left( n \right)  
.\] 
So, we have
\[
\frac{ 1 }{ n } \sum_{ i=1 } ^{ n } \ln^{  } \left( i \right) - \frac{ 1 }{ n } \sum_{ i=1 } ^{ n } \ln^{  } \left( n \right) \implies \frac{ 1 }{ n } \sum_{ i=1 } ^{ n } \left( \ln^{  } \left( i \right) -\ln^{  } \left( n \right)  \right) \text{ or } \frac{ 1 }{ n } \sum_{ i=1 } ^{ n } \ln^{  } \left( \frac{ i }{ n }  \right) 
.\] 
\newpage
\subsection*{(ii) Show that $ \lim_{ n \to \infty} \ln^{  } \left( b_n \right) =\int_{ a }^{ b } f\left( x \right) dx$; decide what $ a,b,f\left( x \right)  $ should be. Hint: Think Riemann Sum.}%
Start with our previous sum, $  \frac{ 1 }{ n } \sum_{ i=1 } ^{ n } \ln^{  } \left( \frac{ i }{ n }  \right) $ and find that $ x_i =\frac{ i }{ n }  $. Looking at $ \frac{ i }{ n }  $ as $ n \to \infty $ we can see that we will at max have a $ x_i $ of 1 and a min of 0. We can use these as our bounds for the integral and find our f(x). Because we know our limits, we can find $ \Delta x $ to be $ \frac{ \left( 1-0 \right)  }{ n } $. This matches with our $ \frac{ 1 }{ n } $ in our sum. Using the definition of an integral we can rewrite our sum as

\[
\lim_{ n \to \infty} \sum_{ i=1 } ^{ n } \ln^{  } \left( x_i \right) \cdot \frac{ 1 }{ n } \to \int_{ 0 }^{ 1 } \ln^{  } \left( x \right) dx
.\] 

\subsection*{(iii) Deduce the exact result of $ \lim_{ n \to \infty} b_n $.}%
Because we already found what $ \lim_{ n \to \infty} \ln^{  } \left( b_n \right)  $ is, we can just simplify further and cancel the ln at the end. To start we can find the result of our integral,
\[
\int_{ 0 }^{ 1 } \ln^{  } \left( x \right) dx = \left[ x\ln^{  } \left( x \right) -x \right] _{ 0 }^{ 1 } = 1\ln^{  } \left( 1 \right) -1-0\ln^{  } \left( 0 \right) = 0-1 = -1
.\] 
and now we can just raise $ e $ to the power of both sides to cancel our ln,
\[
\lim_{ n \to \infty} \ln^{  } \left( b_n \right) = -1 \implies \lim_{ n \to \infty} e^{ \ln^{  } \left( b_n \right)  }=e^{ -1 } \implies \lim_{ n \to \infty} b_n = e^{ -1 } = \frac{ 1 }{ e } 
.\] 
Leaving us with $ \frac{ 1 }{ e }  $ as the result of $ \lim_{ n \to \infty} b_n $.
\newpage
\section*{Question 6}%
\paragraph{For the recursive sequence, $ a_0 =0$ and for $ n\ge 1, a_{ n+1 }=\sqrt{ 2+ a_n} $. Assume that the sequence converges. What will be the exact value of $ \lim_{ n \to \infty} a_{ n } $? \\ \\ }
Because we can assume that our limit will exist, we can go straight to finding our limit. Assuming that $ L=\lim_{ n \to \infty} a_n $ and $ L=\lim_{ n \to \infty} a_{ n+1 } $, we can start to rewrite our sequence. 
\[
\lim_{ n \to \infty} a_{ n+1 } = \sqrt{ 2+a_n} \implies L = \sqrt{ 2+L } \implies L^2-L-2= 0 \to L=2 \text{ or } L=-1
.\] 
We ignore $ L=-1 $ because it is not possible for $ a_n $ to be negative, so, we are left with $ \lim_{ n \to \infty} a_n =2 $.

\section*{Question 7}%
\paragraph{Each layer of protective film blocks $ \frac{ 1 }{ 3 }  $ of the ultra violet rays reaching it. What portion of UV rays are transmitted through Layer1, Layer 2, Layer 3? Find a general rule to describe the portion of UV rays transmitted through n layers. \\ \\}
I first started with $ a_0 =1 \text{ and }a_n=a_{ n-1 }-\frac{ 1 }{ 3 } \left( a_{ n-1 } \right) \implies a_n= \frac{ 2a_{ n-1 } }{ 3 } $, but then wrote out a few terms and realized it could be written with just \[
\left\{ a_n \right\} _{ n=0 }^{ \infty }= \frac{ 2^{ n } }{ 3^{ n } }\text{ where n is the number of layers}
.\] 
\section*{Question 8}%
\paragraph{Give an example of divergent sequences $ \left\{ a_n \right\}  $ and $ \left\{ b_n \right\}  $ such that $ \left\{ a_n + b_n \right\}  $ converges. }
\[
\left\{ a_n \right\} _{ n=1 }^{ \infty }=\left( -1 \right) ^{ n }\text{ and }\left\{ b_n \right\} _{ n=1 }^{ \infty }=\left( -1 \right) ^{ n+1 }
.\] 



\end{document}
