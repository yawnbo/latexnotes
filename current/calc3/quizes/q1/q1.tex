\documentclass[a4paper]{article}

\usepackage[utf8]{inputenc}
\usepackage[T1]{fontenc}
\usepackage{textcomp}
\usepackage[english]{babel}
\usepackage{amsmath, amssymb}

\title{Quiz #1 Calculus 3}
\author{yawnbo}
\date{\today}
\maketitle

\pdfsuppresswarningpagegroup=1

\begin{document}
\section*{Question #1 }%
\label{sec:Question #1 }
\begin{gather*}
\text{ For a sequence, }a_n = \frac{ \left( -1 \right) ^{ n+1 }n }{ n+\sqrt{ n} } \\ 
\text{ form an argument on whether the sequence converges or diverges. }
\end{gather*}
For this we can start by writing out the first few terms of the sequence to find a pattern, $ a_1 = \frac{1}{2}, a_2 = -\frac{2}{2+\sqrt{ 2}}, a_3 = \frac{3}{3+\sqrt{ 3}}, a_4 = -\frac{4}{6}  $. Which shows that we have an ossilation between evens and odds because of $ \left( -1 \right) ^{ n+1 } $. Because this happens on even intervals we can use subsequences to show that they will diverge. 
To start, let our subsequences be
\[
\lim_{ n \to \infty} a_{ 2n } = L \text{ and } \lim_{ n \to \infty} a_{ 2n+1 }=L
.\] 
or
\begin{align*}
	a_{ 2n }=\frac{ \left( -1 \right) ^{ 2n+1 }2n }{ 2n+\sqrt{ 2n} } \text{ and } a_{ 2n+1 }=\frac{ \left( -1 \right) ^{ 2n+2 }2n+1 }{ 2n+1+\sqrt{ 2n+1} }
.\end{align*}
Because we set our terms to always be even or odd, we can predict how $ \left( -1 \right) ^{ f+1 } $ will behave. So when,
\[
f=2n \implies \left( -1 \right) ^{ 2n+1 }=-1 \text{ and } f=2n+1 \implies \left( -1 \right) ^{ 2n+2 }=1
.\]
Rewriting our equations we get,
\[
a_{ 2n }=\frac{ -2n }{ 2n+\sqrt{ 2n} } \text{ and } a_{ 2n+1 }=\frac{ 2n+1 }{ 2n+1+\sqrt{ 2n+1} }
.\] 
Where we can use Dominance Theory to find our limits.
\[
\lim_{ n \to \infty} a_{ 2n }\approx \lim_{ n \to \infty} \frac{ -2n }{ 2n }=-1 \text{ and } \lim_{ n \to \infty} a_{ 2n+1 }\approx \lim_{ n \to \infty} \frac{ 2n }{ 2n }=1
.\] 
Because these two limits do not converge to the same value, we can say that the sequence will diverge.

\newpage
\section*{Question 2}%
\label{sec:Question 2}

\begin{gather*}
\text{ For a sequence, }a_n = \frac{ \left( 2n-1 \right) ! }{ \left( 2n+1 \right) ! }\\ 
\text{ form an argument on whether the sequence converges or diverges. }
\end{gather*}
We can start to rewrite the sequence using factorial laws,
\[
	\left( 2n-1 \right) ! \to \frac{ 2n! }{ 2n } \text{ and } \left( 2n+1 \right) ! \to 2n!\left( 2n+1 \right) 
.\] 
Simplifying our sequence we get,
\[
\frac{ \frac{ 2n! }{ 2n } }{ 2n!\left( 2n+1 \right)  }\to \frac{ 2n! }{ 2n!\left( 2n+1 \right) \left( 2n \right)  }\to \frac{1}{\left( 2n+1 \right) \left( 2n \right) }
.\] 
Which we can now take the limit of. 
\[
\lim_{ n \to \infty} \frac{1}{\left( 2n+1 \right) \left( 2n \right) }= 0
.\] 
\end{document}
