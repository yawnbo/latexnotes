\documentclass[a4paper]{article}

\usepackage[utf8]{inputenc}
\usepackage[T1]{fontenc}
\usepackage{textcomp}
\usepackage[english]{babel}
\usepackage{amsmath, amssymb}
\title{Partials cont.}
\author{yawnbo}
\date{\today}

\maketitle

\pdfsuppresswarningpagegroup=1

\begin{document}
\section{Cont of previous example}
\subsection{Example Eval $\int_{}^{} \frac{3x-9}{(x-1)(x+2)} $}%

This would be decompose as
\[
\frac{A}{x-1}+\frac{B_{1}}{x+2}+\frac{B_{2}}{(x+2)^2}
.\] 
\newpage

Decoposing this would get us to 
\[
3x-9=A(x+2)^2+B(x+2)(x-1)+C(x-1)
.\] 
Finding A can be done by letting $x=1$, so 
\[
-6=9A \to A=-\frac{2}{3}
.\] 
Worry about B in a second, for C let $x=-2$ so,
\[
-12=-3C \to C=4
.\] 
Now we need to find the B. For this we just let x equal ANYTHING ELSE. In this example we can use 0 because its not used yet.
\[
-9=A\cdot 4 + B(-2)+C(-1)
.\] 
Because we already know what A and C have to be we just plug those values in.
This just simplifies to 
\[
-9=-\frac{2}{3}\cdot 4 - 2B- 4 \to B=\frac{7}{6}
.\] 

This just sets up what we integrate for tomorrow. Long ass equation time

Again, $A=-\frac{2}{3}$ ; $B=\frac{3}{2}$ ; $C=5$ 

Go back to integral of
\[
\int_{}^{} \frac{3x-9}{(x-1)(x+2)^2}dx=\frac{A}{x-1}+\frac{B}{x+2}+\frac{C}{(x+2)^2} 
.\] 

\section{Another example}%
\label{sec:Another example}
\[
\int_{}^{} \frac{18}{(x+3)(x^2+9)}dx 
.\] 
Start with the usual and create partials. 
\[
18=A(x^2+9)+(Bx+C)(x+3)
.\] 
Now we use the method of undeterminated Coefficients. This means that we foil and distribute everything giving us
\[
18=Ax^2+9+Bx^2+3Bx+Cx+3c
.\] 
Where we can now gather like terms, so,
\[
=(A+B)x^2+(3B+C)x+9a+3c
.\] 
Which gives us a nicer quadratic. We now make the left side also quadratic
\[
0x^2+0x+18 = (A+B)x^2+(3B+C)x+9a+3c
.\] 

Where we can now set coefficients equal to each other,
\[
0=(A+B) = 3B+C \text{ and }9A+3C = 18
.\] 
\newpage
Which can be solved using a matrix that i have no idea how to put in using tex which is something with the mat snippet but idk

\subsection{Harder way of doing this}%
\label{sub:Harder way of doing this}
Given
\begin{gather*}
A+B=0 \\
3B+C=0\\
9A+3C=18 \\
\end{gather*}
We can just sub in B for everything by setting things equal but takes longer

\subsection{Back to normal integral}%
\label{sub:Back to normal integral}
Subbing in the values we just got gets us to 
\[
\int_{}^{} \frac{18}{(x+3)(x^2+9)}dx \int_{}^{} \frac{1}{x+3}dx+ \frac{-x+3}{x^2+9} 
.\] 
The first integral is simple and is just $\ln^{}(\|x+3\|)$ 

For the second one we can split it into
\[
\int_{}^{} -\frac{x}{x^2+9} + \int_{}^{} \frac{3}{x^2+9}  
.\] 
Wchich can be done by letting $U=x^2+9 \text{ and } \frac{du}{2}=xdx$
which gets us 
\[
-\frac{1}{2}\ln^{}(x^2+9)
.\] 
Where we can now work on teh other trig integral
\[
  \int_{}^{} \frac{3}{x^2+9}dx 
.\] 
Where we let $x=3tan\theta$ and  $dx=3 \sec^2\thetad\theta$
so
\[
\int_{}^{} d\theta = \theta 
.\] 
We can do the inverse of $x=3\tan\theta$ to find  $\theta = tan^{-1}\frac{x}{3}$ \\
So we can bring these back into the main integral to get
\[
\ln^{}(\|x+3\|)-\frac{1}{2}\ln^{}(\|x^2+9\|)+ \tan^{-1}(\frac{x}{3}) + C
.\] 
\end{document}
