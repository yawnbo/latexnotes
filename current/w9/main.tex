\documentclass[a4paper]{article}

\usepackage[utf8]{inputenc}
\usepackage[T1]{fontenc}
\usepackage{textcomp}
\usepackage[english]{babel}
\usepackage{amsmath, amssymb}
\title{Partials}
\author{yawnbo}
\date{\today}

\maketitle

\pdfsuppresswarningpagegroup=1

\begin{document}
\section{Partials}%
\label{sec:Partials}
\subsection{Begin}%
\label{sub:Begin}

Partials work when you have two polynomails in a proper fraction. It starts with factoring the denomintor into linear factors that can then be used to break apart the numerator. 

\subsubsection{Ex.}%
\label{sub:Ex.}

\[
\frac{1}{x^2-1}=\frac{1}{(x-1)(x+1)}=\frac{A}{x-1}+\frac{B}{x+1}
.\] 
In this example A would be $\frac{1}{2}$ and B would be $-\frac{1}{2}$ 

This can be proven by 
\begin{gather*}
\frac{1}{2(x-1)}-\frac{1}{2(x+1)} \\
\frac{2(x+1)-2(x-1)}{4(x-1)(x+1)} \\
\frac{x+1-x+1}{2(x-1)(x+1)} \\
\frac{1}{x^2-1}
\end{gather*}

\section{Partial integration}%
\label{sec:Partial integration}

\subsection{Evaluate $\int_{}^{} \frac{dx}{x^2-7x+10} $}%

We would just start by completing the square on a normal version,
\[
\frac{1}{x^2-7x+10}=\frac{1}{(x-5)(x-2)}
.\] 
Now using partial decomp we can use something like 
\[
  \frac{A}{x-5}+\frac{B}{x-2}
.\] 
Start by cancelling the denominators by multiplying by the LCD (least common denominator) or (x-5)(x-2)
\newpage
\[
  \to 1=A(x-2)+B(x-5)
.\] 
We can now sub in numbers that would get use 0 for the variables. When we put in 2 we get
\[
B(-3) \to B=-\frac{1}{3}
.\] 
And the opposite can be done to figure out the A which in this case would be $A=\frac{1}{3}$ 

\subsubsection{Onto the integral}
Now that we know that the polynomial is equal to 
\[
\frac{1}{3\left( x-5 \right)} -\frac{1}{3\left( x-2 \right)}
.\] 

Which can each be integrated by themselves using
\[
\frac{1}{k}\cdot \ln^{}(\|kx+b\|)+C
.\] 
So we can rewrite the integral as
\[
  \frac{1}{3}\ln^{}(\|x-5\|)-\frac{1}{3}\ln^{}(\|x-2\|)+C
.\] 

\section{Integrals again}%
\label{sec:}

\paragraph{Evaluate $\int_{}^{} \frac{x^2+2}{(x-1)(2x-8)(x+2)} $}
We can start by pulling out a 2 from 2x-8
\[
\frac{1}{2}\int_{}^{} \frac{x^2+2}{(x-1)(x-4)(x+2)} 
.\] 
Because this is proper we can start off our partial. 
\[
  \frac{1}{2}\left[\frac{A}{x-1}+\frac{B}{x-4}+\frac{C}{x+2}\right]
.\] 
Do the same thing and wipe out denominators while setting equal to original expression
\[
x^2+2=A(x-4)(x+2)+B(x-1)(x+2)+C(x-1)(x-4)
.\] 
Again make values disappear using an x value that cancels the others. Use 1 to start,
\[
3=A(-3)(3) \to A = \frac{-1}{3}
.\] 
Use $x=4$ to find B,
\[
18=B(3)(6) \to B=1
.\] 
Use $x=-2$ to find C,
\[
6=C(-3)(-6) \to C=\frac{1}{3}
.\] 

\newpage

\section{Integrate ex.}%
\label{sec:Integrate ex.}

\subsection{$\int_{}^{} \frac{x^3+1}{x^2-4} $}%
Because this is improper we can use long division which simpilfies to 
\[
\int_{}^{} (x+ \frac{4x+1}{x^2-4}) dx
.\] 
Check paper version for the long division. 
\subsubsection{Partial decomp}

Now we can decompose the fraction
\[
\frac{4x+1}{x^2-4}=\frac{A}{x-2}+\frac{B}{x+2}
.\] 
\[
4x+1=A(x+2)+B(x-2)
.\] 
This get us 
\begin{align*}
  a&= \frac{9}{4} \\
  b&= \frac{7}{4} \\
\end{align*}

Rewrite the integral as a partial now,
\[
\int_{}^{} \frac{9}{4(x-2)}+\frac{7}{4(x+2)}dx 
.\] 
Which can again be solved using the ln rule
\[
=\frac{9}{4}\ln^{}(\|x-2\|)+\frac{7}{4}\ln^{}(\|x+2\|)
.\] 
This can be brought down into the greater integral now so
\[
\frac{x^2}{2}+ \frac{9}{4}\ln^{}(\|x-2\|)+\frac{7}{4}\ln^{}(\|x+2\|)+C
.\] 

\section{Repeated Linear Factors}%
\label{sec:Repeated Linear Factors}

If we were to have somthing like $x^2+6x+9$ this would get us x+3 repeated twice. This can be canceled out by letting the exponents increase to their highest degree.
\subsection{Example Eval $\int_{}^{} \frac{3x-9}{(x-1)(x+2)} $}%

This would be decompose as
\[
\frac{A}{x-1}+\frac{B_{1}}{x+2}+\frac{B_{2}}{(x+2)^2}
.\] 
\newpage

Decoposing this would get us to 
\[
3x-9=A(x+2)^2+B(x+2)(x-1)+C(x-1)
.\] 
Finding A can be done by letting $x=1$, so 
\[
-6=9A \to A=-\frac{2}{3}
.\] 
Worry about B in a second, for C let $x=-2$ so,
\[
-12=-3C \to C=4
.\] 
Now we need to find the B. For this we just let x equal ANYTHING ELSE. In this example we can use 0 because its not used yet.
\[
-9=A\cdot 4 + B(-2)+C(-1)
.\] 
Because we already know what A and C have to be we just plug those values in.
This just simplifies to 
\[
-9=-\frac{2}{3}\cdot 4 - 2B- 4 \to B=\frac{7}{6}
.\] 

This just sets up what we integrate for tomorrow. Long ass equation time
\end{document}
